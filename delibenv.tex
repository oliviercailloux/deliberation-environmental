\RequirePackage{amsmath}
\RequirePackage[l2tabu, orthodox]{nag}
%\documentclass[smallextended,nospthms,natbib]{svjour3}
\documentclass[version=last, pagesize, twoside=off, bibliography=totoc, DIV=calc, fontsize=14pt, a4paper, french, english]{scrartcl}
\newcommand{\institute}[1]{}
\newcommand{\keywords}[1]{}
\newenvironment{acknowledgements}{
	\section*{Acknowledgements}
}{
}
%\makeatletter \let\cl@chapter\relax \makeatother
%\smartqed  % flush right qed marks, e.g. at end of proof
%INSTALL
\pdfgentounicode=1 %permits (with package glyphtounicode) to copy eg x ⪰ y iff v(x) ≥ v(y) from pdf to unicode data. 
\input{glyphtounicode}%TODO avoid warning when redefining → (and others) ; make it work for ℝ (U+211D) as well
\usepackage[T1]{fontenc}%encode resulting accented characters correctly in resulting PDF, permits copy-paste
\usepackage[utf8]{inputenc}
\usepackage{newunicodechar}%able to use e.g. → or ≤ in source
\usepackage{lmodern}%has more characters such as ligatures, permit copy from resulting PDF.
\usepackage{textcomp}%useful for redefining → and ¬ and … (otherwise seems to attempt to use \textrightarrow and the like but are not defined)
% solves bug in lmodern, https://tex.stackexchange.com/a/261188/
\DeclareFontShape{OMX}{cmex}{m}{n}{
  <-7.5> cmex7
  <7.5-8.5> cmex8
  <8.5-9.5> cmex9
  <9.5-> cmex10
}{}

\SetSymbolFont{largesymbols}{normal}{OMX}{cmex}{m}{n}
\SetSymbolFont{largesymbols}{bold}  {OMX}{cmex}{m}{n}
%warn about missing characters
\tracinglostchars=2

%REDAC
\usepackage{booktabs}
\usepackage{calc}
\usepackage{tabularx}

\usepackage{etoolbox} %for addtocmd, newtoggle commands
\newtoggle{LCpres}
\togglefalse{LCpres}

\usepackage{mathtools} %load this before babel!
	\mathtoolsset{showonlyrefs,showmanualtags}

\usepackage{natbib}%Package frenchb asks to load natbib before babel/frenchb

%\usepackage[super]{nth}%better use fmtcount! (loaded by datetime anyway; see below about pbl with warnings and package silence)
\usepackage{listings} %typeset source code listings
	\lstset{language=XML,tabsize=2,captionpos=b,basicstyle=\NoAutoSpacing}%NoAutoSpacing avoids space before colon or ?}%,literate={"}{{\tt"}}1, keywordstyle=\fontspec{Latin Modern Mono Light}\textbf, emph={String, PreparedStatement}, emphstyle=\fontspec{Latin Modern Mono Light}\textbf, language=Java, basicstyle=\small\NoAutoSpacing\ttfamily, frame=single, aboveskip=0pt, belowskip=0pt, showstringspaces=false
\usepackage[nolist,smaller,printonlyused]{acronym}%,smaller option produces warnings from relsize in some cases, it seems.% Note silence and acronym and hyperref make (xe)latex crash when ac used in section (http://tex.stackexchange.com/questions/103483/strange-packages-interaction-acronyms-silence-hyperref), rather use \section{\texorpdfstring{\acs{UE}}{UE}}.
\usepackage{fmtcount}
\usepackage[nodayofweek]{datetime}%must be loaded after the babel package. However, loading it after {nth} generates a warning from fmtcount about ordinal being already defined. Better load it before nth? (then we can remove the silence package which creates possible crashes, see above.) Or remove nth?
%\usepackage{xspace}%do we need this?
\nottoggle{LCpres}{
	\usepackage[textsize=small]{todonotes}
}{
}
\iftoggle{LCpres}{
	%remove pdfusetitle (implied by beamer)
	\usepackage{hyperref}
}{
% option pdfusetitle must be introduced here, not in hypersetup.
	\usepackage[pdfusetitle]{hyperref}
}
\nottoggle{LCpres}{
%seems like authblk wants to be later than hyperref, but sooner than silence
\usepackage{authblk}
\renewcommand\Affilfont{\small}
}{
}
\usepackage{silence}
\WarningFilter{newunicodechar}{Redefining Unicode character}
%breaklinks makes links on multiple lines into different PDF links to the same target.
%colorlinks (false): Colors the text of links and anchors. The colors chosen depend on the the type of link. In spite of colored boxes, the colored text remains when printing.
%linkcolor=black: this leaves other links in colors, e.g. refs in green, don't print well.
%pdfborder (0 0 1, set to 0 0 0 if colorlinks): width of PDF link border
%hidelinks or: colorlinks, linkcolor=black, citecolor=black, urlcolor={blue!80!black}
\hypersetup{breaklinks, bookmarksopen}
%add hyperfigures=true in hypersetup (already defined in article mode)
\iftoggle{LCpres}{
	\hypersetup{hyperfigures}
}{
}

%in Beamer, sets url colored links but does not change the rest of the colors (http://tex.stackexchange.com/questions/13423/how-to-change-the-color-of-href-links-for-real)
%\hypersetup{breaklinks,bookmarksopen,colorlinks=true,urlcolor=blue,linkcolor=,hyperfigures=true}
% hyperref doc says: Package bookmark replaces hyperref’s bookmark organization by a new algorithm (...) Therefore I recommend using this package.
\usepackage{bookmark}

% center floats by default, but do not use with float
% \usepackage{floatrow}
% \makeatletter
% \g@addto@macro\@floatboxreset\centering
% \makeatother
\nottoggle{LCpres}{
	\usepackage{enumitem} %follow enumerate by a string saying how to display enumeration
}{
}
\usepackage{ragged2e} %new com­mands \Cen­ter­ing, \RaggedLeft, and \RaggedRight and new en­vi­ron­ments Cen­ter, FlushLeft, and FlushRight, which set ragged text and are eas­ily con­fig­urable to al­low hy­phen­ation (the cor­re­spond­ing com­mands in LaTeX, all of whose names are lower-case, pre­vent hy­phen­ation al­to­gether). 
\usepackage{siunitx} %[expproduct=tighttimes, decimalsymbol=comma] ou (plus récent ?) [round-mode=figures, round-precision=2, scientific-notation = engineering]
\sisetup{detect-all, locale = FR, strict}% to detect e.g. when in math mode (use a math font) - check whether this makes sense with strict
\usepackage{braket} %for \Set
\usepackage{doi}

\usepackage{amsmath,amsthm}
\usepackage{amssymb}%for \mathbb{R} %includes amsfonts
\usepackage{bm}%“The \boldsymbol command is obtained preferably by using the bm package, which provides a newer, more powerful version than the one provided by the amsmath package. Generally speaking, it is ill-advised to apply \boldsymbol to more than one symbol at a time.” — AMS Short math guide. “If no bold font appears to be available for a particular symbol, \bm will use ‘poor man’s bold’” — bm
% \usepackage{dsfont} %for what?

\usepackage{environ}%for xdescwd command
%BUT see https://tex.stackexchange.com/questions/83798/cleveref-varioref-missing-endcsname-inserted for cleveref with french
\usepackage{cleveref}% cleveref should go "laster" than hyperref
%GRAPHICS
\usepackage{pgf}
\usepackage{pgfplots}
	\usetikzlibrary{babel, matrix, fit, plotmarks, calc, trees, shapes.geometric, positioning, plothandlers, arrows, shapes.multipart}
\pgfplotsset{compat=1.14}
\usepackage{graphicx}

\DeclareMathAlphabet\mathbfcal{OMS}{cmsy}{b}{n}

\graphicspath{{graphics/},{graphics-dm/}}
\DeclareGraphicsExtensions{.pdf}
\newcommand*{\IncludeGraphicsAux}[2]{%
	\XeTeXLinkBox{%
		\includegraphics#1{#2}%
	}%
}%

%HACKING
\usepackage{printlen}
\uselengthunit{mm}
% 	\newlength{\templ}% or LenTemp?
% 	\setlength{\templ}{6 pt}
% 	\printlength{\templ}
\usepackage{scrhack}% load at end. Corrects a bug in float package, which is outdated but might be used by other packages
\usepackage{mathrsfs}% for \mathscr
%see https://tex.stackexchange.com/questions/409212/size-substitution-with-fontsize-14
\DeclareFontFamily{U}{rsfs}{\skewchar\font127 }
\DeclareFontShape{U}{rsfs}{m}{n}{%
   <-6.5> rsfs5
   <6.5-8> rsfs7
   <8-> rsfs10
}{}

%Beamer-specific
%do not remove babel, which beamer uses (beamer uses the \translate command for the appendix); but french can be removed.
\iftoggle{LCpres}{
	\usepackage{appendixnumberbeamer}
	\setbeamertemplate{navigation symbols}{} 
	\usepackage{preamble/beamerthemeParisFrance}
	\usefonttheme{professionalfonts}
	\setcounter{tocdepth}{10}
	%From: http://tex.stackexchange.com/questions/168057/beamer-with-xelatex-on-texlive2013-enumerate-numbers-in-black
%I don’t think it’s useful to submit this as a bug: nothing has been solved since March, 2015. See: https://bitbucket.org/rivanvx/beamer/issues?status=resolved.

\setbeamertemplate{enumerate item}
{
  \begin{pgfpicture}{-1ex}{-0.65ex}{1ex}{1ex}
    \usebeamercolor[fg]{item projected}
    {\pgftransformscale{1.75}\pgftext{\normalsize\pgfuseshading{bigsphere}}}
    {\pgftransformshift{\pgfpoint{0pt}{0.5pt}}
      \pgftext{\usebeamercolor[fg]{item projected}\usebeamerfont*{item projected}\insertenumlabel}}
  \end{pgfpicture}%
}

\setbeamertemplate{enumerate subitem}
{
  \begin{pgfpicture}{-1ex}{-0.55ex}{1ex}{1ex}
    \usebeamercolor[fg]{subitem projected}
    {\pgftransformscale{1.4}\pgftext{\normalsize\pgfuseshading{bigsphere}}}
    \pgftext{%
      \usebeamercolor[fg]{subitem projected}%
      \usebeamerfont*{subitem projected}%
      \insertsubenumlabel}
  \end{pgfpicture}%
}

\setbeamertemplate{enumerate subsubitem}
{
  \begin{pgfpicture}{-1ex}{-0.55ex}{1ex}{1ex}
    \usebeamercolor[fg]{subsubitem projected}
    {\pgftransformscale{1.4}\pgftext{\normalsize\pgfuseshading{bigsphere}}}
    \pgftext{%
      \usebeamercolor[fg]{subsubitem projected}%
      \usebeamerfont*{subitem projected}%
      \insertsubsubenumlabel}
  \end{pgfpicture}%
}


}{
}
% \newcommand{\citep}{\cite}%Better: leave natbib.
% \setbeamersize{text margin left=0.1cm, text margin right=0.1cm} 
% \usetheme{BrusselsBelgium}%no, replace with paris
%\usetheme{ParisFrance}, no, usepackage better!
% Tex Gyre takes too much space, replace with Latin Modern Roman / Sans / Mono.
% Difference when loading explicitly Latin Modern Sans (compared to not using \setsansfont at all):
% the font LMSans17-Regular appears in the document;
% the title of the slides appears differently;
% it does not say (in the log file):
% > LaTeX Font Info:    Font shape `EU1/lmss/m/it' in size <10.95> not available
% > (Font)              Font shape `EU1/lmss/m/sl' tried instead on input line 85.
% > LaTeX Font Info:    Try loading font information for EU1+lmtt on input line 85.

%tikzposter-specific
%remove \usepackage{ragged2e}: causes 1=1 to be printed in the middle of the poster. (Anyway prints a warning about those characters being missing.)
%put [french, english] options next to \usepackage{babel} to avoid warning of 

\newcommand{\R}{ℝ}
\newcommand{\N}{ℕ}
\newcommand{\Z}{ℤ}
\newcommand{\card}[1]{\lvert{#1}\rvert}
\newcommand{\powerset}[1]{\mathscr{P}(#1)}%\mathscr rather than \mathcal: scr is rounder than cal (at least in XITS Math).
%powerset without zero
\newcommand{\powersetz}[1]{\mathscr{P}_\emptyset(#1)}
\newcommand{\suchthat}{\;\ifnum\currentgrouptype=16 \middle\fi|\;}
%\newcommand{\Rplus}{\reels^+\xspace}

\AtBeginDocument{%
	\renewcommand{\epsilon}{\varepsilon}
% we want straight form of \phi for mathematics, as recommended in UTR #25: Unicode support for mathematics.
%	\renewcommand{\phi}{\varphi}
}

% with amssymb, but I don’t want to use amssymb just for that.
% \newcommand{\restr}[2]{{#1}_{\restriction #2}}
%\newcommand{\restr}[2]{{#1\upharpoonright}_{#2}}
\newcommand{\restr}[2]{{#1|}_{#2}}%sometimes typed out incorrectly within \set.
%\newcommand{\restr}[2]{{#1}_{\vert #2}}%\vert errors when used within \Set and is typed out incorrectly within \set.
\DeclareMathOperator*{\argmax}{arg\,max}
\DeclareMathOperator*{\argmin}{arg\,min}


%Decision Theory (MCDA and SC)
\newcommand{\allalts}{\mathscr{A}}
\newcommand{\allcrits}{\mathscr{C}}
\newcommand{\alts}{A}
\newcommand{\dm}{i}
\newcommand{\allF}{\mathscr{F}}
\newcommand{\allvoters}{\mathscr{N}}
\newcommand{\voters}{N}
\newcommand{\allprofs}{\boldsymbol{\mathcal{R}}}
\newcommand{\prof}{\boldsymbol{R}}
\newcommand{\linors}{\mathscr{L}(\allalts)}

%Deliberated Judgment
\newcommand{\allargs}{S^*}
\newcommand{\args}{S}
\newcommand{\ar}{s}
\newcommand{\ileadsto}{⇝}
\newcommand{\ibeatse}{⊳_\exists}
\newcommand{\nibeatse}{⋫_\exists}
\newcommand{\ibeatsst}{⊳_\forall}
\newcommand{\nibeatsst}{⋫_\forall}
\newcommand{\mleadsto}[1][\eta]{⇝_{#1}}
\newcommand{\mbeats}[1][\eta]{⊳_{#1}}
\newcommand{\ibeatseinv}{⊳_\exists^{-1}}

%Logic
\newcommand{\ltru}{\texttt{T}}
\newcommand{\lfal}{\texttt{F}}


\newcommand{\commentOC}[1]{{\small\color{blue}$\big[$OC: #1$\big]$}}
\newcommand{\commentOCf}[1]{{\small\color{blue}{\selectlanguage{french}$\big[$OC : #1$\big]$}}}
\newcommand{\commentYM}[1]{{\small\color{red}{\selectlanguage{french}$\big[$YM : #1$\big]$}}}
\newcommand{\innote}[1]{{\scriptsize{#1}}}
%Or: \todo[color=green!40]

%this probably requires outdated float package, see doc KomaScript for an alternative.
% \newfloat{program}{t}{lop}
% \floatname{program}{PM}

%definition, theorem, lemma, example environments, qed trickery are only needed in article mode (not Beamer)
\nottoggle{LCpres}{
%style is plain by default (italic text)
	\newtheorem{definition}{Definition}
	\newtheorem{theorem}{Theorem}
%no italic: expected.
%http://tex.stackexchange.com/questions/144653/italicizing-of-amsthm-package
	\newtheorem{lemma}{Lemma}
	\newtheorem{condition}{Condition}
%\crefname{axiom}{axiom}{axioms}%might be needed for workaround bug in cref when defining new theorems?

%\ifdefined\theorem\else
%\newtheorem{theorem}{\iflanguage{english}{Theorem}{Théorème}}
%\fi

\theoremstyle{remark}
	\newtheorem{examplex}{Example}
	\newtheorem{remarkx}{Remark}

%trickery allowing use of \qedhere and the like.
\newenvironment{example}{
	\pushQED{\qed}\renewcommand{\qedsymbol}{$\triangle$}\examplex
}{
	\popQED\endexamplex
}
\newenvironment{remark}{
	\pushQED{\qed}\renewcommand{\qedsymbol}{$\triangle$}\remarkx
}{
	\popQED\endremarkx
}
}{
}
\crefname{examplex}{example}{examples}% I wonder why this is unnecessary in case of singular
\crefname{condition}{condition}{conditions}
\makeatletter
\cref@addlanguagedefs{french}{%
	\crefname{examplex}{exemple}{exemples}%
}
\makeatother

%trickery allowing use of \qedhere and the like.
%\newcommand{\xqed}[1]{%
%    \leavevmode\unskip\penalty9999 \hbox{}\nobreak\hfill
%    \quad\hbox{#1}}
%\AtEndEnvironment{example}{
%	\xqed{$\triangle$}%
%}
%\AtEndEnvironment{proof}{
%	\qed%
%}

%which line breaks are chosen: accept worse lines, therefore reducing risk of overfull lines. Default = 200
%\tolerance=2000
%accept overfull hbox up to...
%\hfuzz=2cm
%reduces verbosity about the bad line breaks
%\hbadness 5000
%sloppy sets tolerance to 9999
%\apptocmd{\sloppy}{\hbadness 10000\relax}{}{}

\bibliographystyle{abbrvnat}
%or \bibliographystyle{apalike} for presentations?

%doi package uses old-style dx.doi url, see 3.8 DOI system Proxy Server technical details, “Users may resolve DOI names that are structured to use the DOI system Proxy Server (http://doi.org (preferred) or http://dx.doi.org).”, https://www.doi.org/doi_handbook/3_Resolution.html
\makeatletter
\patchcmd{\@doi}{dx.doi.org}{doi.org}{}{}
\makeatother

% WRITING
%\newcommand{\ie}{i.e.\@\xspace}%to try
%\newcommand{\eg}{e.g.\@\xspace}
%\newcommand{\etal}{et al.\@\xspace}
\newcommand{\ie}{i.e.\ }
\newcommand{\eg}{e.g.\ }
\newcommand{\mkkOK}{\checkmark}%\color{green}{\checkmark}
\newcommand{\mkkREQ}{\ding{53}}%requires pifont?%\color{green}{\checkmark}
\newcommand{\mkkNO}{}%\text{\color{red}{\textsf{X}}}

\makeatletter
\newcommand{\boldor}[2]{%
	\ifnum\strcmp{\f@series}{bx}=\z@
		#1%
	\else
		#2%
	\fi
}
\newcommand{\textstyleElProm}[1]{\boldor{\MakeUppercase{#1}}{\textsc{#1}}}
\makeatother
\newcommand{\electre}{\textstyleElProm{Électre}\xspace}
\newcommand{\electreIv}{\textstyleElProm{Électre Iv}\xspace}
\newcommand{\electreIV}{\textstyleElProm{Électre IV}\xspace}
\newcommand{\electreIII}{\textstyleElProm{Électre III}\xspace}
\newcommand{\electreTRI}{\textstyleElProm{Électre Tri}\xspace}
% \newcommand{\utadis}{\texorpdfstring{\textstyleElProm{utadis}\xspace}{UTADIS}}
% \newcommand{\utadisI}{\texorpdfstring{\textstyleElProm{utadis i}\xspace}{UTADIS I}}

%TODO
% \newcommand{\textstyleElProm}[1]{{\rmfamily\textsc{#1}}} 

\newcommand{\txtlaxiom}{$\alllang$\hyp{}axiom}
\newcommand{\txtlaxioms}{$\alllang$\hyp{}axioms}
\newcommand{\laxiomatisation}{$\alllang$\hyp{}axiomatisation}
\newcommand{\laxiomatization}{$\alllang$\hyp{}axiomatization}
%bold. Requires special command for math bold cal!
\newcommand{\laxiomatisationbd}{$\mathbfcal{L}$\textbf{\hyp{}axiomatisation}}
%\newcommand{\laxiomatisationtitle}{\texorpdfstring{$\alllang$\hyp{}Axiomatisation}{L\hyp{}Axiomatisation}}%Does not work because of AAMAS font switching command (tries to typeset it using ptm font), so we should use something like {\usefont{EU1}{lmr}{m}{n}©} but manually select the right font size and weight (which I ignore).
\newcommand{\laxiomatisationtitle}{\texorpdfstring{{\large$\mathcal{L}$}\hyp{}Axiomatisation}{ℒ\hyp{}Axiomatisation}}
%\commentUE{Should we maybe write $\ell$-axiom rather than l-axiom, for better readability (to avoid ell being read as one)? Or we could even write $\mathcal{L}$-axiom, to have an explicit reference to the language in which the axiom is expressed?}

\newunicodechar{ℕ}{\mathbb{N}}
\newunicodechar{ℝ}{\mathbb{R}}
\newunicodechar{−}{\ifmmode{-}\else\textminus\fi}
\newunicodechar{≠}{\ensuremath{\neq}}
\newunicodechar{≤}{\ensuremath{\leq}}
\newunicodechar{≥}{\ensuremath{\geq}}
\newunicodechar{≻}{\succ}
\newunicodechar{⊁}{\nsucc}
\newunicodechar{▷}{\triangleright}
\newunicodechar{⋫}{\ntriangleright}
\newunicodechar{→}{\ifmmode\rightarrow\else\textrightarrow\fi}
\newunicodechar{⇒}{\ensuremath{\Rightarrow}}
\newunicodechar{⇏}{\ensuremath{\nRightarrow}}
\newunicodechar{⇔}{\ensuremath{\Leftrightarrow}}
\newunicodechar{∪}{\cup}
\newunicodechar{∩}{\cap}
\newunicodechar{∧}{\land}
\newunicodechar{∨}{\lor}
\newunicodechar{¬}{\ifmmode\lnot\else\textlnot\fi}
\newunicodechar{…}{\ifmmode\ldots\else\textellipsis\fi}
\newunicodechar{×}{\ifmmode\times\else\texttimes\fi}
\newunicodechar{γ}{\ensuremath{\gamma}}
\newunicodechar{□}{\Box}


\newlength{\GraphsNodeSep}
\setlength{\GraphsNodeSep}{7mm}

% MCDA Drawing Sorting
\newlength{\MCDSCatHeight}
\setlength{\MCDSCatHeight}{6mm}
\newlength{\MCDSAltHeight}
\setlength{\MCDSAltHeight}{4mm}
%separation between two vertical alts
\newlength{\MCDSAltSep}
\setlength{\MCDSAltSep}{2mm}
\newlength{\MCDSCatWidth}
\setlength{\MCDSCatWidth}{3cm}
\newlength{\MCDSEvalRowHeight}
\setlength{\MCDSEvalRowHeight}{6mm}
\newlength{\MCDSAltsToCatsSep}
\setlength{\MCDSAltsToCatsSep}{1.5cm}
\newcounter{MCDSNbAlts}
\newcounter{MCDSNbCats}
\newlength{\MCDSArrowDownOffset}
\setlength{\MCDSArrowDownOffset}{0mm}

\tikzset{/Graphs/dot/.style={
	shape=circle, fill=black, inner sep=0, minimum size=1mm
}}
\tikzset{/MC/D/S/alt/.style={
	shape=rectangle, draw=black, inner sep=0, minimum height=\MCDSAltHeight, minimum width=2.5cm, anchor=north east
}}
\tikzset{MC/D/S/pref/.style={
	shape=ellipse, draw=gray, thick
}}
\tikzset{/MC/D/S/cat/.style={
	shape=rectangle, draw=black, inner sep=0, minimum height=\MCDSCatHeight, minimum width=\MCDSCatWidth, anchor=north west
}}
\tikzset{/MC/D/S/evals matrix/.style={
	matrix, row sep=-\pgflinewidth, column sep=-\pgflinewidth, nodes={shape=rectangle, draw=black, inner sep=0mm, text depth=0.5ex, text height=1em, minimum height=\MCDSEvalRowHeight, minimum width=12mm}, nodes in empty cells, matrix of nodes, inner sep=0mm, outer sep=0mm, row 1/.style={nodes={draw=none, minimum height=0em, text height=, inner ysep=1mm}}
}}

% GUI
\tikzset{/GUI/button/.style={
	rectangle, very thick, rounded corners, draw=black, fill=black!40%, top color=black!70, bottom color=white
}}

% Beliefs
\newlength{\BDNodeSep}
\setlength{\BDNodeSep}{5mm}
\newlength{\BDDecLength}
\setlength{\BDDecLength}{3mm}
\newlength{\BDDecWidth}
\setlength{\BDDecWidth}{1mm}
\tikzset{/Beliefs/attacker/.style={
	shape=rectangle, draw, minimum size=8mm
}}
\tikzset{/Beliefs/supporter/.style={
	shape=circle, draw
}}
\tikzset{/Beliefs/undefeated/.style={
	shape=circle, draw
}}
\tikzset{/Beliefs/defeated/.style={
	shape=rectangle
}}
\tikzset{/Beliefs/attack/.style={
%TODO à changer, mais crée warnings à la compilation avec version 3.0.0
%	arrows=-{Stealth}
	arrows=-{>}
}}
\tikzset{/Beliefs/attackst/.style={
	arrows=-{>}, double
}}
\tikzset{/Beliefs/attackun/.style={
	arrows=-{>},
	decorate,
	decoration={coil, segment length=1.5mm, aspect=0},
%fails with ! Dimension too large. Too hard for TeX, see http://tex.stackexchange.com/questions/4404/tikz-multiple-decorations-on-a-path
%	postaction={
%		decorate,
%		decoration={
%			markings, mark=at position 0.5 with {
%				\node[right] {unstable};
%			}
%		}
%	},
}}
\tikzset{/Beliefs/nattack/.style={
	arrows=-{>},
	decoration={
		markings, mark=at position 0.5 with {
			\draw[-] ++ (-1mm, 1mm) -- (1mm, -1mm);
		}
	},
	postaction={decorate},
}}
\tikzset{/Beliefs/decisive/.style={
	alias=thisone,
	append after command={
      (thisone.south) edge[draw] +(0, -\BDDecLength) ++(0, -\BDDecLength) edge[draw] ++(-\BDDecWidth, 0) edge[draw] ++(\BDDecWidth, 0)
	}
}}

\newcommand{\tikzmark}[1]{%
	\tikz[overlay, remember picture, baseline=(#1.base)] \node (#1) {};%
}


\DeclareAcronym{AMCD}{short=AMCD, long={Aide Multicritère à la Décision}}
\DeclareAcronym{AR}{short=AR, long={Argumentative Recommender}}
\DeclareAcronym{DA}{short=DA, long={Decision Analysis}}
\DeclareAcronym{DJ}{short=DJ, long={Deliberated Judgment}}
\DeclareAcronym{DM}{short=DM, long={Decision Maker}}
\DeclareAcronym{DMV}{short=DMV, long={Deliberative Monetary Valuation}}
\DeclareAcronym{DP}{short=DP, long={Deliberated Preference}}
\DeclareAcronym{ÉMD}{short=ÉMD, long={Évaluation Monétaire Délibérative}}
\DeclareAcronym{MAVT}{short=MAVT, long={Multiple Attribute Value Theory}}
\DeclareAcronym{MCDA}{short=MCDA, long={Multicriteria Decision Aid}}
\DeclareAcronym{MIP}{short=MIP, long={Mixed Integer Program}}
\DeclareAcronym{SPM}{short=SPM, long={Stated Preference Methods}}
\DeclareAcronym{WTP}{short=WTP, long={Willingness To Pay}}



%\journalname{}

\begin{document}
\title{Deliberation and environmental decision}
\author{Olivier Cailloux \and Yves Meinard}
\institute{
	Olivier Cailloux 
	\and
	Yves Meinard 
	\at 
	Université Paris-Dauphine, \\
	PSL Research University, \\
	CNRS, \\
	LAMSADE\\
	75016 PARIS, FRANCE\\
	\email{olivier.cailloux@dauphine.fr}
}
\makeatletter
	\hypersetup{
		pdfsubject={preference},
		pdfkeywords={decision aid, justification, empirical validation, methodology}
	}
\makeatother
\maketitle

\keywords{decision aid, justification, empirical validation, methodology}

\begin{abstract}
A voluminous literature addresses the weaknesses of stated preference methods used to value non-market environmental goods and services, such as contingent valuation and choice experiment. Deliberative monetary valuation (DMV) has emerged as a prominent alternative to these methods. It combines deliberative institutions with preference elicitation. Despite an anchorage in an extensive philosophical literature on deliberative democracy, the theoretical foundations of DVM are underinvestigated.  A noteworthy exception is \citet{bartkowski_beyond_2018}'s effort to use \citet{sen_idea_2009}'s philosophical views to elaborate such theoretical foundations. The present article pursue this theoretical effort by addressing two issues left unanswered by the above contribution: the first issue is to establish the precise role that deliberation is expected to play in DMV and, more broadly, in environmental decision-making (we term this \emph{the deliberation/decision-making issue}); the second issue is to define the role that economists and consultants involved in the proceedings of deliberation are supposed to play (we term this \emph{the deliberation/decision-aiding issue}). In order to address these issues, we use a formal framework introduced by Cailloux and Meinard, designed to capture the stance that an individual has on a given issue once s/he has performed a deliberation: his/her ``deliberated judgment''. This framework allows to identify a key empirical question lying at the heart of \emph{the deliberation/decision-making issue}: is there a consensus on how the outcomes of the deliberation are to be translated into a collective decision? If the answer is ``no'', then DMV is irrelevant for decision-making purposes. Our framework also allows to identify how empirical investigations should proceed to answer this question in real-life situations. When it comes to \emph{the deliberation/decision-making issue}, our framework shows that claiming that economists/consultants implementing DMV are neutral observers is untenable. Implementing DMV or any other deliberative approach implies endorsing a normative stance, which our framework allows to characterize.
\end{abstract}

\section{Introduction}
A voluminous literature now addresses the practical, methodological and philosophical weaknesses of the standard economic methods used to value non-market environmental goods and services \citep{meinard_ethical_2016}. These standard approaches are stated preference methods such as contingent valuation and choice experiment, which are used to elicit individual willingness to pay (WTP) and then aggregate it through cost-benefit analysis. As aptly summurized by \citet{bartkowski_beyond_2018}, the most important criticisms addressed at these methods raise two political and ethical concerns. The first one is articulated in terms of the so-called consumer-citizen dichotomy, and refers to the idea that methods based on stated preferences would discourage respondents from acting as citizens, in particular by taking into account the social implications of their choices and statements \citep{soma_representing_2014, vatn_institutional_2009}. The second political and ethical concern is that stated preferences methods reduce respondent's expressions to one number, whereas understanding the reasons underlying their statements or choices is, according to some authors like \citet{sen_environmental_1995}, perhaps even more relevant and important from a political point of view.

A prominent alternative to these standard stated preference methods has emerged in the past 10 years: deliberative monetary valuation (DMV) \citep{spash_deliberative_2007,bartkowski_economic_2017}. These methods combine deliberative institutions, such as focus groups, with preference elicitation or choice experiments. The emerging literature on these methods suggests that they have important strengths as compared with standard stated preference methods, at several levels. At a very practical level, it appears that respondents find it easier to make sense of these methods, and are less likely to refuse to answer \citep{lienhoop_contingent_2007,szabo_reducing_2011}. At a philosophical level, the very label of these methods and the deliberative institutions that they use refer to the notion of deliberative democracy, which has been extensively investigated since the 1970 and now largely dominates the political philosophical scene \citep{chappell_deliberative_2012}.

Despite this anchorage in the philosophy of deliberative democracy, the theoretical foundations of DVM are arguably underinvestigated \citep{bartkowski_economic_2017,bartkowski_beyond_2018,bunse_what_2015,kenter_what_2015}. A noteworthy exception is \citet{bartkowski_beyond_2018}'s pioneering effort to use \citet{sen_idea_2009}'s philosophical views to elaborate such theoretical foundations. Although \citet{bartkowski_beyond_2018} articulate convincing solutions to some important issues in the foundations of these methods, certainly because all the problems cannot be solved in a single article, they also leave aside some prominent theoretical problem. Our aim in this article is to tackle part of these remaining problems. More specifically, we tackle two sets of questions:
\begin{itemize}
\item A first set of questions has to do with the precise role that deliberation is expected to play in DMV and, more broadly, in environmental decision-making: what deliberation is precisely expected to do, when can one considers that there have been \emph{enough deliberation}, when can one consider that deliberation has done its job? Let us call this set of questions \emph{the deliberation/decision-making issue}.
\item A second set of questions has to do with the role of economists and consultants involved in the proceedings of deliberation: are they here simply to observe what happens, are they supposed to actively interact with participants, what are the ethical stakes of their intervention? Let us call this set of questions \emph{the deliberation/decision-aiding issue}.
\end{itemize}
The scope of these questions is clearly larger than DVM, and even larger than \emph{environmental} issues. However, has it happens, the pioneering research on DVM and associated methological problems have been developped in the context of environmental issues. In the present article, we pursue this dynamics, while emphasizing that our arguments could be adapted to other contexts.
In a first part, we argue that the extant literature does not provide convincing answers to these questions. We then take advantage of a formal framework introduced by Cailloux and Meinard to articulate an approach to solve these problems.

\section{Two blindspots of deliberative approaches}
In this first, relatively short part, we explain why we claim that the current literature on the theoretical foundations of DVM, which in effect is largely limited to \citet{bartkowski_beyond_2018}'s contribution, does not convincingly tackle \emph{the deliberation/decision-making issue} and \emph{the deliberation/decision-aiding issue}. We emphasize at the outset that, though our aim is to point problems that \citet{bartkowski_beyond_2018} did not solve, our contribution should not be seen as a criticism of the latter article, but rather as a sympathetic complement.

Authors on DVM vary in the answer they give to a very basic question: should deliberation give rise to a consensus?

Some authors such as \citet{vatn_institutional_2009} claim that the point of exchanges of arguments and interactions during deliberation is to reach a mutual understanding of the environmental problem that participants face and to identify a common solution to this problem. This vision of the point of deliberation is endorsed by versions of DVM where the goal is to reach a consent among all the participants in the form of a single social WTP (e.g. \citet{orchard-webb_deliberative_2016}).

By contrast, other studies (most DVM studies, according to \citet{bunse_what_2015}) use deliberative aspects as a preliminary stage, before eliciting and aggregating individual WTP in an approach quite similar to the one deployed by standard stated preference methods. In such approaches, deliberative institutions and settings are conceived as tools enabling respondents to gather information about the issues on which they have to take a stance, to take time to think through the stakes, and to engage in a process of constructing their preference \citep{braga_preference_2005}. Here, the point of deliberation is to improve the respondents' ability to identify their own preferences and to express them, and the usefulness of deliberation for that purpose is attested by the fact that deliberation appears to improve the robustness of responses to WTP surveys, to enable respondents to feel ``more confident regarding what should be valued and why'' (\citet{svedsater_economic_2003}, p. 125, cited by \citet{bartkowski_beyond_2018}) and to reduce the number of protest non-responses \citep{szabo_reducing_2011}.

The idea underlying the first kind of approach, according to which deliberation is conducive to consensus, is often presented as a basic tenet of deliberative democacy theory \citep{wilson_discourse-based_2002}. \citet{bartkowski_beyond_2018} contrast this consensus tenet with the notion of a ``plurality of impartial reasons'' most prominently championed by \citet{sen_idea_2009}, which they present as a recognition that there can be deep moral disagreeements \citep{dryzek_deliberative_2013}. This stance leads \citet{bartkowski_beyond_2018} to defend an approach to DVM where consensus is neither expected nor required. Such an approach might seem more open-minded than the doctrine that deliberation unavoidably generates consensus, and it accommodates doubts such as those voiced by \citet{elster_sour_1983} that ``unanimity, even if sincere, could easily be spurious in the sense of deriving from conformity rather than from rational conviction'', echoing the fact that DMV, like any deliberative setting, is vulnerable to exclusion and power dynamics \citep{volker_exploring_2016,vargas_background_2016,vargas_problem_2017}.

This approach however creates two important problems.

First, it evades rather than solving the question: when can one consider that there has been \emph{enough} deliberation? In a vision in which there is consensus if and only if there is deliberation, observing consensus is necessary and sufficient to be confident that there has been enough deliberation of the right sort. As soon as one acknowledges that consensus can also be generated by brute force and that deliberation can leave room to dissensus, observing a consensus looses its relevance. How, then, can one assess the deliberative credentials of a given DMV? \citet{bartkowski_beyond_2018} neither raise nor answer this question.

Second, as \citet{bartkowski_beyond_2018} themselves notice, in an approach such as theirs, an aggregation mechanism is needed \emph{after} the deliberative step. Unfortunately, \citet{bartkowski_beyond_2018} appear to have lacked space to properly discuss the aggregation issue (as illustrated, for example, by their reference to ``the question whether WTP elicitation is the right way to aggregate individual preferences'', which confuses elicitation and aggregation, and leads to a discussion of the former rather than the latter in a paragraph 4.5 supposedly devoted to the latter). This leads them to advocate the use of additive aggregation of individual WTP, ``for lack of better alternatives''. \citet{bartkowski_beyond_2018}'s reference to the notion of a ``plurality of impartial reasons'' however contains \emph{in nuce} all the elements to develop a stronger understanding of the aggregation issue, epitomized by there own contention that ``deliberation based on communicative rationality aims to reach a workable agreement,’ which involves that participants agree on a course of action without requiring a convergence of preferences supporting the course of action\citep{dryzek_deliberative_2002}''. This sentense points a contrast that the rest of the article ignores, between two levels at which a consensus might exist or fail to exist: on the one hand, the level of preferences, on the other hand, the level of the decision to be collectively made on the issue at hand. Indeed, in the rest of their reasoning, \citet{bartkowski_beyond_2018} talk about possible disagreements at the level of individual preferences but, apart from stating that ``CBA  does not replace a political decision-making process, it can only inform such a process'', they do not investigate the question of possible disagreements about what is done with preferences once elicited. A natural suggestion would however be that, if one has good reasons to champion the usage of deliberation to foster preference formation and respondent's understanding of preference elicitation, then for the very same reasons one should also champion the usage of deliberation in the choice of aggregation processes. In concrete terms, in this approach, the usage of additive aggregation or any other aggregation rule, and the motivations underlying them (such as preference utilitarianism in the case of additive aggregation of preference) should be among the things about which respondents deliberate.

We therefore argue that there is a need to articulate much more clearly what the precise role of deliberation is supposed be in DMV: this is the \emph{deliberation/decision-making issue}.

This first blindspot of the DMV literature is associated with a second one. The literature on deliberative democracy, and especially the writtings of its founding fathers such as \citet{rawls_political_2005} and \citet{habermas_faktizitat_1992}, is largely devoted to an investigation of the \emph{status} that philosophers are assumed to enjoy when they state the tenets of moral or political theories \citep{meinard_du_2014}. By contrast, the literature on DVM appears strickingly silent about the status of economists and consultants involved in the proceedings of deliberation as part of DVM, as if they were transparent and neutral observers. This stands in stark contrast, not only with the philosophical literature on deliberation, but also with some groundbreaking contributions to the literature on decision-aiding methodology and practice, emphazising the importance of decision support interactions that consist, for the analyst, in ensuring that the aided individuals understand and accept the reasoning on which decision-aid is based \citep{roy_multicriteria_1996}. This is what we term the \emph{deliberation/decision-aiding issue}.

These two blindspots set the agenda for our work in the remainder of this article. In the next section, we start by presenting a formal framework, which is a simplified version of a framework developped by Cailloux and Meinard, and we explain how this framework allows to articulate the \emph{deliberation/decision-making} and \emph{deliberation/decision-aiding} issues clearly and transparently.

\section{A formal framework}
Let us start by spelling out what a formal framwork should capture, in order to be useful for our purposes. We want to carve out a formal representation of the stance that an individual has on a given issue once he has performed a deliberation. Following Cailloux and Meinard, let us call this the individual's \ac{DJ}. This reference to deliberation captures, in our view, two central ideas:
\begin{itemize}
\item The first idea is that deliberated judgments are the result of a careful examination of arguments and counter-arguments: through deliberation, the individual gathers new information, he learns about the perspectives of other people, he takes the time to think about all these elements, which are all arguments for or against this or that stance. This idea echoes the approach to the notion of rationality developped most prominently by \citet{habermas_theorie_1981}. In this approach, actions, attitudes or utterances can be termed “rational” so long as the actor(s) performing or having them can account for them, explain them and use arguments and counter-arguments to withstand criticisms that other people could raise against them. Variants of this vision of rationality play a key role in other prominent philosophical frameworks, such as \citeauthor{scanlon_what_2000}’s \citeyearpar{scanlon_what_2000} and \citeauthor{sen_idea_2009}’s \citeyearpar{sen_idea_2009}. (Using Sen's vocabulary, \citet{bartkowski_beyond_2018} talk about ``reasonableness'' to stress the interpersonal aspect of this idea. This choice of vocabulary can create confusions, because the term ``reasonable'' is more classically associated with \citet{rawls_political_2005}, who understand this term in a different sense.) For simplicity's sake, we will simple talk about ``rationality'' to refer to this first idea.
\item The second idea is that deliberated judgments are nevertheless the individual's own judgments, in the sense that they do not reflect the application of any exogenous criterion. This second idea will be nicknamed ``anti-paternalism'' in what follows.
\end{itemize}

Let us attempt to formalize these ideas. Let us assume that a decision maker is given--call her $i$--together with a topic $T$, about which $i$ wants to make up her mind using arguments. Elaborate typologies of the kinds of arguments involved in environmental deliberations have been developped in the literature, for example by \citet{chateauraynaud_contrainte_2007}. By using here a very abstract notion of argument, we aim to encompass all the diversity included in such typologies. Let us then define $\allargs$ as a set that contains all the arguments that one can make use of when trying to make up one’s mind about $T$.

Using these notions, we want to define $i$’s stance towards the topic once a deliberation has allowed her to ponder all the arguments possibly relevant to the situation. To define this \ac{DJ}, we first need to capture the stance that $i$ takes towards arguments: given an argument $\ar$, does she endorse it or not? She can change her mind concerning some arguments because of reasons independent of her endeavor to tackle the problem she addresses, for example depending on her mood. More interestingly, $i$ will possibly change her mind when confronted with new arguments as she struggles to make up her mind about the issue at hand during the deliberation.

To formalize these ideas, let us define a set of possible perspectives $P$ that $i$ can be exposed to or have towards the topic $T$. That set represents elements that can influence $i$’s stances. The notion of perspective introduced here is purportedly rather abstract, and can encompass all sorts of aspects of situations that can lead an individual to change her mind. However, as will be clarified shortly, in the bulk of this article we will be mainly concerned with changes of perspective that correspond to the fact that our individual happens to be exposed to different sets of arguments. In real-life decision situations, such perspectival changes unfold in time. But in the formalism introduced here, it will not be necessary to explicitly introduce time.

Based on these notions, given $T$ and $\allargs$, define $i$'s \acl{AS} towards $T$ as $(\ileadsto, \ibeatse, \nibeatse)$, the three following relations.
\begin{description}
	\item[$\ileadsto$] is a relation on $\allargs \times T$. An argument $\ar$ \emph{supports} a proposition $t$, denoted by $\ar \ileadsto t$, iff $i$ considers that $\ar$ is an argument in favor of $t$. An argument $\ar$ may support several propositions in $i$'s view, or none.
	\item[$\ibeatse$] is a binary relation over $\allargs$ representing what $i$ considers as an attack in \emph{some} perspective. Let $\ar_1, \ar_2 \in \allargs$ be two arguments. We note $\ar_2 \ibeatse \ar_1$ ($\ar_2$ \emph{attacks} $\ar_1$) iff there is at least one perspective from which $i$ considers that $\ar_2$ turns $\ar_1$ into an invalid argument. Let us emphasize that here we are concerned with how $i$ sees the arguments $\ar_2$ and $\ar_1$, not about whether $\ar_2$ should be considered a good argument to attack $\ar_1$ by any independent standard. 
	\item[$\nibeatse$] is a binary relation over $\allargs$ defined in a similar way: $\ar_2 \nibeatse \ar_1$ iff there is at least one perspective from which $i$ does not consider that $\ar_2$ turns $\ar_1$ into an invalid argument.
\end{description}

We will admit that $¬(\ar_2 \ibeatse \ar_1) ⇒ \ar_2 \nibeatse \ar_1$. We emphasize that this assumption means that if, in a given perspective, $i$ does not know whether $\ar_2$ attacks $\ar_1$, or considers that $\ar_2$ is irrelevant to attack $\ar_1$, then we will say that $\ar_2$ does not attack $\ar_1$, and therefore there is at least one perspective in which $\ar_2$ does not attack $\ar_1$, that is: $\ar_2 \nibeatse \ar_1$. For simplicity's stake, we also assume that the pair of attack relations ($\ibeatse$ and $\nibeatse$) suffice to capture $i$’s changes of mind towards the topic, and therefore use a single $\ileadsto$ relation.

Let us now define $\ibeatsst$ as $\ar_2 \ibeatsst \ar_1 ⇔ ¬(\ar_2 \nibeatse \ar_1)$ ($\ar_2 \ibeatsst \ar_1$ if and only if $\ar_2$ attacks $\ar_1$ in all perspectives), and $\ar_2 \nibeatsst \ar_1 ⇔ ¬(\ar_2 \ibeatse \ar_1)$ ($\ar_2$ never attacks $\ar_1$). This enables us to define a decisive argument as one that is never attacked by any argument belonging to $\allargs$.
\begin{definition}[Decisive argument]
	\label{def:decisiveargument}
	Given a decision situation $(T, \allargs, {\ileadsto}, {\ibeatse}, {\nibeatse})$, we say that an argument $\ar \in \allargs$ is \emph{decisive} iff, $\forall \ar' \in \allargs$: $\ar' \nibeatsst \ar$.
\end{definition}
 
A decisive argument is hence an argument that is never attacked by any other argument. Notice that decisive arguments can be of very different sorts. Some decisive arguments will be very simple and straightforward arguments, which are so simple that they will be accepted by $i$ whatever the perspective. By contrast, some decisive arguments will be very elaborate ones, taking many aspects of the topic into account and anticipating all sorts of attacks, and accordingly never attacked by any other argument.

Given a decision situation, we are now in a position to characterize $i$'s stance towards the propositions in the topic $T$ once he has considered all the relevant arguments. We say that a proposition is acceptable if it is supported by a decisive argument. 

\begin{definition}[Acceptable proposition]
	\label{def:acceptreject}
	Given a decision situation $(T, \allargs,\allowbreak {\ileadsto},\allowbreak {\ibeatse}, {\nibeatse})$, a proposition $t$ is \emph{acceptable} iff $\exists \ar \in \allargs \suchthat \ar \ileadsto t \text{ and } \forall \ar': \ar' \nibeatsst \ar$.
\end{definition}

This definition echoes \citeauthor{rawls_political_2005}’ \citeyearpar{rawls_political_2005} emphasis on the requirement of \emph{acceptability} by reasonable citizens. Indeed, our choice of vocabulary is not an innocent one. In our framework, acceptable propositions are the ones that are supported by decisive arguments. These are the propositions which, once all the possibly relevant counter-arguments have been taken into account, that is, \emph{once all the perspectives have been envisaged}, appear to be supported by arguments that are never defeated.

Notice that we used a modal term to name this notions: we talk about “accept\emph{able}” rather than about “accept\emph{ed}” propositions. This is because, at a given point of time, individual $i$ might well fail to accept, as a matter of brute empirical fact, a proposition supported by a decisive argument, for example, because she does not know of this argument. All this is despite the fact that the decisive arguments referred to in the definitions of acceptable and excludable propositions are decisive \emph{according to $i$'s argumentative stance} – that is, by $i$'s own standards.

Notice also that, according to our definition, it is possible for a proposition $t$ to be acceptable and for not-$t$, or more generally for any proposition $t'$ in logical contradiction with $t$ or having empirical incompatibilities with $t$, to be acceptable too. To our best knowledge, the philosophical literature elaborating on Rawls' approach to reasonableness and acceptability did not investigate this point, despite the fact that, as our formal framework shows, it is a natural implication of the notion of acceptability. This means that the notion of acceptability does not exclude the possibility that, on due reflexion, $i$ might admit that both $t$ and its opposite are acceptable, as would argue thinkers inspired by the Humanist \emph{ars rhetorica} \citep{skinner_reason_1996} claiming that it is always possible to construct convincing arguments \emph{in utramque partem}.


Based on the definitions above, we can now define $i$'s \ac{DJ} as those propositions $t \in T$ that are acceptable. \begin{definition}[\ac{DJ} of $i$]
\label{def:acceptable}
	The \acl{DJ} corresponding to a decision situation $(T, \allargs, {\ileadsto}, {\ibeatse}, {\nibeatse})$ is:
	\begin{equation}
		\label{eq:DJ-u}
		T_i = \set{t \in T \suchthat t \text{ is acceptable}}.
	\end{equation}
\end{definition}

Thanks to the series of definitions above, the stance that respondants take thanks to deliberation is no longer vaguely defined in terms of a situation where they have been exposed \emph{to some indeterminate extent} to \emph{an indeterminate diversity} of argument that is considered \emph{according to some indeterminate standard} to be enough to consider that their stance has been formed and informed by deliberation. Now we rather have a formal definition, spelling out formally the basic idea that $i$'s stance can be called \emph{deliberated} once $i$ has taken into account all the relevant arguments and counterarguments.


Having such a formal definitin is a \emph{sine qua non} condition to be able to spell out clearly an answer to the question raised by the first blindspot of the current literature on DVM: when can we admit that we have reach deliberated stances?

Our formal definition allows here to identify a major problem: \acp{DJ} are not observable. Indeed, they are defined in terms of $\nibeatsst$. But in order to observe $\nibeatsst$ for $i$, we would need to lead $i$ to take successively all the possible perspectives she can have. The only observable data for the analyst are the ones obtained by querying $i$ to ask her if a given argument $\ar_2$ attacks another argument $\ar_1$. If she replies that it does, this is enough to conclude that, according to her, $\ar_2 \ibeatse \ar_1$. Indeed, in such a case we know that there is at least one perspective from which she thinks that $\ar_2$ attacks $\ar_1$: namely, the perspective that she currently has. If one wants to identify $i$'s  \acp{DJ}, one has to find a way to capture something defined in terms of $\nibeatsst$, exclusively on the basis of the answers that $i$ can give to questions such as: ``does $\ar_2$ attacks $\ar_1$''?

This above idea is, in our view, a major but so far insufficiently acknowledged problem facing approaches based on deliberation: they have to account for how they identify a properly \emph{deliberated} stance, but fail to do so, for want of a clear understanding of the very need to do so. By contrast, in the present contribution, we tackle this problem head-on. For that purpose, we elaborate a strategy that consists in defining a procedure through which a analyst (an economist, or a consultant) should organize his interaction with deliberating individuals. By explicitly thinking through a procedure performed by an analyst, this approach will not only tackle the first blindspot of deliberative approaches, but also the second one.

So, how is the analyst to proceed? As in any empirical scientific approach, the analyst concerned to identify $i$'s \ac{DJ} will start its interaction with $i$ with a \emph{model} $\eta$, which we define as a pair $(\mleadsto, \mbeats)$. Let us then define $T_\eta$ as the set of propositions that the model $\eta$ claims are supported:
\begin{equation}
	T_\eta = {\mleadsto}(\allargs) = \set{t \in T \suchthat \exists \ar \in \allargs \suchthat \ar \mleadsto t}.
\end{equation}

The analyst is concerned to identify $i$’s \ac{DJ}. Therefore, a valid model for him will be one that correctly captures $T_i$ (and not necessarily one that mimicks $\ileadsto, \ibeatse$).

\begin{definition}[Validity]
\label{valid}
	A model $\eta$ is valid iff $T_\eta=T_i$.
\end{definition}

Let us now define the following “operational” validity criterion for a model $\eta$ intended to capture $i$'s \ac{DJ}. We term it “operational” to emphasize that, as opposed to the definition of validity (\cref{valid}), it can be fed by the sole basis of observable data.

\begin{definition}[Operational validity criterion]
	\label{def:validity}
	A model $\eta$ of a decision situation is operationally valid iff, whenever $(\ar \mleadsto t)$, it holds that $[\ar \ileadsto t]$ and $[\forall \ar_c \in \allargs: (\ar_c \nibeatse \ar) ∨ (\exists \ar_{cc} \mbeats \ar_c ∧ \ar_{cc} \ibeatse \ar_c)]$, and whenever $t$ is not supported by $\eta$, $\forall \ar \ileadsto t: \exists \ar_c \mbeats \ar ∧ \ar_c \ibeatse \ar$.
\end{definition}

This criterion amounts to partially comparing, on the one hand, $i$'s stances towards propositions and arguments, and on the other, $\eta$'s representations of $i$’s stances. More precisely, a model satisfies the operational validity criterion (for short: is operationally valid) iff:
\begin{enumerate}[label=({\roman*}), ref={\roman*}]
	\item arguments that, according to the model, support a proposition $t$ are indeed considered by $i$ as supporting $t$;
	\item whenever a model uses an argument $\ar$ to support a claim, and that argument is attacked by a counter-argument $\ar_c$, the model can answer with a counter-counter-argument, using a counter-counter-argument that $i$ confirms indeed attacks the counter-argument $\ar_c$;
	\item whenever an argument $\ar$ supports a proposition that the model does not consider as supported, the model is able to attack that argument using a counter-argument that $i$ indeed thinks is an attack against $\ar$.
\end{enumerate}

At this stage, it is important to emphasize that, if an analyst wants to check whether the above operational validity criterion is satisfied, he does so by querrying $i$ about whether a given argument attacks another argument, which can, in some case, means that he shows to $i$ an argument that \emph{she had not thought about}. The analyst therefore actively interacts with $i$, he is not a pure, neutral observer who keeps his distances with his objects study. He interacts with his object study and in a sense modifies it as he checks the operational validity criterion. We therefore witness here a first level at which our approach tackles the second blindspot of the current deliberative literature.

That said, the operational validity criterion above does not bear on its face its relation with validity. Indeed, one cannot admit, in all generality, that a model is operationally model if and only of it is valid. This equivalence however holds if three conditions are satisfied, as established by our theorem below.

For simplicity's sake in the exposition, let us mandate that each proposition in the topic may be supported by at most one argument according to $\mleadsto$, and $\eta$ has at most one counter-argument for each argument. These two conditions, which can be relaxed without much difficulty, are used here to simplify the presentation. Our theorem is based on four conditions.

Our first condition assumes that $i$ possibly changes her mind about whether an argument $\ar'$ attacks another one only when there exists another argument attacking $\ar'$. 

\begin{condition}[Justifiable unstability]
	\label{def:justifiableStrong}
	A decision situation $(T, \allargs, {\ileadsto}, {\ibeatse}, {\nibeatse})$ is \emph{justifiably unstable} iff, for all pairs of arguments $(\ar, \ar')$:
	\begin{equation}
		\ar' \ibeatse \ar \text{ and } \ar' \nibeatse \ar ⇒ \exists \ar_c \suchthat \ar_c \ibeatse \ar'.
	\end{equation}
\end{condition}

Our second condition has to do with the way $i$ reasons. Imagine that $i$ finds himself in the following situation. He declares that $\ar_1$ is attacked by $\ar_2$. However, $i$ is also ready to declare that $\ar_2$ is in turn attacked by $\ar_3$, a decisive argument. Clearly, in such a situation, $i$'s credence in the strength of $\ar_2$ appears fragile. It therefore seems natural enough to assume that, if we carve out an argument $\ar$, playing the same argumentative role as $\ar_1$, but anticipating and defeating the attack by $\ar_2$, $i$ will endorse $\ar$. This assumption is formalized by the condition Closed under reinstatement, below. To write it down, we first need to formalize, thanks to the following notion of \emph{replacement}, the idea that a set of arguments plays an argumentative role similar to the one of another argument. We say a set of arguments $\args \subseteq \allargs$ replaces an argument $\ar \in \allargs$ whenever all arguments attacked by $\ar$ are also attacked by some argument $\ar' \in \args$, and all propositions supported by $\ar$ are also supported by some argument $\ar' \in \args$. %(At this stage, all we need is a means to formalize the idea of a single argument replacing another argument, but we will need the notion of a set of replacers later.)
\begin{definition}[Replacing arguments]
	A set of arguments $\args \subseteq \allargs$ \emph{replaces} $\ar \in \allargs$ iff ${\ibeatse}(\ar) \subseteq {\ibeatse}(\args)$ and ${\ileadsto}(\ar) \subseteq {\ileadsto}(\args)$. 
	We say that $\ar'$ replaces $\ar$, with $\ar, \ar' \in \allargs$, to mean that $\{\ar'\}$ replaces $\ar$.
\end{definition}
	
\begin{condition}[Closed under reinstatement]
	\label{def:closed}
	A decision situation $(T, \allargs, {\ileadsto}, {\ibeatse}, {\nibeatse})$ is \emph{closed under reinstatement} iff, $\forall \ar_1 ≠ \ar_2 ≠ \ar_3 ≠ \ar_1 \in \allargs$ such that $\ar_3 \ibeatsst \ar_2 \ibeatse \ar_1$, with $\ar_3$ decisive:
	\begin{equation}
		\exists \ar \suchthat \ar \text{ replaces } \ar_1 \text{ and } {\ibeatseinv}(\ar) \subseteq \ibeatseinv(\ar_1) \setminus \{\ar_2\}.
	\end{equation}
\end{condition}

Finally, we introduce two conditions on the size of chains of arguments and counter-arguments.
We talk about a chain of length $k$ in $\ibeatse$ to refer to a finite sequence $\ar_i$ in $\allargs$, $1 ≤ i ≤ k$, such that $\ar_i \ibeatse \ar_{i + 1}$ for $1 ≤ i ≤ k - 1$. An infinite chain is an infinite sequence $\ar_i$ such that $\ar_i \ibeatse \ar_{i + 1}$ for all $i \in \N$. (We do not mandate that a chain should be transitively closed.) 

\begin{condition}[Bounded breadth]
\label{def:B.br}	
A decision situation $(T, \allargs, {\ileadsto}, {\ibeatse}, {\nibeatse})$ has a \emph{bounded breadth} iff there is no argument that is attacked by an infinite number of counter-arguments.
\end{condition}

\begin{condition}[Bounded length]
\label{def:B.lg}
	A decision situation $(T, \allargs, {\ileadsto}, {\ibeatse}, {\nibeatse})$ has a \emph{bounded length} iff there is no infinite chain of counter-arguments. (Cycles in $\ibeatse$ are therefore excluded as well.)
\end{condition}

The four conditions above establish a link between operational validity and validity thanks to the following theorem:

\begin{theorem}
	\label{thm:clearcutWeak}
	Assume a decision situation $(T, \allargs, {\ileadsto}, {\ibeatse}, {\nibeatse})$ is closed under reinstatement, justifiably unstable and has bounded length and breadth. Then: i) there exists an operationally valid model of that decision situation; ii) any operationally valid model $\eta$ satisfies $T_i = T_\eta$.
\end{theorem}

It should be clear by now that the strength of this theorem is that it allows to say something about $i$'s \ac{DJ} despite the fact that directly identifying $i$'s \ac{DJ} is hopeless. But this strength has a price: it holds only if the above conditions are met. Therefore, in order to understand the true meaning of this theorem, and thereby to understand how likely it is that one will ever be able to correctly identify $i$'s \ac{DJ}, it is pivotal to ponder on the meaning of these conditions.

A first thing to notice in this respect is that the conditions above are quite heroic. One cannot realistically expect that real-life decision situations will fulfill these conditions. Fortunately, it is possible to weaken these conditions subtantially, by distinguishing $\allargs$ from the restricted set of arguments with which the analyst works in practice. This trick allow to define conditions whose meaning is very similar to the one of the simple conditions spelled out here, but which are more realistic because they refer to a restricted set of arguments. This task however involves quite a lot of technical subtelties. We refer to Cailloux and Meinard for a complete exposition. For our purposes here, it will not be necessary to delve into these technicalities. Indeed, because the weaker and simpler conditions have very similar meanings, we can content ourselves with a discussion of the simple conditions. This will provide all we need for the purpose of our methodological investigation into the theoretical foundations of DVM and deliberative approaches.

In general terms, our conditions can be interpreted in three different ways:
\begin{enumerate}[label=({\roman*})]%, ref={\roman*}
	\item \label{inter:axioms} as axioms capturing minimal properties concerning arguments and the way $i$ reasons,
	\item \label{inter:empir} as empirical hypotheses,
	\item \label{inter:rules} as rules governing the decision aiding process (rules that $i$ can commit to abide by, or can consider to be well-founded safeguards for the proper unfolding of the process).
\end{enumerate}

Some of the conditions of our theorems are arguably more congenial to a given interpretation. For example, it seems natural enough to interpret \cref{def:closed} as a rationality requirement of the kind that it makes sense to use as an axiom (interpretation \ref{inter:axioms}). By contrast, \cref{def:justifiableStrong} is the kind of condition that can easily be translated in the form of rules than decision-makers can be asked to abide by when they engage in a decision-aiding process (interpretation \ref{inter:rules}). \Cref{def:B.br,def:B.lg} can easily be seen as empirical hypotheses (interpretation \ref{inter:empir}).

We do not want, here, to take a rigid stance on which interpretation or mix of interpretations should be preferred. Our point is rather to stress that, whatever the chosen interpretation, it has implicatons for analysts implementing deliberation and DMV in practice. If the conditions are understood as axioms\ref{inter:axioms}, analyst have to be able to entrench their justifiability. This might be possible, at least in some cases, for \cref{def:closed}, but this is much more questionable for the other conditions. If the conditions are understood as empirical hypotheses, then analysts have to be able to empirically test them, which raises unsettled difficulties falling well beyond the scope of the present article. Lastly, if they are understood as \ref{inter:rules}, then, even more than through the process of operational validity checking, the analyst is engaged in an active interaction with and modification of his object study, that is should be able to account for and justify.

\section{Meaning and perspectives}
\label{disc}
Before coming back to deliberation and DVM, it is useful to take a broader view to notice that the two notions of rationality and anti-paternalism, whose formalization provides the basis of our framework, play an important role in contemporary normative philosophy. Indeed, the search for a compromise, an equilibrium or a satisfactory articulation between these two requirements can be seen as a key thread running through contemporary political philosophy.

A case in point is Rawls’ notion of a “reflective equilibrium”. Following \citet{goodman_fact_1983}, \citet[][p. 18]{rawls_theory_1999} used “reflective equilibrium” to refer to a “process of mutual adjustment of principles and considered judgments”. This formulation highlights that, in Rawls' \emph{Theory of Justice}, a prominent role attributed to this concept was to do justice both to people's moral intuitions and to the need to systematize visions of justice. Rawls thereby granted a prominent importance to people's own judgments (both as to how specific cases should be adjudicated and as to whether a given general principle is acceptable), which is what we termed  “anti-paternalism”. As for the reference to principles, and the idea that the judgments to be taken into account are the ones that can be termed “considered”, they echo our rationality requirement, if one admits that judgments are  “considered” when they are buttressed on a careful analysis of arguments and counterarguments, and that principles systematizing considered judgments provide strong arguments in favor of these judgments. Rawls' “reflective equilibrium” hence embodies the two ideas that lie at the core of our concept of deliberated judgment. The credibility of this interpretation is reinforced by the fact that \citeauthor{rawls_political_2005}’ \citeyearpar{rawls_political_2005} later work grants an increasing importance to the notion of “due reflexion” --- a notion that does not refer to principles and it more general than the one of “reflective equilibrium”.

Although we obviously agree with \citet{bartkowski_beyond_2018} that Sen's philosophy provides important insights into the theoretical foundations of DVM, by displaying this anchorage of our formal framework in the search for an articulation between the requirements of rationality and anti-paternalism (and its examplification by Rawls' notion of a “reflective equilibrium”), we hence want to stress that the larger literature in political philosophy also contains useful sources that are liable to help structuring the foundations of DVM at an even deeper level.

This broader philosophical perspective also usefully points another important debate. A prominent aspect of the concept of reflective equilibrium that our reasoning has so far set aside is its purported interpersonal dimension. In \emph{A Theory of Justice}, the reflective equilibrium is not presented as the result of the endeavor of an insulated individual, but is rather defined from the very begining in collective terms. Similarly, when Rawls makes use of the concept in \emph{Political Liberalism}, he presents it as a  “device of representation” that citizens can use to calibrate their public discussions. Like many other key concepts of the rawlsian framework, the one of reflective equilibrium is hence systematically presented by Rawls in pluri-individual settings --- another notorious example being the “partie\emph{s}” choosing the principles of justice, which are unequivocally presented as a collective. At first sight one might object to our approach that it lacks such an interpersonal dimension.

If true, this would be a worrying weakness, especially given that some authors such as \citet{vatn_institutional_2009} argue that one of the distinctive strengths of deliberative approaches is that they lead people to reason according to ``We-rationality'', as opposed to ``I-rationality''. It is therefore important to stress how the interpersonal dimension comes into play in our approach, and to contrast it with with how it comes into play in some prominent philosophical approaches.

Rawls' claim to integrate an interpersonal dimension has been fiercely criticized in the literature. \citet{habermas_short_1999} famously argued that Rawls' approach involved a preemption, by the philosopher, of issues that have their proper place in public debates among citizens. Seen through these lenses, the proximity between our framework and Rawls' theory might seem to reinforce the impression that our frameworks lacks this dimension.

\citeauthor{habermas_moralbewustsein_1983}'s \citeyearpar{habermas_moralbewustsein_1983} attempt at overcoming this problem is of particular significance from our point of view. In his theory, the content of the theory of justice should not be seen as the result of an explicit deliberation or reflexion, but are rather the result of a transcendental deduction (though of a weaker sort that the Kantian one) --- that is, they are demonstrated to be conditions of possibility of all sorts of interactions mediated by communication in a society. As opposed to these so-called “moral” tenets, a given “ethical” notion can be consensually accepted in a given society or group, but can become a bone of contention when various groups meet or merge. Thus, in this approach, there can exist a dissensus between various individuals in the society on many issues. But when it comes to the subject matter of moral theory, any dissensus is bound to be ephemeral, because the very process of communication through which people try to settle their disagreement presupposes an implicit acceptance of the tenets that moral theory aims to capture.

Our frameworks shares with this reasoning of Habermas' the idea that it is important to distinguish different kinds of issues about which people can agree or disagree: disagreements are not necessarily fatal to deliberation, but disagreements on \emph{some fondamental issues}, pertaining to the role of deliberation, are fatal. Habermas' purported solution through a transcendantal deduction has itself been criticized \citep{heath_communicative_2001}. In particular, it is not self-evident that there is such a thing as a determinate set of conditions of possibility common to all sorts of interactions mediated by communication in a society. Habermas' theory hence appears to be plagued by the same problem that he denonced in Rawls' theory: both wanted to prove that consensus will occur on certain issues, but they both collapsed in ``theoretical preemption''. Rawls only provided conceptual reasons to believe that the “considered judgments” of various individual in a society should converge or share an “overlapping consensus”, and these concepual reasons fail to win over everyone, even among philosophers largely inspired by Rawls \citep{estlund_insularity_1998, estlund_democratic_2009}. Similarly Habermas' theory is weakened by the fact that its fixed point (the content of moral theory, the so-called “U” and “D” tenets \citep{habermas_moralbewustsein_1983}), is a purely conceptual finding that happens to arouse conceptual criticisms.

Our approach ventures an alternative solution to this problem. We have to face the fact that, if various people can diverge in their considered judgments, collective deliberation and decision-making unavoidably faces issues of aggregation of diverging attitudes. One cannot simply take this idea as a mandate to accept the divergence of preference and additively aggregate them, as \citet{bartkowski_beyond_2018} suggest, because if this there is no agreement on the relevance of such an aggregation, this aggregation is pointless. There is therefore a crucial need to empirically inquire whether such problems occur, or can occur, and if so in which conditions. Indeed, in such conditions, believing in deliberative approaches is wishful thinking. Like any method, DVM and other deliberative methods have a potentially restricted domain of application, and any application of these methods should therefore start by checking if it indeed falls in its domain of application. For such a verification, an empirical approach is needed. Both Rawls, Habermas, Sen and the DVM literature as a whole lack such an approach. Though the concrete details of empirical studies implementing it fall beyond the scope of the present article, our framework spelled out above provides the backbone of such an approach.

That said, to conclude we have to emphasize that, despite the promises offered by our framework, this is no magical tool. In particular, we do not claim that it overcomes Hume's law by providing the means to deduce ought from is. As repeatedly emphazised, our approach is anchored in two unmistakably normative tenets: rationality and anti-paternalism. Our reasoning in \cref{disc} was devoted to show that these two tenets play a very fundamental role in contemporary normative philosophy, and can therefore in that very specific sense be considered minimal. But one could accordingly argue that the criticism that we address at Habermas also applies to our framework, one step deeper. This is certainly true. We accordingly do not claim that it solves all the philosophical difficulties of the theorie of deliberative democracy, DMV and their fundations. Our more modest claim is to have introduced an approach that allows to go deeper than the current literature in the direction of providing strong philosophical foundations to empirical applications of philosophies of deliberation, such as DVM. 


\begin{acknowledgements}
We thank xxxx for very helpful comments.
\end{acknowledgements}

\bibliography{beliefs,philo-eco,deliber}


\end{document}
