\RequirePackage{amsmath}
\RequirePackage[l2tabu, orthodox]{nag}
%\documentclass[smallextended,nospthms,natbib]{svjour3}
\documentclass[version=3.21, pagesize, twoside=off, bibliography=totoc, DIV=calc, fontsize=12pt, a4paper, french, english]{scrartcl}
\newcommand{\institute}[1]{}
\newcommand{\keywords}[1]{}
\newenvironment{acknowledgements}{
	\section*{Acknowledgements}
}{
}
%\makeatletter \let\cl@chapter\relax \makeatother
%\smartqed  % flush right qed marks, e.g. at end of proof 
%INSTALL
\pdfgentounicode=1 %permits (with package glyphtounicode) to copy eg x ⪰ y iff v(x) ≥ v(y) from pdf to unicode data. 
\input{glyphtounicode}%TODO avoid warning when redefining → (and others) ; make it work for ℝ (U+211D) as well
\usepackage[T1]{fontenc}%encode resulting accented characters correctly in resulting PDF, permits copy-paste
\usepackage[utf8]{inputenc}
\usepackage{newunicodechar}%able to use e.g. → or ≤ in source
\usepackage{lmodern}%has more characters such as ligatures, permit copy from resulting PDF.
\usepackage{textcomp}%useful for redefining → and ¬ and … (otherwise seems to attempt to use \textrightarrow and the like but are not defined)
% solves bug in lmodern, https://tex.stackexchange.com/a/261188/
\DeclareFontShape{OMX}{cmex}{m}{n}{
  <-7.5> cmex7
  <7.5-8.5> cmex8
  <8.5-9.5> cmex9
  <9.5-> cmex10
}{}

\SetSymbolFont{largesymbols}{normal}{OMX}{cmex}{m}{n}
\SetSymbolFont{largesymbols}{bold}  {OMX}{cmex}{m}{n}
%warn about missing characters
\tracinglostchars=2

%REDAC
\usepackage{booktabs}
\usepackage{calc}
\usepackage{tabularx}

\usepackage{etoolbox} %for addtocmd, newtoggle commands
\newtoggle{LCpres}
\togglefalse{LCpres}

\usepackage{mathtools} %load this before babel!
	\mathtoolsset{showonlyrefs,showmanualtags}

\usepackage{natbib}%Package frenchb asks to load natbib before babel/frenchb

%\usepackage[super]{nth}%better use fmtcount! (loaded by datetime anyway; see below about pbl with warnings and package silence)
\usepackage{listings} %typeset source code listings
	\lstset{language=XML,tabsize=2,captionpos=b,basicstyle=\NoAutoSpacing}%NoAutoSpacing avoids space before colon or ?}%,literate={"}{{\tt"}}1, keywordstyle=\fontspec{Latin Modern Mono Light}\textbf, emph={String, PreparedStatement}, emphstyle=\fontspec{Latin Modern Mono Light}\textbf, language=Java, basicstyle=\small\NoAutoSpacing\ttfamily, frame=single, aboveskip=0pt, belowskip=0pt, showstringspaces=false
\usepackage[nolist,smaller,printonlyused]{acronym}%,smaller option produces warnings from relsize in some cases, it seems.% Note silence and acronym and hyperref make (xe)latex crash when ac used in section (http://tex.stackexchange.com/questions/103483/strange-packages-interaction-acronyms-silence-hyperref), rather use \section{\texorpdfstring{\acs{UE}}{UE}}.
\usepackage{fmtcount}
\usepackage[nodayofweek]{datetime}%must be loaded after the babel package. However, loading it after {nth} generates a warning from fmtcount about ordinal being already defined. Better load it before nth? (then we can remove the silence package which creates possible crashes, see above.) Or remove nth?
%\usepackage{xspace}%do we need this?
\nottoggle{LCpres}{
	\usepackage[textsize=small]{todonotes}
}{
}
\iftoggle{LCpres}{
	%remove pdfusetitle (implied by beamer)
	\usepackage{hyperref}
}{
% option pdfusetitle must be introduced here, not in hypersetup.
	\usepackage[pdfusetitle]{hyperref}
}
\nottoggle{LCpres}{
%seems like authblk wants to be later than hyperref, but sooner than silence
\usepackage{authblk}
\renewcommand\Affilfont{\small}
}{
}
\usepackage{silence}
\WarningFilter{newunicodechar}{Redefining Unicode character}
%breaklinks makes links on multiple lines into different PDF links to the same target.
%colorlinks (false): Colors the text of links and anchors. The colors chosen depend on the the type of link. In spite of colored boxes, the colored text remains when printing.
%linkcolor=black: this leaves other links in colors, e.g. refs in green, don't print well.
%pdfborder (0 0 1, set to 0 0 0 if colorlinks): width of PDF link border
%hidelinks or: colorlinks, linkcolor=black, citecolor=black, urlcolor={blue!80!black}
\hypersetup{breaklinks, bookmarksopen}
%add hyperfigures=true in hypersetup (already defined in article mode)
\iftoggle{LCpres}{
	\hypersetup{hyperfigures}
}{
}

%in Beamer, sets url colored links but does not change the rest of the colors (http://tex.stackexchange.com/questions/13423/how-to-change-the-color-of-href-links-for-real)
%\hypersetup{breaklinks,bookmarksopen,colorlinks=true,urlcolor=blue,linkcolor=,hyperfigures=true}
% hyperref doc says: Package bookmark replaces hyperref’s bookmark organization by a new algorithm (...) Therefore I recommend using this package.
\usepackage{bookmark}

% center floats by default, but do not use with float
% \usepackage{floatrow}
% \makeatletter
% \g@addto@macro\@floatboxreset\centering
% \makeatother
\nottoggle{LCpres}{
	\usepackage{enumitem} %follow enumerate by a string saying how to display enumeration
}{
}
\usepackage{ragged2e} %new com­mands \Cen­ter­ing, \RaggedLeft, and \RaggedRight and new en­vi­ron­ments Cen­ter, FlushLeft, and FlushRight, which set ragged text and are eas­ily con­fig­urable to al­low hy­phen­ation (the cor­re­spond­ing com­mands in LaTeX, all of whose names are lower-case, pre­vent hy­phen­ation al­to­gether). 
\usepackage{siunitx} %[expproduct=tighttimes, decimalsymbol=comma] ou (plus récent ?) [round-mode=figures, round-precision=2, scientific-notation = engineering]
\sisetup{detect-all, locale = FR, strict}% to detect e.g. when in math mode (use a math font) - check whether this makes sense with strict
\usepackage{braket} %for \Set
\usepackage{doi}

\usepackage{amsmath,amsthm}
\usepackage{amssymb}%for \mathbb{R} %includes amsfonts
\usepackage{bm}%“The \boldsymbol command is obtained preferably by using the bm package, which provides a newer, more powerful version than the one provided by the amsmath package. Generally speaking, it is ill-advised to apply \boldsymbol to more than one symbol at a time.” — AMS Short math guide. “If no bold font appears to be available for a particular symbol, \bm will use ‘poor man’s bold’” — bm
% \usepackage{dsfont} %for what?

\usepackage{environ}%for xdescwd command
%BUT see https://tex.stackexchange.com/questions/83798/cleveref-varioref-missing-endcsname-inserted for cleveref with french
\usepackage{cleveref}% cleveref should go "laster" than hyperref
%GRAPHICS
\usepackage{pgf}
\usepackage{pgfplots}
	\usetikzlibrary{babel, matrix, fit, plotmarks, calc, trees, shapes.geometric, positioning, plothandlers, arrows, shapes.multipart}
\pgfplotsset{compat=1.14}
\usepackage{graphicx}

\DeclareMathAlphabet\mathbfcal{OMS}{cmsy}{b}{n}

\graphicspath{{graphics/},{graphics-dm/}}
\DeclareGraphicsExtensions{.pdf}
\newcommand*{\IncludeGraphicsAux}[2]{%
	\XeTeXLinkBox{%
		\includegraphics#1{#2}%
	}%
}%

%HACKING
\usepackage{printlen}
\uselengthunit{mm}
% 	\newlength{\templ}% or LenTemp?
% 	\setlength{\templ}{6 pt}
% 	\printlength{\templ}
\usepackage{scrhack}% load at end. Corrects a bug in float package, which is outdated but might be used by other packages
\usepackage{mathrsfs}% for \mathscr
%see https://tex.stackexchange.com/questions/409212/size-substitution-with-fontsize-14
\DeclareFontFamily{U}{rsfs}{\skewchar\font127 }
\DeclareFontShape{U}{rsfs}{m}{n}{%
   <-6.5> rsfs5
   <6.5-8> rsfs7
   <8-> rsfs10
}{}

%Beamer-specific
%do not remove babel, which beamer uses (beamer uses the \translate command for the appendix); but french can be removed.
\iftoggle{LCpres}{
	\usepackage{appendixnumberbeamer}
	\setbeamertemplate{navigation symbols}{} 
	\usepackage{preamble/beamerthemeParisFrance}
	\usefonttheme{professionalfonts}
	\setcounter{tocdepth}{10}
	%From: http://tex.stackexchange.com/questions/168057/beamer-with-xelatex-on-texlive2013-enumerate-numbers-in-black
%I don’t think it’s useful to submit this as a bug: nothing has been solved since March, 2015. See: https://bitbucket.org/rivanvx/beamer/issues?status=resolved.

\setbeamertemplate{enumerate item}
{
  \begin{pgfpicture}{-1ex}{-0.65ex}{1ex}{1ex}
    \usebeamercolor[fg]{item projected}
    {\pgftransformscale{1.75}\pgftext{\normalsize\pgfuseshading{bigsphere}}}
    {\pgftransformshift{\pgfpoint{0pt}{0.5pt}}
      \pgftext{\usebeamercolor[fg]{item projected}\usebeamerfont*{item projected}\insertenumlabel}}
  \end{pgfpicture}%
}

\setbeamertemplate{enumerate subitem}
{
  \begin{pgfpicture}{-1ex}{-0.55ex}{1ex}{1ex}
    \usebeamercolor[fg]{subitem projected}
    {\pgftransformscale{1.4}\pgftext{\normalsize\pgfuseshading{bigsphere}}}
    \pgftext{%
      \usebeamercolor[fg]{subitem projected}%
      \usebeamerfont*{subitem projected}%
      \insertsubenumlabel}
  \end{pgfpicture}%
}

\setbeamertemplate{enumerate subsubitem}
{
  \begin{pgfpicture}{-1ex}{-0.55ex}{1ex}{1ex}
    \usebeamercolor[fg]{subsubitem projected}
    {\pgftransformscale{1.4}\pgftext{\normalsize\pgfuseshading{bigsphere}}}
    \pgftext{%
      \usebeamercolor[fg]{subsubitem projected}%
      \usebeamerfont*{subitem projected}%
      \insertsubsubenumlabel}
  \end{pgfpicture}%
}


}{
}
% \newcommand{\citep}{\cite}%Better: leave natbib.
% \setbeamersize{text margin left=0.1cm, text margin right=0.1cm} 
% \usetheme{BrusselsBelgium}%no, replace with paris
%\usetheme{ParisFrance}, no, usepackage better!
% Tex Gyre takes too much space, replace with Latin Modern Roman / Sans / Mono.
% Difference when loading explicitly Latin Modern Sans (compared to not using \setsansfont at all):
% the font LMSans17-Regular appears in the document;
% the title of the slides appears differently;
% it does not say (in the log file):
% > LaTeX Font Info:    Font shape `EU1/lmss/m/it' in size <10.95> not available
% > (Font)              Font shape `EU1/lmss/m/sl' tried instead on input line 85.
% > LaTeX Font Info:    Try loading font information for EU1+lmtt on input line 85.

%tikzposter-specific
%remove \usepackage{ragged2e}: causes 1=1 to be printed in the middle of the poster. (Anyway prints a warning about those characters being missing.)
%put [french, english] options next to \usepackage{babel} to avoid warning of 

\newcommand{\commentOC}[1]{{\small\color{blue}$\big[$OC: #1$\big]$}}
\newcommand{\commentOCf}[1]{{\small\color{blue}{\selectlanguage{french}$\big[$OC : #1$\big]$}}}
\newcommand{\commentYM}[1]{{\small\color{red}{\selectlanguage{french}$\big[$YM : #1$\big]$}}}
\newcommand{\innote}[1]{{\scriptsize{#1}}}
%Or: \todo[color=green!40]

%this probably requires outdated float package, see doc KomaScript for an alternative.
% \newfloat{program}{t}{lop}
% \floatname{program}{PM}

%definition, theorem, lemma, example environments, qed trickery are only needed in article mode (not Beamer)
\nottoggle{LCpres}{
%style is plain by default (italic text)
	\newtheorem{definition}{Definition}
	\newtheorem{theorem}{Theorem}
%no italic: expected.
%http://tex.stackexchange.com/questions/144653/italicizing-of-amsthm-package
	\newtheorem{lemma}{Lemma}
	\newtheorem{condition}{Condition}
%\crefname{axiom}{axiom}{axioms}%might be needed for workaround bug in cref when defining new theorems?

%\ifdefined\theorem\else
%\newtheorem{theorem}{\iflanguage{english}{Theorem}{Théorème}}
%\fi

\theoremstyle{remark}
	\newtheorem{examplex}{Example}
	\newtheorem{remarkx}{Remark}

%trickery allowing use of \qedhere and the like.
\newenvironment{example}{
	\pushQED{\qed}\renewcommand{\qedsymbol}{$\triangle$}\examplex
}{
	\popQED\endexamplex
}
\newenvironment{remark}{
	\pushQED{\qed}\renewcommand{\qedsymbol}{$\triangle$}\remarkx
}{
	\popQED\endremarkx
}
}{
}
\crefname{examplex}{example}{examples}% I wonder why this is unnecessary in case of singular
\crefname{condition}{condition}{conditions}
\makeatletter
\cref@addlanguagedefs{french}{%
	\crefname{examplex}{exemple}{exemples}%
}
\makeatother

%trickery allowing use of \qedhere and the like.
%\newcommand{\xqed}[1]{%
%    \leavevmode\unskip\penalty9999 \hbox{}\nobreak\hfill
%    \quad\hbox{#1}}
%\AtEndEnvironment{example}{
%	\xqed{$\triangle$}%
%}
%\AtEndEnvironment{proof}{
%	\qed%
%}

%which line breaks are chosen: accept worse lines, therefore reducing risk of overfull lines. Default = 200
%\tolerance=2000
%accept overfull hbox up to...
%\hfuzz=2cm
%reduces verbosity about the bad line breaks
%\hbadness 5000
%sloppy sets tolerance to 9999
%\apptocmd{\sloppy}{\hbadness 10000\relax}{}{}

\bibliographystyle{abbrvnat}
%or \bibliographystyle{apalike} for presentations?

%doi package uses old-style dx.doi url, see 3.8 DOI system Proxy Server technical details, “Users may resolve DOI names that are structured to use the DOI system Proxy Server (http://doi.org (preferred) or http://dx.doi.org).”, https://www.doi.org/doi_handbook/3_Resolution.html
\makeatletter
\patchcmd{\@doi}{dx.doi.org}{doi.org}{}{}
\makeatother

% WRITING
%\newcommand{\ie}{i.e.\@\xspace}%to try
%\newcommand{\eg}{e.g.\@\xspace}
%\newcommand{\etal}{et al.\@\xspace}
\newcommand{\ie}{i.e.\ }
\newcommand{\eg}{e.g.\ }
\newcommand{\mkkOK}{\checkmark}%\color{green}{\checkmark}
\newcommand{\mkkREQ}{\ding{53}}%requires pifont?%\color{green}{\checkmark}
\newcommand{\mkkNO}{}%\text{\color{red}{\textsf{X}}}

\makeatletter
\newcommand{\boldor}[2]{%
	\ifnum\strcmp{\f@series}{bx}=\z@
		#1%
	\else
		#2%
	\fi
}
\newcommand{\textstyleElProm}[1]{\boldor{\MakeUppercase{#1}}{\textsc{#1}}}
\makeatother
\newcommand{\electre}{\textstyleElProm{Électre}\xspace}
\newcommand{\electreIv}{\textstyleElProm{Électre Iv}\xspace}
\newcommand{\electreIV}{\textstyleElProm{Électre IV}\xspace}
\newcommand{\electreIII}{\textstyleElProm{Électre III}\xspace}
\newcommand{\electreTRI}{\textstyleElProm{Électre Tri}\xspace}
% \newcommand{\utadis}{\texorpdfstring{\textstyleElProm{utadis}\xspace}{UTADIS}}
% \newcommand{\utadisI}{\texorpdfstring{\textstyleElProm{utadis i}\xspace}{UTADIS I}}

%TODO
% \newcommand{\textstyleElProm}[1]{{\rmfamily\textsc{#1}}} 

\newcommand{\txtlaxiom}{$\alllang$\hyp{}axiom}
\newcommand{\txtlaxioms}{$\alllang$\hyp{}axioms}
\newcommand{\laxiomatisation}{$\alllang$\hyp{}axiomatisation}
\newcommand{\laxiomatization}{$\alllang$\hyp{}axiomatization}
%bold. Requires special command for math bold cal!
\newcommand{\laxiomatisationbd}{$\mathbfcal{L}$\textbf{\hyp{}axiomatisation}}
%\newcommand{\laxiomatisationtitle}{\texorpdfstring{$\alllang$\hyp{}Axiomatisation}{L\hyp{}Axiomatisation}}%Does not work because of AAMAS font switching command (tries to typeset it using ptm font), so we should use something like {\usefont{EU1}{lmr}{m}{n}©} but manually select the right font size and weight (which I ignore).
\newcommand{\laxiomatisationtitle}{\texorpdfstring{{\large$\mathcal{L}$}\hyp{}Axiomatisation}{ℒ\hyp{}Axiomatisation}}
%\commentUE{Should we maybe write $\ell$-axiom rather than l-axiom, for better readability (to avoid ell being read as one)? Or we could even write $\mathcal{L}$-axiom, to have an explicit reference to the language in which the axiom is expressed?}

\newunicodechar{ℕ}{\mathbb{N}}
\newunicodechar{ℝ}{\mathbb{R}}
\newunicodechar{−}{\ifmmode{-}\else\textminus\fi}
\newunicodechar{≠}{\ensuremath{\neq}}
\newunicodechar{≤}{\ensuremath{\leq}}
\newunicodechar{≥}{\ensuremath{\geq}}
\newunicodechar{≻}{\succ}
\newunicodechar{⊁}{\nsucc}
\newunicodechar{▷}{\triangleright}
\newunicodechar{⋫}{\ntriangleright}
\newunicodechar{→}{\ifmmode\rightarrow\else\textrightarrow\fi}
\newunicodechar{⇒}{\ensuremath{\Rightarrow}}
\newunicodechar{⇏}{\ensuremath{\nRightarrow}}
\newunicodechar{⇔}{\ensuremath{\Leftrightarrow}}
\newunicodechar{∪}{\cup}
\newunicodechar{∩}{\cap}
\newunicodechar{∧}{\land}
\newunicodechar{∨}{\lor}
\newunicodechar{¬}{\ifmmode\lnot\else\textlnot\fi}
\newunicodechar{…}{\ifmmode\ldots\else\textellipsis\fi}
\newunicodechar{×}{\ifmmode\times\else\texttimes\fi}
\newunicodechar{γ}{\ensuremath{\gamma}}
\newunicodechar{□}{\Box}


\newcommand{\R}{ℝ}
\newcommand{\N}{ℕ}
\newcommand{\Z}{ℤ}
\newcommand{\card}[1]{\lvert{#1}\rvert}
\newcommand{\powerset}[1]{\mathscr{P}(#1)}%\mathscr rather than \mathcal: scr is rounder than cal (at least in XITS Math).
%powerset without zero
\newcommand{\powersetz}[1]{\mathscr{P}_\emptyset(#1)}
\newcommand{\suchthat}{\;\ifnum\currentgrouptype=16 \middle\fi|\;}
%\newcommand{\Rplus}{\reels^+\xspace}

\AtBeginDocument{%
	\renewcommand{\epsilon}{\varepsilon}
% we want straight form of \phi for mathematics, as recommended in UTR #25: Unicode support for mathematics.
%	\renewcommand{\phi}{\varphi}
}

% with amssymb, but I don’t want to use amssymb just for that.
% \newcommand{\restr}[2]{{#1}_{\restriction #2}}
%\newcommand{\restr}[2]{{#1\upharpoonright}_{#2}}
\newcommand{\restr}[2]{{#1|}_{#2}}%sometimes typed out incorrectly within \set.
%\newcommand{\restr}[2]{{#1}_{\vert #2}}%\vert errors when used within \Set and is typed out incorrectly within \set.
\DeclareMathOperator*{\argmax}{arg\,max}
\DeclareMathOperator*{\argmin}{arg\,min}


%Decision Theory (MCDA and SC)
\newcommand{\allalts}{\mathscr{A}}
\newcommand{\allcrits}{\mathscr{C}}
\newcommand{\alts}{A}
\newcommand{\dm}{i}
\newcommand{\allF}{\mathscr{F}}
\newcommand{\allvoters}{\mathscr{N}}
\newcommand{\voters}{N}
\newcommand{\allprofs}{\boldsymbol{\mathcal{R}}}
\newcommand{\prof}{\boldsymbol{R}}
\newcommand{\linors}{\mathscr{L}(\allalts)}

%Deliberated Judgment
\newcommand{\allargs}{S^*}
\newcommand{\args}{S}
\newcommand{\ar}{s}
\newcommand{\ileadsto}{⇝}
\newcommand{\ibeatse}{⊳_\exists}
\newcommand{\nibeatse}{⋫_\exists}
\newcommand{\ibeatsst}{⊳_\forall}
\newcommand{\nibeatsst}{⋫_\forall}
\newcommand{\mleadsto}[1][\eta]{⇝_{#1}}
\newcommand{\mbeats}[1][\eta]{⊳_{#1}}
\newcommand{\ibeatseinv}{⊳_\exists^{-1}}

%Logic
\newcommand{\ltru}{\texttt{T}}
\newcommand{\lfal}{\texttt{F}}


%\definecolor{airforceblue}{rgb}{0.36, 0.54, 0.66}
%\definecolor{ao}{rgb}{0.0, 0.0, 1.0}
%\definecolor{ao(english)}{rgb}{0.0, 0.5, 0.0}

%\journalname{}

%I find these settings useful in draft mode. Should be removed for final versions.
	%Which line breaks are chosen: accept worse lines, therefore reducing risk of overfull lines. Default = 200.
		\tolerance=2000
	%Accept overfull hbox up to...
		\hfuzz=2cm
	%Reduces verbosity about the bad line breaks.
		\hbadness 5000
	%Reduces verbosity about the underful vboxes.
		\vbadness=1300

\begin{document}
\title{Deliberation and environmental decision}
\author{Yves Meinard \and Olivier Cailloux}
\institute{
	Yves Meinard
	\and
	Olivier Cailloux 
	\at 
	Université Paris-Dauphine, \\
	PSL Research University, \\
	CNRS, \\
	LAMSADE\\
	75016 PARIS, FRANCE\\
	\email{olivier.cailloux@dauphine.fr}
}
\makeatletter
	\hypersetup{
		pdfsubject={preference},
		pdfkeywords={decision aid, justification, empirical validation, methodology}
	}
\makeatother
\maketitle

\keywords{decision aid, justification, empirical validation, methodology}

\begin{abstract}
A voluminous literature addresses the weaknesses of stated preference methods used to value non-market environmental goods and services, such as contingent valuation and choice experiment. 
Deliberative monetary valuation (DMV) has emerged as a prominent alternative to these methods. It combines deliberative institutions with preference elicitation. 
Despite an anchorage in an extensive philosophical literature on deliberative democracy, the theoretical foundations of DMV are underinvestigated.  
A noteworthy exception is \citeauthor{bartkowski_beyond_2018}'s effort to use \citeauthor{sen_idea_2009}'s philosophical views to elaborate such theoretical foundations. 
The present article pursues this theoretical effort by pointing out two issues left unanswered by the above contribution: 
the first issue is the precise role that deliberation is expected to play in DMV and, more broadly, in environmental decision-making (we term this the \emph{aim of the deliberation} issue); 
the second issue is the role that economists and consultants involved in the proceedings of deliberation are supposed to play (we term this the \emph{role of the analyst} issue). 
In order to clarify the kind of investigations that DMV or any alternative method should implement to be unequivocal on these issues, we use a formal framework introduced by \citeauthor{cailloux_formal_2018}, designed to capture the stance that an individual has on a given topic once she has performed a deliberation: her ``deliberated judgment''. 
This framework allows to identify empirical questions that DMV do not tackle whereas answering these questions would be necessary to clarify the stance that DMV takes on the \emph{aim of the deliberation} issue. 
When it comes to the \emph{role of the analyst} issue, our framework advocates an active role of the practitioner in creating what we call models of \aclp{DJ}. 
This framework helps to characterize the normative stance adopted when implementing a deliberative approach.
\end{abstract}

\section{Introduction}
A voluminous literature now addresses the practical, methodological and philosophical weaknesses of the standard economic methods used to value non-market environmental goods and services \citep{meinard_ethical_2016}. 
These standard approaches are stated preference methods such as contingent valuation and choice experiment, which are used to elicit individual willingness to pay (WTP) and then aggregate it through cost-benefit analysis. 
As summarized by \citet{bartkowski_beyond_2018}, the most important criticisms addressed at these methods raise two political and ethical concerns. 
The first one is articulated in terms of the so-called consumer-citizen dichotomy, and refers to the idea that methods based on stated preferences would discourage respondents from acting as citizens, in particular by taking into account the social implications of their choices and statements \citep{soma_representing_2014, vatn_institutional_2009}. 
The second political and ethical concern is that stated preferences methods hide the reasons underlying respondents' statements or choices, whereas from a political point of view understanding those reasons is, according to some authors like \citet{sen_environmental_1995}, perhaps even more relevant and important than the statements and choices themselves. A further concern is that respondants typically need time and thinking before they can form a meaningful answer about their willingness to pay for such subtle matters as environmental questions, but standard willingness to pay methods do not consider ways of helping respondants form their preferences.

A prominent alternative to these standard stated preference methods has emerged in the past 15 years: deliberative monetary valuation (DMV) \citep{spash_deliberative_2007,bartkowski_economic_2017}. 
These methods combine deliberative institutions, such as focus groups, with preference elicitation or choice experiments. 
The emerging literature on these methods suggests that they have important strengths as compared with standard stated preference methods, at several levels. 
At a very practical level, it appears that respondents find it easier to make sense of these methods, and are less likely to refuse to answer \citep{lienhoop_contingent_2007,szabo_reducing_2011}. 
At a philosophical level, the very label of these methods and the deliberative institutions that they use refer to the notion of deliberative democracy, which has been extensively investigated since the 1970 and now largely dominates the political philosophical scene \citep{chappell_deliberative_2012}.

Despite this anchorage in the philosophy of deliberative democracy, the theoretical foundations of DMV are arguably underinvestigated \citep{bartkowski_economic_2017,bartkowski_beyond_2018,bunse_what_2015,kenter_what_2015}. 
A noteworthy exception is \citet{bartkowski_beyond_2018}'s pioneering effort to use \citet{sen_idea_2009}'s philosophical views to elaborate such theoretical foundations. 
Although \citet{bartkowski_beyond_2018} articulate convincing solutions to some important issues in the foundations of these methods, they also leave aside some prominent theoretical problems. 
Our aim in this article is to highlight the importance of some of these remaining problems. 
More specifically, we focus on two sets of questions.
\begin{itemize}
\item A first set of questions has to do with the precise role that deliberation is expected to play in DMV and, more broadly, in environmental decision-making: what precisely is deliberation expected to do, when can one consider that there has been \emph{enough} deliberation, when can one consider that deliberation has done its job?
Let us call this set of questions the \emph{aim of the deliberation} issue.
\item A second set of questions has to do with the role of economists and consultants involved in the proceedings of deliberation: are they here simply to observe what happens, are they supposed to actively interact with participants, what are the ethical stakes of their intervention? 
Let us call this set of questions the \emph{role of the analyst} issue.
\end{itemize}
The scope of these questions is clearly larger than DMV, and even larger than \emph{environmental} issues. 
However, the pioneering research on DMV and associated methological problems have been developped in the context of environmental issues. 
In the present article, we pursue this dynamics, while emphasizing that our arguments could be adapted to other contexts.

In section 2, we argue that the current literature does not provide convincing answers to the questions above. 
In section 3, we then take advantage of a formal framework introduced by \citet{cailloux_formal_2018} to clarify the challenges raised by these two questions.
In section 4, we ponder on their philosophical meaning.

\section{Two blindspots of deliberative approaches}
In this first part, we explain why we claim that the current literature on the theoretical foundations of DMV, which in effect is largely limited to \citet{bartkowski_beyond_2018}'s contribution, does not convincingly tackle the \emph{aim of the deliberation} issue and the \emph{role of the analyst} issue.
To establish this point, we examine in this section how \citet{bartkowski_beyond_2018} use the existing literature (and we use for that purpose the very citations they refer to in their article), and we argue that their interpretation leave entirely open the \emph{aim of the deliberation} issue and the \emph{role of the analyst} issue.

Authors on DMV vary in the answer they give to a very basic question: should deliberation give rise to a consensus? 
\commentOCf{As-tu vérifié pour l’erreur possible, on cite Elster 1983 peut-être au lieu de 1982 ?}

Some authors such as \citet{vatn_institutional_2009} claim that the point of exchanges of arguments and interactions during deliberation is to reach a mutual understanding of the environmental problem that participants face and to identify a common solution to this problem. 
This vision of the point of deliberation is endorsed by versions of DMV where the goal is to reach a consensus among all the participants in the form of a single social WTP \citep{orchard-webb_deliberative_2016}.

The idea underlying the first kind of approach, according to which deliberation is conducive to consensus, is often presented as a basic tenet of deliberative democracy theory \citep{wilson_discourse-based_2002}. 
However, this idea is challenged by two important objections.
The first one is that there can exist deep moral disagreements \citep{dryzek_deliberative_2013} between some people or groups, and it might prove neither possible nor even desirable to heal such disagreements.
The second one is that “unanimity, even if sincere, could easily be spurious in the sense of deriving from conformity rather than from rational conviction” \citep{elster_sour_1983}, and could thereby reflect exclusion and power dynamics \citep{volker_exploring_2016,vargas_background_2016,vargas_problem_2017} rather than a normatively meaningful convergence of views.

\citeauthor{bartkowski_beyond_2018} contrast this consensus tenet with the notion of a ``plurality of impartial reasons'' most prominently championed by \citet{sen_idea_2009}, in part in an attempt to overcome the above problems of undesirable or spurious consensus.
This stance leads \citeauthor{bartkowski_beyond_2018} to defend an approach to DMV where consensus is neither required nor expected. 

Such an approach might seem more open-minded than the doctrine that deliberation unavoidably generates consensus. However, it evades rather than answering the question: when can one consider that there has been \emph{enough} deliberation? 
In a vision in which there is consensus if and only if there is enough deliberation, observing consensus, or the lack thereof, permits to determine whether there has been enough deliberation of the right sort. 
As soon as one acknowledges that consensus can also be generated by brute force and that deliberation can leave room to dissensus, observing a consensus loses its relevance. 
How, then, can one assess the deliberative credentials of a given DMV? 
\citeauthor{bartkowski_beyond_2018} neither raise nor answer this question.

\citeauthor{bartkowski_beyond_2018} notice that, in an approach such as theirs, an aggregation mechanism is needed \emph{after} the deliberative step.
However, \citeauthor{bartkowski_beyond_2018} do not properly discuss the aggregation issue: they highlight the usefulness of additive aggregation of individual WTP, ``for lack of better alternatives'', but do not investigate if and how using this aggregation rule could possibly solve the problem (a possible line of argument, which they do not develop, could be to argue that disagreeing people could all agree that their disagreements should be solved thanks to an additive aggregation; but this argument would require further development).\commentOCf{Je ne sais plus, mais je crois deviner que cette parenthèse est là par compromis avec une suggestion que j’avais dû faire à un moment. En tous cas, je la trouve bizarre, elle ouvre une porte uniquement pour la refermer immédiatement, sans rien dire d’utile. Je suggère de l’enlever.} \commentYM{oui, cette parenthèse est là pour la raison que tu mentiones. Je la trouve utile, mais ce n'est pas non plus la prunelle de mes yeux. On peut enlever si elle te dérange }\commentOCf{Elle me dérange pour la raison indiquée, mais je ne suis pas outré si on la laisse. Tu as le \og{}privilège\fg{} de premier auteur de pouvoir trancher au moment de la soumission.}

We therefore argue that there is a need to articulate much more clearly what the precise role of deliberation is supposed to be in DMV: this is the \emph{aim of the deliberation} issue.

This first blindspot of the DMV literature is associated with a second one. The literature on deliberative democracy, and especially the writings of its founding fathers such as \citet{rawls_political_2005} and \citet{habermas_faktizitat_1992}, is largely devoted to an investigation of the \emph{status} that philosophers are assumed to enjoy when they state the tenets of moral or political theories \citep{meinard_du_2014}. 
By contrast, the literature on DMV appears strickingly silent about the status of economists and consultants involved in the proceedings of deliberation as part of DMV, as if they were transparent and neutral observers. 
This stands in stark contrast, not only with the philosophical literature on deliberation, but also with some important contributions to the literature on decision analysis methodology and practice, emphasizing the importance of decision support interactions that consist, for the analyst, in ensuring that the aided individuals understand and accept the reasoning on which decision support is based. \commentOCf{Je retire la fin de la phrase et la citation de Roy ici, car à la réflexion c’est un peu litigieux qu’il soit un défenseur particulièrement fervent de ce point de vue. Je pense que cette phrase peut se passer de citation ; sinon, il faudrait trouver des citations plus emblématiques des discussions sur la neutralité de l’analyste (il y en a pas mal en OR).} \commentYM{je ne suis pas un exégète de Bernard suffisamment compétent pour savoir si c'est litigieux ou pas, mais il me semble qu'avec tout ce que tu as enlevé, l''argument semble creux et gratuit. Je pense qu'il faudrait restaurer la fin de la phrase, quitte à la changer si tu penses que c'est nécessaire, et mettre des références. Il suffit à mon avis de mettre une série de stars du constructivisme critque, Mingers, Ulrich, Jackson, et au niveau de détail où on est dans cette phrase, la fin de phrase de la précédente version accolée à des références ne me semblerait pas choquante. Qu'en dis-tu ?} 
\commentOCf{Ok, j’ai restauré la fin de phrase en question, mais il faut, pour ne pas déformer les propos, trouver des références qui parlent particulièrement de faire comprendre le raisonnement au décideur (chose que je croyais avoir lue chez Bernard Roy mais que je ne retrouve nulle part explicitement en cherchant rapidement, je pense que c’est plutôt moi qui ai interprété son propos de cette manière) : à la réflexion, je ne suis pas si sûr que ce soit là le focus de beaucoup de gens. Mais tu as une vision plus large que moi sur ce point.}
This is what we term the \emph{role of the analyst} issue.

These two blindspots set the agenda for our work in the remainder of this article. 
In the next section, we start by presenting a formal framework, which is a simplified version of a framework developed by \citet{cailloux_formal_2018}, and we explain how this framework allows to articulate the \emph{aim of the deliberation} and the \emph{role of the analyst} issues more clearly and transparently.

\section{A formal framework}
Let us start by spelling out what a formal framework should capture, in order to be useful for our purposes. 
We want to carve out a formal representation of the stance that an individual has on a given issue once he has performed a deliberation, where “deliberation” is understood in a precise way spelled out here below. 
Following \citet{cailloux_formal_2018}, let us call this the individual's \ac{DJ}. 
This reference to deliberation captures, in our view, two central ideas.

The first idea is that \acp{DJ} are the result of a careful examination of arguments and counter-arguments: through deliberation, the individual gathers new information, he learns about the viewpoints of other people, he takes the time to think about all these elements, 
which are all arguments for or against this or that stance. 
This idea echoes the approach to the notion of rationality developed most prominently by \citet{habermas_theorie_1981}. 
In his approach, actions, attitudes or utterances can be termed “rational” so long as the actor performing or having them can account for them, explain them and use arguments and counter-arguments to withstand criticisms that other people could raise against them. 
Variants of this vision of rationality play a key role in other prominent philosophical frameworks, such as \citeauthor{scanlon_what_2000}’s \citeyearpar{scanlon_what_2000} and \citeauthor{sen_idea_2009}’s \citeyearpar{sen_idea_2009}. 
(Using Sen's vocabulary, \citeauthor{bartkowski_beyond_2018} talk about ``reasonableness'' to stress the interpersonal aspect of this idea. 
This choice of vocabulary can create confusions, because the term ``reasonable'' is more classically associated with \citet{rawls_political_2005}, who understands this term in a different sense.) 
For simplicity's sake, we will simply talk about ``rationality'' to refer to this first idea: the \ac{DJ} of an individual is a judgment that considers all relevant arguments.

The second idea is that \acp{DJ} are nevertheless the individual's own judgments, in the sense that they do not reflect the application of any exogenous criterion. 
This second idea will be nicknamed ``anti-paternalism'' in what follows. 

We develop these ideas here in a very simplified form, and refer to the companion article \citep{cailloux_formal_2018} for a more in-depth presentation and formal definitions. 

Let us assume that a decision-maker is given -- call her $i$ -- together with a topic $\allprops$, about which $i$ wants to make up her mind. 
Let us then define $\allargs$ as a set that contains all the arguments that one can make use of when trying to make up one’s mind about $\allprops$.
Elaborate typologies of the kinds of arguments involved in environmental deliberations have been developed in the literature, for example by \citet{chateauraynaud_contrainte_2007}. 
By using here a very abstract notion of argument, we aim to encompass all the diversity included in such typologies. 

Using these notions, we want to define $i$’s stance towards the topic once a deliberation has allowed her to ponder all the arguments possibly relevant to the situation. 
Thus, we want to study changes of mind of $i$ (or the lack thereof).

This is done by defining three relations, that we assume are partially observable. We say that an argument $\ar$ supports a proposition $\prop \in T$ when $i$ declares that $\ar$ is an argument in favor of the proposition. We say that an argument $\ar$ \emph{sometimes trumps} another argument $\ar'$ when $i$ sometimes declares, upon seeing $\ar$, that in her current state of mind, $\ar'$ is not a valid argument for some of the propositions it claims to support or some of the propositions it claims to trump. We use the word ``sometimes'' because $i$ may change her mind during the process of being confronted to various arguments: she might initially, for example, state that $\ar$ does not trump $\ar'$, but then, after having seen a third argument which lets her better understand the content of $\ar$, declare that $\ar$ trumps $\ar'$. Finally, we say that an argument $\ar$ sometimes does not trump another argument $\ar'$ when $i$ sometimes declares, upon seeing $\ar$, that in her current state of mind, $\ar'$ is a valid argument for all the propositions it claims to support, and a valid argument against all the propositions it claims to trump. (Thus, in the previous example, it would hold that $\ar$ sometimes trumps $\ar'$ and $\ar$ sometimes does not trump $\ar'$.)

These relations permit to define the \ac{DJ} of $i$: it consists of all and only the propositions that are supported by at least one argument $\ar$ that is never trumped by any other argument (thus, an argument $\ar$ such that it never holds that $\ar'$ sometimes trumps $\ar$, for any $\ar'$ in $\allargs$). Those propositions are called acceptable. This definition echoes \possessivecite{rawls_political_2005} emphasis on the requirement of \emph{acceptability} by reasonable citizens. 

We repeat here (with minor modifications) a remark from the companion article defining these notions, as it links these definitions to the literature in philosophy and rhetorics. \commentYM{je trouve que cette présentation des choses donne trop l'impression qu'on recopie l'autre texte. Si effectivement ce paragraphe est trop proche du texte de base, alors mieux vaut reformuler (au demeurant, je ne me souvenais pas qu'on avait parlé d'ars rhetorica finalement dans le Th and Dec). Je me propose de relire à tête reposée les deux textes en parallèle, mais j'ai peur que même en faisant ça des similitudes m'échappent, tant que j'ai lu et relu les 2 dans des dizaines de versions...}
\commentOCf{Bof, ça ne me semble aucunement honteux de reprendre simplement tel quel cette remarque et de le dire honnêtement. Mais là aussi, je te laisse décider et reformuler si tu veux éviter de dire qu’on la reprend de l’autre article. Tu verras bien les similitudes : de mémoire, les deux passages sont je crois quasiment identiques actuellement.}
\begin{remark}
Notice also that, according to our definition, it is possible for a proposition $\prop$ to be acceptable and for not-$\prop$, or more generally for any proposition $\prop'$ in logical contradiction with $\prop$ or having empirical incompatibilities with $\prop$, to be acceptable too; depending on $i$’s declarations.
To our best knowledge, the philosophical literature elaborating on Rawls' approach to reasonableness and acceptability did not investigate this point, despite its being a natural implication of the notion of acceptability. 
This means that the notion of acceptability does not exclude the possibility that, on due reflexion, $i$ might admit that both $\prop$ and its opposite are acceptable in some cases, because in these specific cases it is possible to construct convincing arguments \emph{in utramque partem} (to use the vocabulary of Humanist \emph{ars rhetorica} \citep{skinner_reason_1996}), or because $i$ does not see significant differences between $\prop$ and $¬t$ (think about the story of Buridan’s ass).
\end{remark}

This series of definitions determines precisely when exposure to arguments is enough to consider that someone’s stance has been formed and informed by deliberation. 
This is in our view a \emph{sine qua non} condition to be able to spell out clearly an answer to the question raised by the first blindspot of the current literature on DMV: when can we admit that we have reached deliberated stances?

Moreover, tackling the two blindspots of deliberative approaches identified previously would require explicitly thinking a procedure to be performed by an analyst to capture deliberated stances. Our formal framework is useful to outline the first steps of such an endeavour. More precisely, it allows to test \emph{models} of \acp{DJ} by confronting them to the declarations of $i$.

We assume that the analyst has come up with a model of the \ac{DJ} of $i$ (given a topic $\allprops$ and a set of arguments $\allargs$), defined as follows. A model is constituted, first, by postulated supports, indicating which propositions the model considers are in the \ac{DJ} of $i$, and indicating the arguments that the model considers will be accepted by $i$ as indeed supporting those propositions. The second ingredient of a model is an argumentation strategy: it produces counter-arguments against arguments from $\allargs$ that argue against the model's claims. We do not investigate here how an analyst might elaborate such a model in practice.

\begin{example}
	\commentOCf{J’aimerais changer l’éternel exemple de la pluviosité du lendemain ; j’aimerais aussi qu’il soit relativement sans enjeux (pour qu’il ne soit pas honteux d’affirmer qu’on peut résumer le débat en quatre arguments) ; et j’aimerais qu’il ait au moins un vague rapport avec l’environnement. Je n’ai pas réussi à satisfaire toutes ces contraintes.}
	Consider $T=\set{\prop_1, \prop_2}$, where $\prop_1$ is the proposition “I should eat all the cake” and $\prop_2$ is “I should share half the cake with my flatmate”. The set of arguments $\allargs = \set{\ar_1, \ar_2, \ar_3, \ar_4, \ar_5}$, representing arguments in favor ($\ar_1, \ar_3, \ar_5$) and against ($\ar_2, \ar_4$) eating all the cake. The model claims that $i$’s \ac{DJ} contains only $\prop_1$. It uses the argument $\ar_1$ to support $\prop$. If attacked with argument $\ar_2$, the model’s argumentation strategy consists in playing the argument $\ar_3$ against $\ar_2$. The argumentation strategy does not play a response to $\ar_4$.
\end{example}

The goal of a model is to capture the \ac{DJ} of $i$. Accordingly, we say that a model is \emph{valid} if $i$’s \ac{DJ} equals the one postulated by the model.

We have so far given no operational way of testing the claims of a model: the \ac{DJ} of $i$ can’t be queried directly, as it is potentially not even known by $i$ himself. Testing validity is done thanks to, first, defining an \emph{operational validity} criterion, which, given a model, permits to validate the model using observable data only; second, a series of conditions defined on the way $i$ reasons; and third, a theorem which shows that when the conditions are satisfied, testing operational validity is enough to guarantee validity of a model.

A model is operationally valid if, whenever an argument threatens its claims (that is, when the argument sometimes trumps an argument played by the model), the model is able to produce a counter-argument, thanks to its argumentation strategy, that indeed sometimes trumps the threatening argument; and when $i$ agrees with the supports postulated by the model.

\begin{example}[cont.]
	Assume that $i$ declares that $\ar_1$ indeed supports $\prop_1$. Also, according to $i$’s declaration, $\ar_2$ sometimes trumps $\ar_1$. In such a case, deciding about the operational validity of the model requires to test the answer from the model to $\ar_2$, that is, $\ar_3$. The model will be operationally valid (concerning its claim that $\prop_1$ belongs to $i$’s \ac{DJ}) if $\ar_3$ sometimes trumps $\ar_2$ and $\ar_4$ and $\ar_5$ sometimes do not trump $\ar_1$.
\end{example}

At this stage, it is important to emphasize that, if an analyst wants to check whether the above operational validity criterion is satisfied, he does so by querying $i$ about whether a given argument trumps another argument, which can, in some case, mean that he shows to $i$ an argument that she had not thought about. 
The analyst therefore actively interacts with $i$, and possibly modifies $i$’s perspective as he checks the operational validity criterion. 
This contributes to highlight the second blindspot of the current deliberative literature.

The three conditions of interest are \emph{Justifiable unstability}; \emph{Closed under reinstatement}; and \emph{Boundedness}. \emph{Justifiable unstability} mandates that the apparent unstabilities of appreciation of arguments as declared by $i$ can be explained by other arguments playing a role in this judgment: when both $\ar'$ sometimes trumps $\ar$ and $\ar'$ sometimes does not trump $\ar$, it holds that some argument $\ar''$ sometimes trumps $\ar'$. \emph{Closed under reinstatement} mandates that whenever an argument $\ar_3$ always trumps another argument $\ar_2$, itself sometimes trumping a third argument $\ar_1$, then some argument $\ar$ exists in $\allargs$ that plays a similar argumentative role as $\ar_1$ (trumping all the arguments that $\ar_1$ trumps and supporting everything that $\ar_1$ supports), and that is not trumped by more arguments than $\ar_1$ is, and that is never trumped by $\ar_2$. The idea is that $\ar$ would adopt the same argumentation than $\ar_1$ but also incorporate the reasoning of $\ar_3$, thereby preventing the attack put forward by $\ar_2$. Finally, \emph{Boundedness} requires that the sometimes trump relation be of bounded length (including, acyclic) and width.

It is proven in the companion article that when these three conditions are met, there exists an operationally valid model, and any operationally valid model is valid. This theorem provides an operational way of testing a model’s claims, and thus, checking whether someone’s \ac{DJ} has been captured correctly, in our sense.

\begin{example}[cont.]
	Let us illustrate the way the conditions permit to obtain validity from operational validity in our toy example. Operational validity only checks that $\ar_4$ sometimes does not trump $\ar_1$, whereas for $\prop_1$ to be effectively in $i$’s \ac{DJ}, we need that $\ar_1$ be decisive, thus, that no argument ever trumps it. But, when Justifiable unstability is met, we know that, if $\ar_4$ is not itself trumped, then $\ar_4$ either always trumps, or $\ar_4$ never trumps. Because we know that $\ar_4$ sometimes does not trump, we conclude that it never trumps. Similarly, we can deduce, assuming that no argument ever trumps $\ar_3$, that $\ar_3$ always trumps $\ar_2$. Adding the observation that $\ar_2$ sometimes trumps $\ar_1$, we can apply Closed under reinstatement, and deduce the existence of an argument that also supports $\prop_1$ but that is not trumped by any more arguments than $\ar_1$ (thus, in particular, never trumped by $\ar_4$ or $\ar_5$), and also not trumped by $\ar_2$. This is the decisive argument that we look for. (This reasoning holds only under condition that $\ar_3$ and $\ar_4$ are not in turn trumped by some arguments; if they are, a more complex reasoning is required, which will involve using Closed under reinstatement repeatedly as needed to “reduce” the chains of arguments trumping each other, together with Boundedness which ensures that this process will end.)
\end{example}
	
The strength of this theorem is that it allows to say something about $i$'s \ac{DJ} despite the fact that directly identifying $i$'s \ac{DJ} is hopeless. 
But this strength has a price: it holds only if the above conditions are met. 
Therefore, in order to understand the true meaning of this theorem, and thereby to understand how likely it is that one will ever be able to correctly identify $i$'s \ac{DJ}, it is pivotal to ponder on the meaning of these conditions.

A first thing to notice in this respect is that the conditions above are quite heroic. 
One cannot realistically expect that real-life decision situations will fulfill these conditions. 
Fortunately, it is possible to weaken these conditions subtantially, by distinguishing $\allargs$ from the restricted set of arguments with which the analyst works in practice. 
This trick allows to define conditions whose meaning is very similar to the one of the simple conditions spelled out here, but which are more realistic because they refer to a restricted set of arguments. 
This task however involves quite a lot of technical subtelties. We refer to \citet{cailloux_formal_2018} for a complete exposition. For our purposes here, it will not be necessary to delve into these technicalities. 
Indeed, because the weaker and simpler conditions have very similar meanings, we can content ourselves with a discussion of the simple conditions. 
This will provide all we need for the purpose of our methodological investigation into the theoretical foundations of DMV and deliberative approaches.

In general terms, our conditions can be interpreted in three different ways:
\begin{enumerate}[label=\emph{\roman*}, ref=\emph{\roman*}]
	\item \label{inter:axioms} as rationality requirements capturing minimal properties concerning arguments and the way $i$ reasons,
	\item \label{inter:empir} as empirical hypotheses,
	\item \label{inter:rules} as rules governing the decision support process (rules that $i$ can commit to abide by, or can consider to be well-founded safeguards for the proper unfolding of the process).
\end{enumerate}

Some of the conditions of our theorem are arguably more congenial to a given interpretation. 
We do not want, here, to take a rigid stance on which interpretation or mix of interpretations should be preferred. 
Our point is rather to stress that, whatever the chosen interpretation, it has implications for analysts implementing deliberation and DMV in practice. 
If the conditions are understood as rationality requirements (\ref{inter:axioms}), analysts have to be able to entrench their justifiability. 
This might be possible, at least in some cases, for some of these conditions. 
If the conditions are understood as empirical hypotheses, then analysts have to be able to empirically test them, which raises difficulties falling beyond the scope of the present article.
Lastly, if they are understood as rules (\ref{inter:rules}), then, even more than through the process of operational validity checking, the analyst is engaged in an active interaction with and modification of the decision process, that it should be able to account for and justify.

\section{Meaning and perspectives}
\label{disc}
Before coming back to deliberation and DMV, it is useful to take a broader view to notice that the two notions of rationality and anti-paternalism, whose formalization provides the basis for our framework, play an important role in contemporary normative philosophy. 
Indeed, the search for a compromise, an equilibrium or a satisfactory articulation between these two requirements can be seen as a key thread running through contemporary political philosophy.

A case in point is Rawls’s notion of a “reflective equilibrium”. 
Following \citet{goodman_fact_1983}, \citet[][p.18]{rawls_theory_1999} used “reflective equilibrium” to refer to a “process of mutual adjustment of principles and considered judgments”. 
This formulation highlights that, in Rawls's \emph{Theory of Justice}, a prominent role attributed to this concept was to do justice both to people's moral intuitions and to the need to systematize visions of justice. 
Rawls thereby granted a prominent importance to people's own judgments (both as to how specific cases should be adjudicated and as to whether a given general principle is acceptable), which is what we termed  “anti-paternalism”. 
As for the reference to principles, and the idea that the judgments to be taken into account are the ones that can be termed “considered”, they echo our rationality requirement, if one admits that judgments are  “considered” when they are buttressed on a careful analysis of arguments and counterarguments, and that principles systematizing considered judgments provide arguments in favor of these judgments. 
Rawls's “reflective equilibrium” hence embodies the two ideas forming the core of our concept of \ac{DJ}. 
The credibility of this interpretation is reinforced by the fact that \citeauthor{rawls_political_2005}’s \citeyearpar{rawls_political_2005} later work grants an increasing importance to the notion of “due reflection” --- a notion that does not refer to principles and is more general than the one of “reflective equilibrium”. 

This broader philosophical perspective also usefully points another important debate. 
A prominent aspect of the concept of reflective equilibrium that our reasoning has so far set aside is its purported interpersonal dimension. 
In \emph{A Theory of Justice}, the reflective equilibrium is not presented as the result of the endeavor of an insulated individual, but is rather defined from the very beginning in collective terms. 
Similarly, when Rawls makes use of the concept in \emph{Political Liberalism}, he presents it as a  “device of representation” that citizens can use to calibrate their public discussions. 
Like many other key concepts of the rawlsian framework, the one of reflective equilibrium is hence systematically presented by Rawls in pluri-individual settings --- another notorious example being the “parties” choosing the principles of justice, which are unequivocally presented as a collective. 
At first sight one might object to our approach that it lacks such an interpersonal dimension.
If true, this would be a worrying weakness, especially given that some authors such as \citet{vatn_institutional_2009} argue that one of the distinctive strengths of deliberative approaches is that they lead people to reason according to ``We-rationality'', as opposed to ``I-rationality''. 
It is therefore important to stress how the interpersonal dimension comes into play in our approach, and to contrast it with how it comes into play in some prominent philosophical approaches.

Rawls's claim to integrate an interpersonal dimension has been fiercely criticized in the literature. 
In particular, \citet{habermas_short_1999} famously argued that Rawls's approach involved a preemption, by the philosopher, of issues that have their proper place in public debates among citizens. 
Seen through these lenses, the proximity between our framework and Rawls's theory might seem to reinforce the impression that our framework lacks this dimension.

\citeauthor{habermas_moralbewustsein_1983}'s \citeyearpar{habermas_moralbewustsein_1983} attempt at overcoming this problem is of particular significance from our point of view. 
In his theory, the content of the theory of justice should not be seen as the result of an explicit deliberation or reflexion, but are rather the result of a transcendental deduction (though of a weaker sort that the Kantian one) --- that is, they are demonstrated to be conditions of possibility of all sorts of interactions mediated by communication in a society. As opposed to these so-called “moral” tenets, a given “ethical” notion can be consensually accepted in a given society or group, but can become a bone of contention when various groups meet or merge. 
Thus, in this approach, there can exist a dissensus between various individuals in the society on many issues. 
But when it comes to the subject matter of moral theory, any dissensus is bound to be ephemeral, because the very process of communication through which people try to settle their disagreement presupposes an implicit acceptance of the tenets that moral theory aims to capture.

Our frameworks shares with this reasoning of Habermas's the idea that it is important to distinguish different kinds of issues about which people can agree or disagree: disagreements are not necessarily fatal to deliberation, but disagreements on \emph{some fondamental issues}, pertaining to the role of deliberation, are fatal. Habermas's purported solution through a transcendantal deduction has itself been criticized \citep{heath_communicative_2001}. 
In particular, it is not self-evident that there is such a thing as a determinate set of conditions of possibility common to all sorts of interactions mediated by communication in a society. 
Habermas's theory hence appears to be plagued by the same problem that he denonced in Rawls's theory: both wanted to prove that consensus will occur on certain issues, but they both collapsed in ``theoretical preemption''. Rawls only provided conceptual reasons to believe that the “considered judgments” of various individuals in a society should converge towards an “overlapping consensus”, and these concepual reasons fail to win over everyone, even among philosophers largely inspired by Rawls \citep{estlund_insularity_1998, estlund_democratic_2009}. 
Similarly Habermas's theory is weakened by the fact that its fixed point (the content of moral theory, the so-called “U” and “D” tenets \citep{habermas_moralbewustsein_1983}) is a purely conceptual finding that happens to arouse conceptual criticisms.

Our approach ventures an alternative solution to this problem. 
We have to face the fact that, in situations in which various people diverge in their considered judgments, collective deliberation and decision-making unavoidably face issues of aggregation of diverging attitudes. 
One cannot simply take this idea as a mandate to accept the divergence of preference and additively aggregate them, as \citeauthor{bartkowski_beyond_2018} suggest, because the relevance of such aggregation must in turn be justified. 
There is therefore a crucial need to empirically inquire whether such problems occur, or can occur, and if so in which conditions. Indeed, in such conditions, the relevance of deliberative approaches is questionable. 
Like any method, DMV and other deliberative methods have a potentially restricted domain of application, and any application of these methods should therefore start by checking if it indeed falls in its domain of application. 
For such a verification, an empirical approach is needed. 
Both Rawls, Habermas, Sen and the DMV literature as a whole lack such an approach. 
To put it otherwise, a precise definition of the \ac{DJ}, together with empirical tests such as the one proposed here, open an avenue of research aiming to detect which protocols of deliberation better achieve the goal of capturing the participants’ \ac{DJ}.
Though the concrete details of such empirical studies fall beyond the scope of the present article, our framework spelled out above provides a backbone for such an approach.

That said, to conclude, we have to emphasize that, despite the promises offered by our framework, this is no magical tool. 
In particular, we do not claim that it overcomes Hume's law by providing the means to deduce ought from is. 
As repeatedly emphazised, our approach is anchored in two unmistakably normative tenets: rationality and anti-paternalism. 
Our reasoning in \cref{disc} was devoted to show that these two tenets play a very fundamental role in contemporary normative philosophy, and can therefore in that very specific sense be considered minimal. 
But one could accordingly argue that the criticism that we address at Habermas also applies to our framework, one step deeper. This is certainly true. 
We accordingly do not claim that it solves all the philosophical difficulties of the theory of deliberative democracy, DMV and their fundations. 
We rather attempted to introduce an approach that allows to go deeper than the current literature in the direction of providing strong philosophical foundations to empirical applications of philosophies of deliberation, such as DMV. 

\begin{acknowledgements}
We thank xxxx for very helpful comments.
\end{acknowledgements}

%TODO remove for final version
\hbadness 10000

\bibliography{beliefs,philo-eco,deliber,manual}
\end{document}
