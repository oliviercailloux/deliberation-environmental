\RequirePackage{amsmath}
\RequirePackage[l2tabu, orthodox]{nag}
%\documentclass[smallextended,nospthms,natbib]{svjour3}
\documentclass[version=3.21, pagesize, twoside=off, bibliography=totoc, DIV=calc, fontsize=12pt, a4paper, french, english]{scrartcl}
\newcommand{\institute}[1]{}
\newcommand{\keywords}[1]{}
\newenvironment{acknowledgements}{
	\section*{Acknowledgements}
}{
}
%\makeatletter \let\cl@chapter\relax \makeatother
%\smartqed  % flush right qed marks, e.g. at end of proof 
%INSTALL
\pdfgentounicode=1 %permits (with package glyphtounicode) to copy eg x ⪰ y iff v(x) ≥ v(y) from pdf to unicode data. 
\input{glyphtounicode}%TODO avoid warning when redefining → (and others) ; make it work for ℝ (U+211D) as well
\usepackage[T1]{fontenc}%encode resulting accented characters correctly in resulting PDF, permits copy-paste
\usepackage[utf8]{inputenc}
\usepackage{newunicodechar}%able to use e.g. → or ≤ in source
\usepackage{lmodern}%has more characters such as ligatures, permit copy from resulting PDF.
\usepackage{textcomp}%useful for redefining → and ¬ and … (otherwise seems to attempt to use \textrightarrow and the like but are not defined)
% solves bug in lmodern, https://tex.stackexchange.com/a/261188/
\DeclareFontShape{OMX}{cmex}{m}{n}{
  <-7.5> cmex7
  <7.5-8.5> cmex8
  <8.5-9.5> cmex9
  <9.5-> cmex10
}{}

\SetSymbolFont{largesymbols}{normal}{OMX}{cmex}{m}{n}
\SetSymbolFont{largesymbols}{bold}  {OMX}{cmex}{m}{n}
%warn about missing characters
\tracinglostchars=2

%REDAC
\usepackage{booktabs}
\usepackage{calc}
\usepackage{tabularx}

\usepackage{etoolbox} %for addtocmd, newtoggle commands
\newtoggle{LCpres}
\togglefalse{LCpres}

\usepackage{mathtools} %load this before babel!
	\mathtoolsset{showonlyrefs,showmanualtags}

\usepackage{natbib}%Package frenchb asks to load natbib before babel/frenchb

%\usepackage[super]{nth}%better use fmtcount! (loaded by datetime anyway; see below about pbl with warnings and package silence)
\usepackage{listings} %typeset source code listings
	\lstset{language=XML,tabsize=2,captionpos=b,basicstyle=\NoAutoSpacing}%NoAutoSpacing avoids space before colon or ?}%,literate={"}{{\tt"}}1, keywordstyle=\fontspec{Latin Modern Mono Light}\textbf, emph={String, PreparedStatement}, emphstyle=\fontspec{Latin Modern Mono Light}\textbf, language=Java, basicstyle=\small\NoAutoSpacing\ttfamily, frame=single, aboveskip=0pt, belowskip=0pt, showstringspaces=false
\usepackage[nolist,smaller,printonlyused]{acronym}%,smaller option produces warnings from relsize in some cases, it seems.% Note silence and acronym and hyperref make (xe)latex crash when ac used in section (http://tex.stackexchange.com/questions/103483/strange-packages-interaction-acronyms-silence-hyperref), rather use \section{\texorpdfstring{\acs{UE}}{UE}}.
\usepackage{fmtcount}
\usepackage[nodayofweek]{datetime}%must be loaded after the babel package. However, loading it after {nth} generates a warning from fmtcount about ordinal being already defined. Better load it before nth? (then we can remove the silence package which creates possible crashes, see above.) Or remove nth?
%\usepackage{xspace}%do we need this?
\nottoggle{LCpres}{
	\usepackage[textsize=small]{todonotes}
}{
}
\iftoggle{LCpres}{
	%remove pdfusetitle (implied by beamer)
	\usepackage{hyperref}
}{
% option pdfusetitle must be introduced here, not in hypersetup.
	\usepackage[pdfusetitle]{hyperref}
}
\nottoggle{LCpres}{
%seems like authblk wants to be later than hyperref, but sooner than silence
\usepackage{authblk}
\renewcommand\Affilfont{\small}
}{
}
\usepackage{silence}
\WarningFilter{newunicodechar}{Redefining Unicode character}
%breaklinks makes links on multiple lines into different PDF links to the same target.
%colorlinks (false): Colors the text of links and anchors. The colors chosen depend on the the type of link. In spite of colored boxes, the colored text remains when printing.
%linkcolor=black: this leaves other links in colors, e.g. refs in green, don't print well.
%pdfborder (0 0 1, set to 0 0 0 if colorlinks): width of PDF link border
%hidelinks or: colorlinks, linkcolor=black, citecolor=black, urlcolor={blue!80!black}
\hypersetup{breaklinks, bookmarksopen}
%add hyperfigures=true in hypersetup (already defined in article mode)
\iftoggle{LCpres}{
	\hypersetup{hyperfigures}
}{
}

%in Beamer, sets url colored links but does not change the rest of the colors (http://tex.stackexchange.com/questions/13423/how-to-change-the-color-of-href-links-for-real)
%\hypersetup{breaklinks,bookmarksopen,colorlinks=true,urlcolor=blue,linkcolor=,hyperfigures=true}
% hyperref doc says: Package bookmark replaces hyperref’s bookmark organization by a new algorithm (...) Therefore I recommend using this package.
\usepackage{bookmark}

% center floats by default, but do not use with float
% \usepackage{floatrow}
% \makeatletter
% \g@addto@macro\@floatboxreset\centering
% \makeatother
\nottoggle{LCpres}{
	\usepackage{enumitem} %follow enumerate by a string saying how to display enumeration
}{
}
\usepackage{ragged2e} %new com­mands \Cen­ter­ing, \RaggedLeft, and \RaggedRight and new en­vi­ron­ments Cen­ter, FlushLeft, and FlushRight, which set ragged text and are eas­ily con­fig­urable to al­low hy­phen­ation (the cor­re­spond­ing com­mands in LaTeX, all of whose names are lower-case, pre­vent hy­phen­ation al­to­gether). 
\usepackage{siunitx} %[expproduct=tighttimes, decimalsymbol=comma] ou (plus récent ?) [round-mode=figures, round-precision=2, scientific-notation = engineering]
\sisetup{detect-all, locale = FR, strict}% to detect e.g. when in math mode (use a math font) - check whether this makes sense with strict
\usepackage{braket} %for \Set
\usepackage{doi}

\usepackage{amsmath,amsthm}
\usepackage{amssymb}%for \mathbb{R} %includes amsfonts
\usepackage{bm}%“The \boldsymbol command is obtained preferably by using the bm package, which provides a newer, more powerful version than the one provided by the amsmath package. Generally speaking, it is ill-advised to apply \boldsymbol to more than one symbol at a time.” — AMS Short math guide. “If no bold font appears to be available for a particular symbol, \bm will use ‘poor man’s bold’” — bm
% \usepackage{dsfont} %for what?

\usepackage{environ}%for xdescwd command
%BUT see https://tex.stackexchange.com/questions/83798/cleveref-varioref-missing-endcsname-inserted for cleveref with french
\usepackage{cleveref}% cleveref should go "laster" than hyperref
%GRAPHICS
\usepackage{pgf}
\usepackage{pgfplots}
	\usetikzlibrary{babel, matrix, fit, plotmarks, calc, trees, shapes.geometric, positioning, plothandlers, arrows, shapes.multipart}
\pgfplotsset{compat=1.14}
\usepackage{graphicx}

\DeclareMathAlphabet\mathbfcal{OMS}{cmsy}{b}{n}

\graphicspath{{graphics/},{graphics-dm/}}
\DeclareGraphicsExtensions{.pdf}
\newcommand*{\IncludeGraphicsAux}[2]{%
	\XeTeXLinkBox{%
		\includegraphics#1{#2}%
	}%
}%

%HACKING
\usepackage{printlen}
\uselengthunit{mm}
% 	\newlength{\templ}% or LenTemp?
% 	\setlength{\templ}{6 pt}
% 	\printlength{\templ}
\usepackage{scrhack}% load at end. Corrects a bug in float package, which is outdated but might be used by other packages
\usepackage{mathrsfs}% for \mathscr
%see https://tex.stackexchange.com/questions/409212/size-substitution-with-fontsize-14
\DeclareFontFamily{U}{rsfs}{\skewchar\font127 }
\DeclareFontShape{U}{rsfs}{m}{n}{%
   <-6.5> rsfs5
   <6.5-8> rsfs7
   <8-> rsfs10
}{}

%Beamer-specific
%do not remove babel, which beamer uses (beamer uses the \translate command for the appendix); but french can be removed.
\iftoggle{LCpres}{
	\usepackage{appendixnumberbeamer}
	\setbeamertemplate{navigation symbols}{} 
	\usepackage{preamble/beamerthemeParisFrance}
	\usefonttheme{professionalfonts}
	\setcounter{tocdepth}{10}
	%From: http://tex.stackexchange.com/questions/168057/beamer-with-xelatex-on-texlive2013-enumerate-numbers-in-black
%I don’t think it’s useful to submit this as a bug: nothing has been solved since March, 2015. See: https://bitbucket.org/rivanvx/beamer/issues?status=resolved.

\setbeamertemplate{enumerate item}
{
  \begin{pgfpicture}{-1ex}{-0.65ex}{1ex}{1ex}
    \usebeamercolor[fg]{item projected}
    {\pgftransformscale{1.75}\pgftext{\normalsize\pgfuseshading{bigsphere}}}
    {\pgftransformshift{\pgfpoint{0pt}{0.5pt}}
      \pgftext{\usebeamercolor[fg]{item projected}\usebeamerfont*{item projected}\insertenumlabel}}
  \end{pgfpicture}%
}

\setbeamertemplate{enumerate subitem}
{
  \begin{pgfpicture}{-1ex}{-0.55ex}{1ex}{1ex}
    \usebeamercolor[fg]{subitem projected}
    {\pgftransformscale{1.4}\pgftext{\normalsize\pgfuseshading{bigsphere}}}
    \pgftext{%
      \usebeamercolor[fg]{subitem projected}%
      \usebeamerfont*{subitem projected}%
      \insertsubenumlabel}
  \end{pgfpicture}%
}

\setbeamertemplate{enumerate subsubitem}
{
  \begin{pgfpicture}{-1ex}{-0.55ex}{1ex}{1ex}
    \usebeamercolor[fg]{subsubitem projected}
    {\pgftransformscale{1.4}\pgftext{\normalsize\pgfuseshading{bigsphere}}}
    \pgftext{%
      \usebeamercolor[fg]{subsubitem projected}%
      \usebeamerfont*{subitem projected}%
      \insertsubsubenumlabel}
  \end{pgfpicture}%
}


}{
}
% \newcommand{\citep}{\cite}%Better: leave natbib.
% \setbeamersize{text margin left=0.1cm, text margin right=0.1cm} 
% \usetheme{BrusselsBelgium}%no, replace with paris
%\usetheme{ParisFrance}, no, usepackage better!
% Tex Gyre takes too much space, replace with Latin Modern Roman / Sans / Mono.
% Difference when loading explicitly Latin Modern Sans (compared to not using \setsansfont at all):
% the font LMSans17-Regular appears in the document;
% the title of the slides appears differently;
% it does not say (in the log file):
% > LaTeX Font Info:    Font shape `EU1/lmss/m/it' in size <10.95> not available
% > (Font)              Font shape `EU1/lmss/m/sl' tried instead on input line 85.
% > LaTeX Font Info:    Try loading font information for EU1+lmtt on input line 85.

%tikzposter-specific
%remove \usepackage{ragged2e}: causes 1=1 to be printed in the middle of the poster. (Anyway prints a warning about those characters being missing.)
%put [french, english] options next to \usepackage{babel} to avoid warning of 

\newcommand{\commentOC}[1]{{\small\color{blue}$\big[$OC: #1$\big]$}}
\newcommand{\commentOCf}[1]{{\small\color{blue}{\selectlanguage{french}$\big[$OC : #1$\big]$}}}
\newcommand{\commentYM}[1]{{\small\color{red}{\selectlanguage{french}$\big[$YM : #1$\big]$}}}
\newcommand{\innote}[1]{{\scriptsize{#1}}}
%Or: \todo[color=green!40]

%this probably requires outdated float package, see doc KomaScript for an alternative.
% \newfloat{program}{t}{lop}
% \floatname{program}{PM}

%definition, theorem, lemma, example environments, qed trickery are only needed in article mode (not Beamer)
\nottoggle{LCpres}{
%style is plain by default (italic text)
	\newtheorem{definition}{Definition}
	\newtheorem{theorem}{Theorem}
%no italic: expected.
%http://tex.stackexchange.com/questions/144653/italicizing-of-amsthm-package
	\newtheorem{lemma}{Lemma}
	\newtheorem{condition}{Condition}
%\crefname{axiom}{axiom}{axioms}%might be needed for workaround bug in cref when defining new theorems?

%\ifdefined\theorem\else
%\newtheorem{theorem}{\iflanguage{english}{Theorem}{Théorème}}
%\fi

\theoremstyle{remark}
	\newtheorem{examplex}{Example}
	\newtheorem{remarkx}{Remark}

%trickery allowing use of \qedhere and the like.
\newenvironment{example}{
	\pushQED{\qed}\renewcommand{\qedsymbol}{$\triangle$}\examplex
}{
	\popQED\endexamplex
}
\newenvironment{remark}{
	\pushQED{\qed}\renewcommand{\qedsymbol}{$\triangle$}\remarkx
}{
	\popQED\endremarkx
}
}{
}
\crefname{examplex}{example}{examples}% I wonder why this is unnecessary in case of singular
\crefname{condition}{condition}{conditions}
\makeatletter
\cref@addlanguagedefs{french}{%
	\crefname{examplex}{exemple}{exemples}%
}
\makeatother

%trickery allowing use of \qedhere and the like.
%\newcommand{\xqed}[1]{%
%    \leavevmode\unskip\penalty9999 \hbox{}\nobreak\hfill
%    \quad\hbox{#1}}
%\AtEndEnvironment{example}{
%	\xqed{$\triangle$}%
%}
%\AtEndEnvironment{proof}{
%	\qed%
%}

%which line breaks are chosen: accept worse lines, therefore reducing risk of overfull lines. Default = 200
%\tolerance=2000
%accept overfull hbox up to...
%\hfuzz=2cm
%reduces verbosity about the bad line breaks
%\hbadness 5000
%sloppy sets tolerance to 9999
%\apptocmd{\sloppy}{\hbadness 10000\relax}{}{}

\bibliographystyle{abbrvnat}
%or \bibliographystyle{apalike} for presentations?

%doi package uses old-style dx.doi url, see 3.8 DOI system Proxy Server technical details, “Users may resolve DOI names that are structured to use the DOI system Proxy Server (http://doi.org (preferred) or http://dx.doi.org).”, https://www.doi.org/doi_handbook/3_Resolution.html
\makeatletter
\patchcmd{\@doi}{dx.doi.org}{doi.org}{}{}
\makeatother

% WRITING
%\newcommand{\ie}{i.e.\@\xspace}%to try
%\newcommand{\eg}{e.g.\@\xspace}
%\newcommand{\etal}{et al.\@\xspace}
\newcommand{\ie}{i.e.\ }
\newcommand{\eg}{e.g.\ }
\newcommand{\mkkOK}{\checkmark}%\color{green}{\checkmark}
\newcommand{\mkkREQ}{\ding{53}}%requires pifont?%\color{green}{\checkmark}
\newcommand{\mkkNO}{}%\text{\color{red}{\textsf{X}}}

\makeatletter
\newcommand{\boldor}[2]{%
	\ifnum\strcmp{\f@series}{bx}=\z@
		#1%
	\else
		#2%
	\fi
}
\newcommand{\textstyleElProm}[1]{\boldor{\MakeUppercase{#1}}{\textsc{#1}}}
\makeatother
\newcommand{\electre}{\textstyleElProm{Électre}\xspace}
\newcommand{\electreIv}{\textstyleElProm{Électre Iv}\xspace}
\newcommand{\electreIV}{\textstyleElProm{Électre IV}\xspace}
\newcommand{\electreIII}{\textstyleElProm{Électre III}\xspace}
\newcommand{\electreTRI}{\textstyleElProm{Électre Tri}\xspace}
% \newcommand{\utadis}{\texorpdfstring{\textstyleElProm{utadis}\xspace}{UTADIS}}
% \newcommand{\utadisI}{\texorpdfstring{\textstyleElProm{utadis i}\xspace}{UTADIS I}}

%TODO
% \newcommand{\textstyleElProm}[1]{{\rmfamily\textsc{#1}}} 

\newcommand{\txtlaxiom}{$\alllang$\hyp{}axiom}
\newcommand{\txtlaxioms}{$\alllang$\hyp{}axioms}
\newcommand{\laxiomatisation}{$\alllang$\hyp{}axiomatisation}
\newcommand{\laxiomatization}{$\alllang$\hyp{}axiomatization}
%bold. Requires special command for math bold cal!
\newcommand{\laxiomatisationbd}{$\mathbfcal{L}$\textbf{\hyp{}axiomatisation}}
%\newcommand{\laxiomatisationtitle}{\texorpdfstring{$\alllang$\hyp{}Axiomatisation}{L\hyp{}Axiomatisation}}%Does not work because of AAMAS font switching command (tries to typeset it using ptm font), so we should use something like {\usefont{EU1}{lmr}{m}{n}©} but manually select the right font size and weight (which I ignore).
\newcommand{\laxiomatisationtitle}{\texorpdfstring{{\large$\mathcal{L}$}\hyp{}Axiomatisation}{ℒ\hyp{}Axiomatisation}}
%\commentUE{Should we maybe write $\ell$-axiom rather than l-axiom, for better readability (to avoid ell being read as one)? Or we could even write $\mathcal{L}$-axiom, to have an explicit reference to the language in which the axiom is expressed?}

\newunicodechar{ℕ}{\mathbb{N}}
\newunicodechar{ℝ}{\mathbb{R}}
\newunicodechar{−}{\ifmmode{-}\else\textminus\fi}
\newunicodechar{≠}{\ensuremath{\neq}}
\newunicodechar{≤}{\ensuremath{\leq}}
\newunicodechar{≥}{\ensuremath{\geq}}
\newunicodechar{≻}{\succ}
\newunicodechar{⊁}{\nsucc}
\newunicodechar{▷}{\triangleright}
\newunicodechar{⋫}{\ntriangleright}
\newunicodechar{→}{\ifmmode\rightarrow\else\textrightarrow\fi}
\newunicodechar{⇒}{\ensuremath{\Rightarrow}}
\newunicodechar{⇏}{\ensuremath{\nRightarrow}}
\newunicodechar{⇔}{\ensuremath{\Leftrightarrow}}
\newunicodechar{∪}{\cup}
\newunicodechar{∩}{\cap}
\newunicodechar{∧}{\land}
\newunicodechar{∨}{\lor}
\newunicodechar{¬}{\ifmmode\lnot\else\textlnot\fi}
\newunicodechar{…}{\ifmmode\ldots\else\textellipsis\fi}
\newunicodechar{×}{\ifmmode\times\else\texttimes\fi}
\newunicodechar{γ}{\ensuremath{\gamma}}
\newunicodechar{□}{\Box}


\newcommand{\R}{ℝ}
\newcommand{\N}{ℕ}
\newcommand{\Z}{ℤ}
\newcommand{\card}[1]{\lvert{#1}\rvert}
\newcommand{\powerset}[1]{\mathscr{P}(#1)}%\mathscr rather than \mathcal: scr is rounder than cal (at least in XITS Math).
%powerset without zero
\newcommand{\powersetz}[1]{\mathscr{P}_\emptyset(#1)}
\newcommand{\suchthat}{\;\ifnum\currentgrouptype=16 \middle\fi|\;}
%\newcommand{\Rplus}{\reels^+\xspace}

\AtBeginDocument{%
	\renewcommand{\epsilon}{\varepsilon}
% we want straight form of \phi for mathematics, as recommended in UTR #25: Unicode support for mathematics.
%	\renewcommand{\phi}{\varphi}
}

% with amssymb, but I don’t want to use amssymb just for that.
% \newcommand{\restr}[2]{{#1}_{\restriction #2}}
%\newcommand{\restr}[2]{{#1\upharpoonright}_{#2}}
\newcommand{\restr}[2]{{#1|}_{#2}}%sometimes typed out incorrectly within \set.
%\newcommand{\restr}[2]{{#1}_{\vert #2}}%\vert errors when used within \Set and is typed out incorrectly within \set.
\DeclareMathOperator*{\argmax}{arg\,max}
\DeclareMathOperator*{\argmin}{arg\,min}


%Decision Theory (MCDA and SC)
\newcommand{\allalts}{\mathscr{A}}
\newcommand{\allcrits}{\mathscr{C}}
\newcommand{\alts}{A}
\newcommand{\dm}{i}
\newcommand{\allF}{\mathscr{F}}
\newcommand{\allvoters}{\mathscr{N}}
\newcommand{\voters}{N}
\newcommand{\allprofs}{\boldsymbol{\mathcal{R}}}
\newcommand{\prof}{\boldsymbol{R}}
\newcommand{\linors}{\mathscr{L}(\allalts)}

%Deliberated Judgment
\newcommand{\allargs}{S^*}
\newcommand{\args}{S}
\newcommand{\ar}{s}
\newcommand{\ileadsto}{⇝}
\newcommand{\ibeatse}{⊳_\exists}
\newcommand{\nibeatse}{⋫_\exists}
\newcommand{\ibeatsst}{⊳_\forall}
\newcommand{\nibeatsst}{⋫_\forall}
\newcommand{\mleadsto}[1][\eta]{⇝_{#1}}
\newcommand{\mbeats}[1][\eta]{⊳_{#1}}
\newcommand{\ibeatseinv}{⊳_\exists^{-1}}

%Logic
\newcommand{\ltru}{\texttt{T}}
\newcommand{\lfal}{\texttt{F}}


%\definecolor{airforceblue}{rgb}{0.36, 0.54, 0.66}
%\definecolor{ao}{rgb}{0.0, 0.0, 1.0}
%\definecolor{ao(english)}{rgb}{0.0, 0.5, 0.0}

%\journalname{}

%I find these settings useful in draft mode. Should be removed for final versions.
	%Which line breaks are chosen: accept worse lines, therefore reducing risk of overfull lines. Default = 200.
		\tolerance=2000
	%Accept overfull hbox up to...
		\hfuzz=2cm
	%Reduces verbosity about the bad line breaks.
		\hbadness 5000
	%Reduces verbosity about the underful vboxes.
		\vbadness=1300

\linenumbers
\begin{document}
\title{Deliberation and environmental decision}
\author{Yves Meinard \and Olivier Cailloux}
\institute{
	Yves Meinard
	\and
	Olivier Cailloux 
	\at 
	Université Paris-Dauphine, \\
	PSL Research University, \\
	CNRS, \\
	LAMSADE\\
	75016 PARIS, FRANCE\\
	\email{olivier.cailloux@dauphine.fr}
}
\makeatletter
	\hypersetup{
		pdfsubject={preference},
		pdfkeywords={decision aid, justification, empirical validation, methodology}
	}
\makeatother
\maketitle

\keywords{decision aid, justification, empirical validation, methodology} 

\begin{abstract}
\commentYM{à ajuster en fonctin de nos modifications, pas encore touché} A voluminous literature addresses the weaknesses of standard stated preference methods used to value non-market environmental goods and services, such as contingent valuation and choice experiment. 
Deliberative monetary valuation (DMV) has emerged as a prominent alternative to the standard versions of these methods. It combines deliberative institutions with preference elicitation. 
Despite an anchorage in an extensive philosophical literature on deliberative democracy, the theoretical foundations of DMV are underinvestigated.  
A noteworthy exception is \citet{bartkowski_beyond_2018}’s effort to use \citeauthor{sen_idea_2009}’s philosophical views to elaborate such theoretical foundations. 
The present article pursues this theoretical effort by pointing out two issues left unanswered by the above contribution: 
the first issue is the precise role that deliberation is expected to play in DMV and, more broadly, in environmental decision-making (we term this the \emph{aim of the deliberation} issue); 
the second issue is the role that economists and consultants involved in the proceedings of deliberation are supposed to play (we term this the \emph{role of the analyst} issue). 
In order to clarify the kind of investigations that DMV or any alternative method should implement to be unequivocal on these issues, we use a formal framework introduced by \citet{cailloux_formal_2019}, designed to capture the stance that an individual has on a given topic once she has participated in a deliberation: her “deliberated judgment”. 
This framework allows to identify empirical questions that DMV do not tackle whereas answering these questions would be necessary to clarify the stance that DMV takes on the \emph{aim of the deliberation} issue. 
When it comes to the \emph{role of the analyst} issue, our framework advocates an active role of the practitioner in creating what we call models of \aclp{DJ}. 
This framework helps to characterize the normative stance adopted when implementing a deliberative approach.
\end{abstract}
\acresetall

\section{Introduction}
\commentOCf{Après lecture de Schaafsma et réflexion, je m’interroge encore. Ils écrivent que The reasons for combining SP valuation with deliberative elements generally fall into three categories: substantive, normative and instrumental (Chilvers, 2009). Si je résume, instrumental correspond aux Deliberated Preference for Orchard-Webb (et les nôtres, on dirait), AKA Valuation Workshops. Obtain Individual WTP and choices. Normative: democratic values per se. Institutional: increase acceptance, building trust, reduce conflict. Both: DDMV for Orchard-Webb. Value Juries. Social type. Je dirais en conséquence que ce que nous faisons ici consiste à clarifier ce qui est déjà reconnu comme un des objectifs de la délibération. Cependant, on propose une toute autre voie pour le faire, non ? Valider un modèle d’un analyse, c’est autre chose que faire discuter des gens. Je ne pense pas que tour ceci soit clair à l’heure actuelle dans l’article. Par ailleurs, je me dis qu’on pourrait tenter d’atteindre le même objectif (stabilité des jugements) sans modèle. Nous pourrions comparer et déterminer les avantages de chaque approche.}

\commentYM{j’ai tâché de gérer le point ci-dessus en changeant la formulation de notre première issue}

\commentRa{How could the conditions you formulated in this article be used to inform the practical design of DMV studies?}

\commentYM{il faut qu’on dise quelques mots d’implications concrètes, mais je ne sais pas encore quoi précisément}

\commentRa{Are there maybe examples from the empirical literature that applied desirable procedures (without knowing about your framework)? 

Does your conceptual framework imply that some of the guidelines formulated by Schaafsma et al (2018) are wrongheaded?}

\commentYM{je pense que dans la nouvelle formulation est est clair que ces recommandations sont insuffisantes}

\commentRa{You switch between “aim of the deliberation” and “role of the deliberation”. In any case, I am sceptical of this particular term/phrase – it may be confused with other issues, such as discussed by Schaafsma et al. (2018) in their section on “Purpose and Chosen Theoretical Foundation”. However, your focus is different. So I’d suggest you rename this challenge/open question/issue.}
\commentYM{fair enough, la première issue était mal nommé. La second par contre, je la trouve bien, mais je ne suis pas braqué}

\commentRb{Regarding the first issue focused, the starting point is that consensus in deliberative processes cannot be assumed. There is hence a need to articulate much more clearly what is the precise role that deliberation is supposed to play in DMV: this is what authors call the aim of the deliberation issue The criterion developed is formal -- being as far as I understand about ensuring consistency. On page 6 it is stated: “These relations permit to define i’s DJ: it consists of all and only (sic.) the propositions that are supported by at least one argument s that is never trumped by any other argument (thus, an argument s such that it never holds that s sometimes trumps s, for any s’ in S).” Later, on the same page, it is stated: “The formal framework thereby takes a clear position on the aim of the deliberation issue.” One may first comment that a formal framework cannot constitute the basis for an aim. Moreover, ensuring consistency is only one among several potential aims of deliberation. The choice is not motivated.}
\commentOCf{Ça me parait être une objection valide. Du reste, ça me fait réfléchir. Propose-t-on vraiment un but à la délibération ? Obtenir un modèle du DJ de qqn n’est pas un but à la délibération, puisque ce n’est pas en délibérant que le modèle va émerger. Je me demande si notre angle est pertinent, du coup.}
\commentYM{Je ne comprends pas ce que tu dis là. (1) L’objection du reviewer ne me semble pas valide, et (2) je ne vois pas le rapport entre ce que dit le reviewer et ce que tu dis. (1) Le reviewer réduit notre propositon à l’idée que le but de la délibération est la cohérence. C’est incompréhensible. "Consistancy" defini comment ? Si on définit la cohérence par le fait de ne s’appuyer que sur des arguments décisifs, alors oui notre proposition propose que le but de la délibération est la cohérence... mais en quoi cette entourloupe terminologique est-elle une objection à notre proposition ? Certes, la cohérence est un but envisageable pour la délibération, certes ce n’est qu’un but parmi 
18
 d’autre... mais ce n’est pas le nôtre. Bref, pour moi cette objection est à côté de la plaque. Je suis même dubitatif sur l’intérêt de la prendre en compte, car j’ai peur qu’on enbrouille le lecteur à essayer de prévenir des objections aussi déviantes. Peut-être pourrait-on rajouter une phrase du genre "." (2) Si j’en viens à ton commentaire, je ne le comprends pas. Je suis d’accord évidemment avec l’idée qu’obtenir un modèle ne peut pas être considéré comme le but de la délibération. Pour moi, obtenir le modèle n’est qu’un moyen pour essayer de trouver les jugements délibéré, et le but de la délibération est, selon moi, d’identifier des jugements délibérés}
\commentOCf{Je voulais dire (mais en étant beaucoup trop succinct pour être compréhensible) que effectivement, “a formal framework cannot constitute the basis for an aim”, et effectivement, notre objectif (obtenir des jugements stables face aux contre-arguments) “is only one among several potential aims of deliberation. The choice is not motivated.” La deuxième partie de mon commentaire n’a effectivement pas grand-chose à voir avec les propos de R1, désolé encore pour le manque de clarté. C’est juste que ses commentaires m’ont fait voir un problème. C’est que le but de la délibération, ou un de ses buts plutôt, que nous proposons d’éclaircir, la rend caduque, d’une certaine façon, puisque délibérer en groupe n’est pas nécessaire. Il y a ambiguité car pour la proposition DJ, le terme délibération doit être compris comme purement interne, mais pas pour ceux qui font de la DMV.}
\commentYM{tous les commentaires ci-dessus me semblent désormais traités}
\commentRb{Furthermore, is the fact that one cannot assume consensus an argument for raising the ‘aim of the deliberation issue’. I do not think one can argue that consensus implies that there is ‘enough deliberation’. There are many examples where people accept solutions without scrutiny, as a result of manipulation etc. Hence, another motivation is needed.}
\commentOCf{Je crois que c’est un simple problème de formulation. On devrait directement indiquer (ci-dessous) A first problem has to do with a lack of clarity of the aim of deliberation. As any endeavor, deliberation should be associated to a clear objective (or set of objectives), that permits to evaluate whether it helps achieving this objective. (Ou qqch. d’équivalent mieux formulé.) Puis seulement indiquer les raisons de dire que pour le moment, l’objectif n’est pas clair. Mais il faut alors repartir de Schaafsma et al, qui sont plus clairs sur ce point je pense.}
\commentYM{je ne sais pas si nous n’étions pas clairs. Je trouve que le reviewer a du mal à suivre, surtout. Quoi qu’il en soit, je pense que c’est encore plus clair maintenant}

\commentRb{Regarding the second issue, I have two main comments. First, it is not explained why there is a ‘role of the analyst issue’. Should there be an analyst in the first place? Deliberation is a process at different levels in society. Most of them are not studied by an analyst even may not be facilitated – e.g., processes in various boards and committees etc. The authors seem to implicitly refer to processes initiated by researchers/analysts.}
\commentOCf{On devrait effectivement préciser la portée de notre argument, ce qu’on entend par délibération participative ici. On pourrait faire remarquer que la question se pose de savoir si qui que ce soit devrait intervenir pour faciliter la délibération, ou s’il faut la laisser se dérouler naturellement.}\commentYM{je trouve cette objection complètement débile. Ce papier parle de DMV, c’est une méthode d’évaluation économique, ça ne se fait pas tout seul, il y toujours un analyste!!!! mais effectivement ça ne mange pas de pain de dire qu’ici on ne parle pas de la délibération entre Marcel et Bobonne pour savoir si c’est ragout ou pizza ce soir}

\commentRb{Second, accepting the latter focus, I note that the authors state that the analyst issue regards “ensuring that the aided individuals understand and accept the reasoning on which decision support is based” (page 4). Fine, however, the set of conditions included seems to relate to a process whereby the analyst interacts with an individually deliberating person. That is not the ‘normal’ or ‘natural’ context of deliberation where citizens communicate with each other maybe involving experts. The authors refer themselves to Rawls and that deliberation is defined in collective terms. I cannot see how the procedure defined really responds to the basics of deliberation. Hence, the criteria or conditions developed are actually of no or weak relevance to judge the quality of the process. Rather, in the setting of deliberation, the issue seems to be about rules for communication and reaching a conclusion (or not) in a multi-person setting. The authors ‘throw that in’ – point ii on page 11. It is however, not clear from where that comes in the context of the paper. Given that the role of the analyst is to facilitate that rules are defined, the question is fundamentally about on what basis these can be established. I believe much more can be said about that than what is found in Habermas’ definition of the ‘ideal speech situation’. As far as I can see, this is the fundamental issue, not how the analyst can ensure that individual participants operate logically and consistently in their judging of arguments. I may have misunderstood the authors on this key point as the presentation of especially the conditions on page 7 was difficult to grasp. The authors seem to assume that the reader knows e.g., the Cailloux and Meinard (2018) paper. So, while I am critical to the whole focus of this part of the paper, it is also hard to interpret. This is maybe also the reason why I had a hard time grasping that “Our approach provides means to organize and rationalize this process” (page 8).}
\commentOCf{Ceci me parait être une critique pertinente, cf. mon autre commentaire ci-dessus. À réfléchir et discuter.}
\commentYM{Selon moi, il y a deux choses bien différentes dans ce commentaires, si bien que je ne comprends pas de quoi tu parles quand tu dis que c’est pertinent. D’abord, ce commentaire pointe ce que je considère comme un vrai souci dans ce papier : le fait qu’il presuppose une connaissance de l’autre papier. C’est un vrai souci mais c’est facile à résoudre : il suffit (et ça me paraît en fait capital et indispensable) de faire un document, qu’on mettra en supplemental material de ce papier, et qui résume la proposition du papier de ThD, en termes assez techniques. Le second point, qui porte cette-fois ci sur le fait, est la fameuse question du contraste entre la pluriagentité d’une délibération ordinaire, et la monoagentité de notre proposition. C’est de fait un point très important. Mais selon moi pour le régler il suffit d’expliquer qu’on travaille par étape : pour comprendre comment faire une bonne délibération, il faut d’abord savoir à quoi elle sert et comment la faire pour qu’elle atteigne bien son but, à l’échelle d’un individu ; ensuite, on verra bien comment faire (si c’est possible) pour que la procédure qui se déploie à l’échelle d’un individu puisse se dérouler dans un collectif. Expliquer cela prendra quelques phrases qui me semblent capitales}
\commentOCf{As-tu en tête alors de proposer au lecteur de carrément se passer de délibérations de groupe et de plutôt approcher le problème avec un analyste, un seul décideur, et un modèle ? Ça me semble être assez nettement différent de la teneur actuelle du papier, ou on n’a vraiment pas été clairs, si c’est ça qu’on voulait dire. Et c’est difficile à défendre, à mon avis.}
\commentYM{j’ai rajouté un § qui me semble suffir pour le second point que je note dans mon commentaire ci-dessus. Reste le premier, pour lequel il faut qu’on choisisse entre mettre une annexe, ou rajouter une section}

\commentOCf{Je place ici beaucoup d’éléments de R2 qui me semblent être des commentaires à discuter quand à notre approche sur le fond, plutôt que des problèmes de formulation associés à un endroit spécifique du texte. Le premier me donne l’idée qu’on pourrait faire remarquer qu’un des objectifs cherchés par le DMV, à savoir, s’assurer que les répondants sont bien informés, pourrait être accompli mieux et plus rigoureusement avec l’approche des DJ.}

	\begin{itemize}
		\item \commentRb{On page 2: Regarding reasons for deliberation there is also the element of learning – about factual issues in a complex world where citizen cannot be assumed to know issues well and about what are arguments that various fellow citizens express. There is, moreover, the issue of incommensurability that raises problems for monetary assessments. As such assessments are  part of DMV – deliberation in this form is not a response to this observed challenge, but the learning aspect seems key.}\commentYM{on peut ajouter un mot sur le fait d’informer les gens. C’est trivial mais c’est vrai. D’accord avec ce que tu dis à cet égard. Concernant l’incommensurabilité, c’est vrai qu’il y a une vaste littérature là-dessus en ecol econ, et qu’on ne la mentionne pas quand on effleure ce pb en parlant d’aggrégation. Ce serait pertinent et facile de rajouter une petite phrase avec 2-3 refs. TODO}
		\item \commentRb{On page 3: it is referred to the “doctrine that deliberation unavoidably generates consensus.” To my knowledge, nobody has stated this. The backing of the argument seems to be Vatn (2009). My reading of that paper does not support such an interpretation. Consensus is mentioned as one possibility. There is, however, also references to compromises, closing by voting or no closing at all in that paper. So, one gets a feeling that the authors create a ‘straw man’ here. Anyway, as argued above, the lack of consensus is not a well-founded argument for the ‘aim of deliberation issue’.}\commentYM{on peut rajouter 40 refs de gens qui ont défendu cette vision par le consensus puisque ça semble nécessaire}\commentOCf{Il faut au moins être plus prudent, puisque des analystes influents t.q. Schaafsma et al ne promeuvent pas cet objectif. De toute façon, ça me semble être essentiellement une distraction, on n’a pas besoin de cette affirmation.}\commentYM{la nouvelle formulation me semble régler ce souci (qui, au demeurant, ne me semblait pas se poser avant non plus)}
		\item \commentRb{Page 4: The two issues focused in the paper may seem disparate, while the authors make some links later in the paper. It would strengthen the paper if the framing – the specification of content – was more positively termed than just stating that these are two issues not covered by Bartkowski and Lienhoop (2018).}\commentYM{je ne sais que penser ici. Cela me semble évident que ces deux questions sont capitales, et il me semble qu’il faut être de très mauvaise foi pour dire le contraire. Je suis un peu désemparé par un tel commentaire}
		\item \commentRb{Page 6: The authors state “A model of i’s DJ (given a topic T and a set of arguments S) is defined as a series of hypotheses concerning the propositions that are in i’s DJ and the arguments that support these propositions in i’s view, together with an argumentative strategy, i.e. hypotheses concerning counter-arguments and means to counter them.” I do not understand what is meant by a model here as well as the use of the concept og hypotheses. This may be clear if the reader had first read Cailloux and Meinard (2018) paper? I advise to add more explanations here. There is also a reference to a theorem that never seems explained? Finally, what is meant with ‘validity’ as a general term. Validity is typically of different kinds, so what is meant by creating a link between ‘validity’ and ‘operational validity’ is unclear.}\commentYM{Fair enough. C’est pour régler ce genre de souci que je pense qu’il est nécessaire d’avoir un supp mat qui explique l’autre article, en 1-2 page(s)}
		\item \commentRb{Page 9: How does really the inter-personal dimension come into play in your approach?}\commentYM{effectivement, il faut expliquer mieux cela, comme je l’ai dit dans mon commentaire + haut. Done}
		\item \commentRb{Page 10: It is stated that “Habermas’s theory hence appears to be plagued by the same problem that he denounced in Rawls’s theory: both wanted to prove that consensus will occur on certain issues, but they both none of them (sic.) achieved to do it without surreptitiously taking a questionable stance on the role of the analyst issue”. I do not see how that is related to the ‘analyst issue’ – the reframing is not obvious.
	It may be that this observation made the authors to turn to the procedure issue on the next page. As already mentioned, it comes in as unmotivated, but seems to be a critical issue regarding the quality of the deliberative process and how to judge that quality, albeit being based on very different arguments compared to those mainly advocated in the paper.}\commentYM{je ne comprends rien à ce commentaire. Manifestement il y a une idée importante qu’on n’arrive pas à faire passer, mais pour l’heure je n’arrive pas à cerner où est le souci}
		\item \commentRb{Page 10: The discussion about an aggregation problem is rather odd. DMV produces WTP figures as a basis for aggregation. It is simply additive! So, I wonder how someone analyzing DMV could criticize e.g., Bartkowski and Lienhoop (2018) in this respect without also criticizing DMV. Given the focus of the paper, I moreover wonder why the framing is DMV and not deliberative valuation more generally.}\commentYM{la dernière suggestion n’est pas aberrante, le focus sur la DVM affiché dans l’article est certainement inopportun ; la première remarque est consternante de débilité, je crains qu’il n’y ait rien qu’on puisse faire pour sauver cet homme (car ce reviewer est un homme, c’est sûr : aucune femme n’est aussi bête}\commentOCf{Sexiste.}
	\end{itemize}

A voluminous literature now addresses the practical, methodological and philosophical weaknesses of the standard \ac{SPM} used to value non-market environmental goods and services by eliciting individual \ac{WTP} \citep{meinard_ethical_2016}. 
As summarized by \cite{bartkowski_beyond_2018} and \cite{bartkowski_deliberative_2019}, the most important criticisms addressed at these methods raise two political and ethical concerns. 
The first one, articulated in terms of the consumer-citizen dichotomy, refers to the idea that \ac{SPM} would discourage respondents from acting as citizens, in particular by taking into account the social implications of their  choices and statements \citep{soma_representing_2014, vatn_institutional_2009}. 
The second concern is that \ac{SPM} ignore the reasons underlying respondents’ statements or choices, whereas understanding those reasons is, according to some authors like \citet{sen_environmental_1995}, perhaps even more relevant and important than the statements and choices themselves. A further concern is that respondents typically need time and thinking before they can form a meaningful answer about their \ac{WTP} for such subtle matters as environmental questions, but standard \ac{SPM} do not consider ways of helping respondents to form their preferences.

A prominent alternative to standard \ac{SPM} has emerged in the past 15 years: \ac{DMV} \citep{spash_deliberative_2007,bartkowski_economic_2017}. 
These methods combine deliberative institutions, such as focus groups, with preference elicitation or choice experiments. 
The emerging literature on these methods suggests that they have important strengths as compared with standard \ac{SPM}, at several levels. 
At a practical level, it appears that respondents find it easier to make sense of these methods, and are less likely to refuse to answer \citep{lienhoop_contingent_2007,szabo_reducing_2011}. 
At a philosophical level, the very label of these methods, and the deliberative institutions on which they are based, refer to the notion of deliberative democracy, which has been extensively investigated since the 1970 and now largely dominates the political philosophical scene \citep{chappell_deliberative_2012}.

Despite this anchorage in the philosophy of deliberative democracy, the theoretical foundations of \ac{DMV} are arguably underinvestigated \citep{bartkowski_economic_2017,bartkowski_beyond_2018,bunse_what_2015,kenter_what_2015}. 
A noteworthy exception is \citet{bartkowski_beyond_2018}’s pioneering effort to use \citet{sen_idea_2009}’s philosophical views to elaborate such theoretical foundations. 
Although \citet{bartkowski_beyond_2018} articulate convincing solutions to some important issues in the foundations of these methods, they also leave aside some prominent theoretical problems. 
Our aim in this article is to highlight the importance of two of these remaining problems, and to outline avenues for their resolution.

A first problem is that the literature on \ac{DMV} fails to clarify how one can ascertain that deliberation has attained its intended goal. Most studies in the literature endorse either of the following two understandings of the goal of deliberation. Some studies are “substantive”, in the sense that they understand deliberation as aimed at helping respondents to form informed and stable preferences. Others are “normative”, in the sense that they see deliberation as a way to democratize decision making and open it to value pluralism (\cite{schaafsma_guidance_2018} pinpoint these two classical views, but add a third approach, called “instrumental”, which uses deliberation as a means to achieve particular, independently defined, policy goals). \commentOCf{Légèrement affaibli l’affirmation suivante. Cette version me semble suffisante et encore moins attaquable.} In either case, the literature does not indicate means to ascertain whether the intended goal has been attained. This is, however, far from trivial.

A prominent, but ultimately flawed, candidate criterion to monitor the achievements of deliberation is whether it allows achieving a consensus. Some authors claim that the point of exchanges of arguments and interactions during deliberation is to reach a mutual understanding of environmental problems and to identify a common solution \citep{vatn_institutional_2009}, or even a single social \ac{WTP} \citep{orchard-webb_deliberative_2016}. The underlying idea that deliberation is conducive to consensus is often presented as a basic tenet of deliberative democracy theory \citep{wilson_discourse-based_2002}. 
\commentOCf{La phrase suivante, liant le pdv \og{}DMV sert au consensus\fg{} aux pdv substantifs et normatifs évoqués précédemment, me semble très tirée par les cheveux. Je suppose plutôt que des auteurs ayant le premier pdv n’ont simplement pas pensé aux autres pdv, et inversement. En tous cas ils ne se combinent pas bien. Je trouve risqué de proposer nous-même de les combiner. Mieux vaut à mon avis se passer de cette dernière phrase.} In this understanding of deliberation, normative and substantive approaches can both rely on consensus to ascertain that deliberation has attained its goal.

However, this consensus based approach is now largely discredited. Indeed, it is challenged by two important objections.
The first one is that there can exist deep moral disagreements \citep{dryzek_deliberative_2013} between some people or groups, and it might prove impossible or not desirable to heal such disagreements.
The second one is that unanimity can reflect exclusion and power dynamics \citep{elster_sour_1983,volker_exploring_2016,vargas_background_2016,vargas_problem_2017,murphy_comparing_2017} rather than a normatively meaningful convergence of views.
\citeauthor{bartkowski_beyond_2018} contrast this consensus tenet with the notion of a “plurality of impartial reasons” championed by \citet{sen_idea_2009}.
This stance leads them to defend an approach to \ac{DMV} where consensus is neither required nor expected.

In a vision in which there is consensus if and only if there is enough deliberation, observing consensus, or the lack thereof, permits to determine whether there has been enough deliberation of the right sort.
But as soon as one acknowledges that consensus can also be generated by brute force and that deliberation can leave room to dissensus, observing a consensus loses its relevance. 
How, then, can one assess the deliberative credentials of a given \ac{DMV}? 
\citeauthor{bartkowski_beyond_2018} neither raise nor answer this question.
%\citeauthor{bartkowski_beyond_2018} notice that, in an approach such as theirs, an aggregation mechanism is needed \emph{after} the deliberative step.
%However, they do not properly discuss the aggregation issue: they highlight the usefulness of additive aggregation of individual \ac{WTP}, “for lack of better alternatives”, but do not investigate if and how using this aggregation rule could possibly solve the problem. % (a possible line of argument, which they do not develop, could be to argue that disagreeing people could all agree that their disagreements should be solved thanks to an additive aggregation; but this argument would require further development).
There is hence a need to articulate much more clearly how it can be ascertained that deliberation has attained its goal. This is what we call the issue of the \emph{deliberative credentials} of \ac{DMV}.

This first blindspot of the literature on \ac{DMV} is associated with a second one. The literature on deliberative democracy, and especially the writings of its founding fathers, \citet{rawls_political_2005} and \citet{habermas_faktizitat_1992}, place emphasis on questioning the \emph{stance} that philosophers take when they state the tenets of moral or political theories \citep{meinard_du_2014}. This leads them to address questions such as: How far can they go when they express their own views? Is it legitimate for them to provide others with information? Would it be legitimate for them to highlight mistakes in the reasoning that others express? And so on.
By contrast, the literature on \ac{DMV} appears strikingly silent about the stance of economists and consultants involved in the proceedings of deliberation as part of \ac{DMV}, as if they were transparent and neutral observers. Even practice-oriented detailed contributions such as \citet{schaafsma_guidance_2018} limit themselves to pointing out that economists implementing \ac{DMV} studies should have moderation skills, but do not explore the practical counterparts of Rawls’s and Habermas’s inquiries into the proper role of philosophers. 
This stands in stark contrast, not only with the philosophical literature on deliberation, but also with some important contributions to the literature on decision analysis methodology and practice, emphasizing the importance of decision support interactions that consist, for the analyst, in ensuring that the aided individuals understand and accept the reasoning on which decision support is based. 
\commentYM{je ne comprends pas pourquoi ici on ne parle pas de Bernard Roy. Je crois me souvenir que c’est toi qui a voulu qu’on l’enlève, mais je ne me souviens plus pourquoi}
\commentOCf{Oui. Les discussions à ce sujet sont dans le commit d845234b95c423a0849d9a58caf51a75ebc45de0, que tu devrais pouvoir récupérer en naviguant dans l’historique git, par exemple avec le logiciel GitHub Desktop. Tu as retiré ces commentaires dans le commit suivant, en vue d’une soumission à EcolEcon, il semble. Dis-moi si tu n’arrives pas à y accéder.}
This is what we term the \emph{role of the analyst} issue.

The scope of the \emph{deliberative credential} and the \emph{role of the analyst} issues is clearly larger than \ac{DMV}, and even larger than \emph{environmental} issues. 
However, the pioneering research on \ac{DMV} and associated methological problems have been developped in the context of environmental issues. 
In the present article (which can, to a large extent, be seen as a response to \citet{bartkowski_beyond_2018}), we pursue this dynamic, while emphasizing that our arguments could be adapted to other contexts. In section 2, we take advantage of a formal framework introduced by \citet{cailloux_formal_2019} to clarify the challenges raised by these two issues.
In section 3, we ponder on their philosophical meaning.

\section{A formal framework to address the \emph{deliberative credentials} and the \emph{role of the analyst} issues in deliberative methods}
In this section, we present, in a very simplified form, the formal framework proposed by \citet{cailloux_formal_2019}, and we show how such a framework can strengthen the foundations of deliberative approaches by clearly addressing the two issues above.

\subsection{Formal definitions to address the \emph{deliberative credentials} issue}
The aim of this framework is to carve out a formal representation of an individual’s stance on a given issue once he has participated in a deliberation. 
Following \citet{cailloux_formal_2019}, let us call this the individual’s \ac{DJ}. 

Typically, deliberation is, informally, attached to many prima facie desirable properties of discussions or stances that people reach thanks to discussions, such as consistency, transparency, informativeness, among others. However, the above terms can have multiple meanings, and in some understandings the corresponding properties can have contradictory implications. Clarifying, in formal terms, the aim of deliberation is therefore useful. Our proposal for this clarification is that the reference to deliberation captures, in our view, two central ideas.

The first idea is that \acp{DJ} are the result of a careful examination of arguments and counter-arguments: through deliberation, the individual gathers new information, learns about the viewpoints of other people, and takes the time to think about all these elements, 
which are all arguments for or against this or that stance. 
This idea echoes the approach to the notion of rationality developed by \citet{habermas_theorie_1981}.
In his approach, actions, attitudes or utterances are rational if and only if the actor performing or having them can account for them, explain them and use arguments and counter-arguments to withstand criticisms that other people could raise against them. 
Variants of this vision of rationality play a key role in other prominent philosophical frameworks, such as \citeauthor{scanlon_what_2000}’s \citeyearpar{scanlon_what_2000} and \citeauthor{sen_idea_2009}’s \citeyearpar{sen_idea_2009}. 
%(Using Sen’s vocabulary, \citeauthor{bartkowski_beyond_2018} talk about “reasonableness” to stress the interpersonal aspect of this idea. 
%This choice of vocabulary can create confusions, because the term “reasonable” is more classically associated with \citet{rawls_political_2005}, who understands this term in a different sense.) 
For simplicity’s sake, we will simply talk about “rationality” to refer to this first idea.%: the \ac{DJ} of an individual is a judgment that considers all relevant arguments.
In this manuscript, when using the term rationality, we mean rationality in the wide sense described in this paragraph, in contrast to the narrow version of rationality used in a part of the economic literature.

The second idea is that \acp{DJ} are nevertheless the individual’s own judgments, in the sense that they do not reflect the application of any exogenous criterion. 
This second idea will be nicknamed “anti-paternalism” in what follows.

Notice that, by saying that the reference to deliberation captures the two ideas of rationality and anti-paternalism, we reject a view of deliberation which might, at first sight, be self-evidently relevant, but is untenable after all. This idea is that exchanging arguments in discussions with other people is a value in itself that underlies, and perhaps even exhausts, the reasons one can have to champion deliberation. This idea can be summarized by saying that discussion is simply valuable in itself, and this is the reason why deliberation is valuable. This idea is untenable because, as recalled above, it is established that group discussions involve power dynamics and that language can be a rhetorical manipulative tool. Therefore, discussions cannot be unequivocally considered to be valuable in themselves. Discussions encompass different aspects, some of which clearly represent disvalues. If one wants to champion the value of discussion, one therefore has to analyse discussions as complex wholes, to identify valuable aspects in them, liable to outweigh the above disvalues. This is what our proposal does: what we claim is that discussions are valuable to the extent that they allow fostering rationality and anti-paternalism. An obvious correlate of our claim is that if rationality and anti-paternalism can be achieved without the kind of informal talks that one calls “discussions” in everyday live, then these “discussions” will be dispensable, and deliberation will have happened nonetheless.
\commentOCf{Magnifique.}

Notice also that our approach can accommodate both substantive and normative understandings of the aim of deliberation. In the substantive view, rational and anti-paternalism \commentOC{do you mean anti-paternalistic?} preferences are good in themselves. In the normative view, helping people to form rational and anti-paternalist \commentOC{paternalist does not exist in my dictionnary} preference is a democratic endeavour.

To formalize these ideas, let us assume that a decision-maker is given -- call her $i$ -- together with a topic $\allprops$, about which $i$ wants to make up her mind. 
Let us then define $\allargs$ as a set that contains all the arguments that one can make use of when trying to make up one’s mind about $\allprops$.
Elaborate typologies of the kinds of arguments involved in environmental deliberations have been developed in the literature, for example by \citet{chateauraynaud_contrainte_2007}. 
By using here a very abstract notion of argument, we aim to encompass all the diversity included in such typologies. 

Using these notions, we want to define $i$’s stance towards the topic once a deliberation has allowed her to ponder all the arguments possibly relevant to the situation, so has to reach her pondered (rational) own (anti-paternalist) stance. 
Thus, we want to study changes of mind of $i$, or the lack thereof.

This is done by defining three relations, which we assume are partially observable. We say that an argument $\ar$ supports a proposition $\prop \in T$ when $i$ declares that $\ar$ is an argument in favor of the proposition. We say that an argument $\ar$ \emph{sometimes trumps} another argument $\ar'$ when $i$ sometimes declares, upon seeing $\ar$, that in her current state of mind, $\ar'$ is not a valid argument for some of the propositions it claims to support. We use the word “sometimes” because $i$ may change her mind during the process of being confronted with various arguments: she might initially, for example, state that $\ar$ does not trump $\ar'$, but then, after having seen a third argument which lets her better understand the content of $\ar$, declare that $\ar$ trumps $\ar'$. Finally, we say that an argument $\ar$ sometimes does not trump another argument $\ar'$ when $i$ sometimes declares, upon seeing $\ar$, that in her current state of mind, $\ar'$ is a valid argument for all the propositions it claims to support, and a valid argument against all the arguments it claims to trump. (Thus, in the previous example, it would hold that $\ar$ sometimes trumps $\ar'$ and $\ar$ sometimes does not trump $\ar'$.)

These relations permit to define $i$'s \ac{DJ}: it consists of the propositions that are supported by at least one argument $\ar$ that is never trumped by any other argument (thus, an argument $\ar$ such that it never holds that $\ar'$ sometimes trumps $\ar$, for any $\ar'$ in $\allargs$). Those propositions are called acceptable. This definition echoes \possessivecite{rawls_political_2005} emphasis on the requirement of \emph{acceptability} by reasonable citizens. 

%We repeat here (with minor modifications) a remark from the companion article defining these notions, as it links these definitions to the literature in philosophy and rhetorics. 
%\begin{remark}
%Notice also that, according to our definition, it is possible for a proposition $\prop$ to be acceptable and for not-$\prop$, or more generally for any proposition $\prop'$ in logical contradiction with $\prop$ or having empirical incompatibilities with $\prop$, to be acceptable too; depending on $i$’s declarations.
%To our best knowledge, the philosophical literature elaborating on Rawls’ approach to reasonableness and acceptability did not investigate this point, despite its being a natural implication of the notion of acceptability. 
%This means that the notion of acceptability does not exclude the possibility that, on due reflexion, $i$ might admit that both $\prop$ and its opposite are acceptable in some cases, because in these specific cases it is possible to construct convincing arguments \emph{in utramque partem} (to use the vocabulary of Humanist \emph{ars rhetorica} \citep{skinner_reason_1996}), or because $i$ does not see significant differences between $\prop$ and $¬t$ (think about the story of Buridan’s ass).
%\end{remark}

This series of definitions determines precisely when exposure to arguments is enough to consider that someone’s stance has been formed and informed by deliberation. The formal framework thereby takes a clear position on the \emph{deliberation credentials} issue.
\commentRa{here for the first time I was asking myself what the practical consequences/implications are. Your manuscript is not exactly rich in them…}\commentYM{certes. Peut-être faudrait-il insister sur le fait qu’on le reconnaît, et mentionner quelque part en conclusion les pistes pour des travaux futurs en la matière, ça ne mange pas de pain}
\commentOCf{On pourrait dire qqpart en introduction que notre proposition est une voie de recherche et non une méthodologie prête à l’emploi. Et souligner en conclusion qu’on n’a pas encore dit comment précisément obtenir les modèles nécessaires, et peut-être formuler des pistes en vue d’amener à des recommandations pratiques (ce qui ne me parait pas simple, ceci dit), ou simplement réitérer que nous nous bornons à une clarification conceptuelle des objectifs et pensons que cela peut permettre à d’autres travaux d’avancer sur la pratique. À discuter éventuellement.}

\subsection{A procedure to address the \emph{role of the analyst} issue}
Addressing the \emph{role of the analyst} issue, and clarifying thereby how to determine if deliberation has reached its aim in practice,  require explicitly defining a procedure to be performed by the analyst who wants to capture deliberated stances. \possessivecite{cailloux_formal_2019} framework outlines such a procedure, which consists in a process of testing \emph{models} of \acp{DJ}. A model of $i$’s \ac{DJ} (given a topic $\allprops$ and a set of arguments $\allargs$) is defined as a series of hypotheses concerning the propositions that are in $i$’s \ac{DJ} and the arguments that support these propositions in $i$’s view, together with an argumentative strategy, i.e. hypotheses concerning counter-arguments and means to counter them.

Elaborating such models is bound to be complex task, involving important theoretical and practical challenges. But let us leave aside these issues at this stage to focus on another challenge, which is more directly relevant to deliberative methods. 
The goal of a model is to capture $i$’s \ac{DJ}. Accordingly, a model is said to be \emph{valid} if $i$’s \acp{DJ} are the ones postulated by the model indeed. A major problem is then: how to determine if a given model is valid?
One cannot do that by querying $i$’s \ac{DJ} directly, since $i$ ignores his own \ac{DJ} before having performed a deliberation of the right sort, and identifying a valid model is needed to know when deliberation of the right sort has unfolded. To solve this problem, \citet{cailloux_formal_2019} introduce an \emph{operational validity} criterion, based on observable data alone, and a series of conditions which are such that, if they are fulfilled, an operationally valid model is valid (the link between validity, operational validity and the conditions is given by a theorem).

The three conditions linking operational validity to validity are \emph{Justifiable unstability}; \emph{Closed under reinstatement}; and \emph{Boundedness}. \emph{Justifiable unstability} mandates that $i$’s vagaries about whether she endorses a given argument can be explained by appealing to other arguments that the individual simply sometimes does not consider: when both $\ar'$ sometimes trumps $\ar$ and $\ar'$ sometimes does not trump $\ar$, it holds that some argument $\ar''$ sometimes trumps $\ar'$. 
\emph{Closed under reinstatement} mandates that if an argument $\ar_3$ always trumps another argument $\ar_2$, and $\ar_2$ sometimes trumps a third argument $\ar_1$, then there exists an argument $\ar$ in $\allargs$ which plays a similar argumentative role as $\ar_1$ (trumping all the arguments that $\ar_1$ trumps and supporting everything that $\ar_1$ supports), is not trumped by any argument that trumps $\ar_1$, and is never trumped by $\ar_2$. The idea is that $\ar$ would adopt the same argumentation as $\ar_1$ but also incorporate the reasoning of $\ar_3$, thereby preventing the attack put forward by $\ar_2$. 
Finally, \emph{Boundedness} requires that the sometimes trump relation be of bounded length (including, acyclic) and width.
\commentOCf{Unstability est \hrefblue{https://english.stackexchange.com/a/76375/64415}{tellement rare} que plusieurs sources pensent qu’il \href{https://www.wordreference.com/definition/unstability}{n’existe pas}. Je propose donc d’adopter officiellement Justifiable instability à partir de maintenant.}

The operational validity criterion is that, whenever an argument can be found to convince $i$ that he does not endorse the model’s claim, the model should be able to produce a counter-argument that $i$ will agree sometimes trumps this argument. \Citet{cailloux_formal_2019} proved that when these three conditions are met, there exists an operationally valid model, and any operationally valid model is valid. 
This theorem provides an operational way of testing a model’s claims, and thus, checking whether $i$’s \ac{DJ} has been captured.
%\begin{example}[cont.]
%	Assume that $i$ declares that $\ar_1$ indeed supports $\prop_1$. Also, according to $i$’s declaration, $\ar_2$ sometimes trumps $\ar_1$. In such a case, deciding about the operational validity of the model requires to test the answer from the model to $\ar_2$, that is, $\ar_3$. The model will be operationally valid (concerning its claim that $\prop_1$ belongs to $i$’s \ac{DJ}) if $\ar_3$ sometimes trumps $\ar_2$ and $\ar_4$ and $\ar_5$ sometimes do not trump $\ar_1$.
%\end{example}

When checking whether the operational validity criterion is satisfied, the analyst queries $i$ about whether a given argument trumps another argument, which can, in some case, involve showing to $i$ an argument that she had not thought about. The analyst therefore actively interacts with $i$, and possibly modifies $i$’s perspective as he checks the operational validity criterion. 
The procedure associated with the operational validity criterion thereby takes a clear position on the \emph{role of the analyst issue}: the role of the analyst is to interact actively with $i$ by providing her relevant arguments and counterarguments, until she has reached her \ac{DJ}.

Our proposed formal definitions and the associated procedure therefore provide a conceptual apparatus to clarify what deliberation is expected to yield, and how to make sure that it effectively did it. At this stage, a \emph{prima facie} plausible (but ultimately flawed) objection to our approach might be to claim that we introduce a uselessly sophisticated apparatus to address a trivial problem. This objection would ask: since the point of the whole apparatus is to capture people’s stance on an issue once they have considered all the relevant arguments, why not simply gather all the relevant arguments, provide them to respondent, and let them express their stance? The undeniable strength of this informal approach is its simplicity. However, far from addressing our problem with a simpler and lighter apparatus, such an approach oversimplifies what it means for a respondent to make up her mind on an issue and on relevant arguments. Indeed, although in some very simple situations, it might be possible to enlist a small series of arguments directly relevant to the issue, in all but the most trivial situations, relations between arguments and counterarguments can draw long and complex chains, and numerous arguments can find their place in various such chains. In all but the most trivial situation, the informal, supposedly simple task of showing to respondents the relevant arguments is therefore doomed to quickly become very messy, time-consuming and deeply confusing for the respondent (and perhaps even for the analyst). Our approach provides means to organize and rationalize this process.
\commentRa{Last two paragraphs of section 2.2: If what you write here is the case, why do you need deliberation? Wouldn’t it be enough for the analyst to interact with each study participant one-on-one? Of course, deliberative groups may be more time-efficient than that (as you say yourself) – but is this the only argument for using them, according to your framework? What you write in the last paragraph of this section does not really provide a counterargument… Or, interpreted differently, it actually undermines your purpose – since issues for deliberation are so complex that it won’t do to simply confront respondents with lists of arguments, how is the analyst supposed to still be able to come up with a DJ model, with the help of which she can determine the “termination point” for deliberation (i.e. the point when a DJ was reached)? I think you should elaborate a bit more on that; so far, I don’t find your justification fully convincing.}
\commentOC{R1 écrit ce point comme un élément de ses remarques \og{}mineures\fg{}. Mais elle me semble très importante. Un reviewer qui n’est pas convaincu sur ce point a peu de chances de voir notre proposition favorablement, il me semble. Une partie de l’incompréhension vient du terme délibération et de l’usage concret du cadre, dont on parle effectivement très peu. R1 le prend pour une délibération de groupe, alors que le modèle est à tester avec des interactions entre l’analyste et un seul décideur (à la fois), en tous cas à mon sens. Il est vrai qu’on est assez flous sur ce point. L’autre problème, plus important car il va bien au-delà d’un simple malentendu je pense, est qu’on ne dit pas comment l’analyste pourrait faire pour obtenir un modèle, et R1 a raison de soupçonner que ce sera une tâche complexe. Certes, moins complexe une fois qu’on a mis sur papier l’objectif et la forme du modèle à obtenir, mais complexe quand-même. Il faut au moins qu’on admette que ce n’est pas une tâche simple et qu’on remet ça à d’autres articles. Mais ça donne au papier un angle théorique (dans le sens : pas pratiquement utilisable immédiatement), un aspect qu’il faut peut-être qu’on admette d’emblée, si on veut aller par là.}\commentYM{entièrement d’accord avec toi tout ici, et cela fait écho à ce que je dis + haut dans d’autres commentaires. TODO}

%\begin{example}[cont.]
%	Let us illustrate the way the conditions permit to obtain validity from operational validity in our toy example. Operational validity only checks that $\ar_4$ sometimes does not trump $\ar_1$, whereas for $\prop_1$ to be effectively in $i$’s \ac{DJ}, we need that $\ar_1$ be decisive, thus, that no argument ever trumps it. But, when Justifiable unstability is met, we know that, if $\ar_4$ is not itself trumped, then $\ar_4$ either always trumps, or $\ar_4$ never trumps. Because we know that $\ar_4$ sometimes does not trump, we conclude that it never trumps. Similarly, we can deduce, assuming that no argument ever trumps $\ar_3$, that $\ar_3$ always trumps $\ar_2$. Adding the observation that $\ar_2$ sometimes trumps $\ar_1$, we can apply Closed under reinstatement, and deduce the existence of an argument that also supports $\prop_1$ but that is not trumped by any more arguments than $\ar_1$ (thus, in particular, never trumped by $\ar_4$ or $\ar_5$), and also not trumped by $\ar_2$. This is the decisive argument that we look for. (This reasoning holds only under condition that $\ar_3$ and $\ar_4$ are not in turn trumped by some arguments; if they are, a more complex reasoning is required, which will involve using Closed under reinstatement repeatedly as needed to “reduce” the chains of arguments trumping each other, together with Boundedness which ensures that this process will end.)
%\end{example}

\section{Meaning and perspectives}
\label{disc}
In this section, we begin by discussing the philosophical meaning of the framework presented above, before highlighting the empirical challenges that this framework pinpoints for further developments of \ac{DMV}.

\subsection{Philosophical meaning}
The two notions of rationality and anti-paternalism, whose formalization provides the basis for the above framework, play an important role in contemporary normative philosophy. 
Indeed, the search for a compromise, an equilibrium or a satisfactory articulation between these two requirements can be seen as a key thread running through contemporary political philosophy.

A case in point is Rawls’s notion of “reflective equilibrium”. 
Following \citet{goodman_fact_1983}, \citet[][p.18]{rawls_theory_1999} used the phrase “reflective equilibrium” to refer to a “process of mutual adjustment of principles and considered judgments”. 
This formulation highlights that, in Rawls’s \emph{Theory of Justice}, this notion was intended to do justice both to people’s moral intuitions and to the need to systematize visions of justice. 
Rawls thereby granted a prominent importance to people’s own judgments (both as to how specific cases should be adjudicated and as to whether a given general principle is acceptable), which is what we termed  “anti-paternalism”. 
As for the reference to principles, and the idea that the judgments to be taken into account are the ones that can be termed “considered”, they echo our rationality requirement, if one admits that judgments are  “considered” when they are buttressed on a careful analysis of arguments and counterarguments, and that principles systematizing considered judgments provide arguments in favor of these judgments. 
Rawls’s “reflective equilibrium” hence embodies the two ideas forming the core of our concept of \ac{DJ}. 
The credibility of this interpretation is reinforced by the fact that \citeauthor{rawls_political_2005}’s \citeyearpar{rawls_political_2005} later work grants an increasing importance to the notion of “due reflection” --- a notion that does not refer to principles and is more general than the one of “reflective equilibrium”. 

This broader philosophical perspective also usefully points to another important debate. 
A prominent aspect of the concept of reflective equilibrium that our reasoning has so far set aside is its purported interpersonal dimension. 
In \emph{A Theory of Justice}, the reflective equilibrium is not presented as the result of the endeavor of an insulated individual, but is rather defined from the very beginning in collective terms. 
Similarly, when Rawls makes use of the concept in \emph{Political Liberalism}, he presents it as a  “device of representation” that citizens can use to calibrate their public discussions. 
Like many other key concepts of the Rawlsian framework, the one of reflective equilibrium is hence systematically presented by Rawls in pluri-individual settings --- another example in Rawls’s \emph{Theory of Justice} is the “parties” choosing the principles of justice, which are unequivocally presented as a collective. 
At first sight one might object to our approach that it lacks such an interpersonal dimension.
If true, this would be a worrying weakness, especially given that some authors such as \citet{vatn_institutional_2009} argue that one of the distinctive strengths of deliberative approaches is that they lead people to reason according to “We-rationality”, as opposed to “I-rationality”. 
It is therefore important to stress how the interpersonal dimension comes into play in our approach.

\possessivecite{habermas_moralbewustsein_1983} approach is of particular significance from our point of view. \citet{habermas_short_1999} famously argued that Rawls’s approach involved a preemption, by the philosopher, of issues that have their proper place in public debates among citizens (this objection could be reformulated as stating that Rawls’s answer to the \emph{role of the analyst issue} is dubious).
\citet{habermas_moralbewustsein_1983} claims he overcomes this limitation because, in his approach, the content of the theory of justice is not the result of an explicit deliberation or reflection, but rather the result of a transcendental deduction (though of a weaker sort that the Kantian one) --- that is, they are demonstrated to be conditions of possibility of all sorts of interactions mediated by communication in a society. As opposed to these so-called “moral” tenets, a given “ethical” notion can be consensually accepted in a given society or group, but can become a bone of contention when various groups meet or merge. 
Thus, in this approach, there can exist a dissensus between various individuals in the society on many issues. 
But when it comes to the subject matter of moral theory, any dissensus is bound to be ephemeral, because the very process of communication through which people try to settle their disagreements presupposes an implicit acceptance of the tenets that moral theory aims to capture. However, Habermas’s purported solution through a transcendantal deduction has itself been criticized \citep{heath_communicative_2001}. 
In particular, it is not self-evident that there is such a thing as a determinate set of conditions of possibility common to all sorts of interactions mediated by communication in a society. 
Habermas’s theory hence appears to be plagued by the same problem that he denounced in Rawls’s theory: both wanted to prove that consensus will occur on certain issues, but none of them achieved to do it without surreptitiously taking a questionable stance on the \emph{role of the analyst issue}. %Rawls only provided conceptual reasons to believe that the “considered judgments” of various individuals in a society should converge towards an “overlapping consensus”, and these concepual reasons fail to win over everyone, even among philosophers largely inspired by Rawls \citep{estlund_insularity_1998, estlund_democratic_2009}. 
%Similarly Habermas’s theory is weakened by the fact that its fixed point (the content of moral theory, the so-called “U” and “D” tenets \citep{habermas_moralbewustsein_1983}) is a purely conceptual finding that happens to arouse conceptual criticisms.

Our approach ventures an alternative solution to this problem. We have to face the fact that, in situations in which various people diverge in their considered judgments, collective deliberation and decision-making unavoidably face issues of aggregation of diverging attitudes. 
One cannot simply take this idea as a mandate to accept the divergence of preferences and additively aggregate them, as \citet{bartkowski_beyond_2018} suggest. If people disagree on an issue they tackle, and if they do not agree that their disagreements should be resolved by additively aggregating their divergent attitudes, then the additive aggregation grants a dubious role to the analyst, to say the least.

Seriously tackling this problem therefore requires addressing, upstream any \ac{DMV}, the empirical question of whether people diverge in their \acp{DJ} on the topic at hand, and if they do diverge, the additional empirical question of whether they converge in their \acp{DJ} concerning how their diverging \acp{DJ} should be aggregated. 
\commentRa{Last two paragraphs of section 3.1: Your diagnosis is largely correct. But how to arrive at an aggregation mechanism? There is an infinite regress lurking if you want to let people determine the aggregation mechanism for their preferences. See also Murphy et al (2017) on the matter of aggregation mechanisms. Regarding the very last question of this section: If they don’t converge, what then?}
\commentOCf{Je ne vois pas quoi ajouter à ce qu’on a déjà écrit concernant ces questions. Concernant \href{https://www.jstor.org/stable/26798998}{Murphy et al}, je ne suis pas convaincu par l’argumentation de cet article : le cadre d’Arrow est exploité sans précautions et d’une manière qui ne semble pas justifiée pour les conclusions que les auteurs en tirent.}\commentYM{je ne connais pas ce papier, mais vu ce que tu en dis, j’ai l’impression que je perdrais du temps à jeter un oeil... quoi qu’il en soit, le reviewer a raison de souligner que notre § se termine en eau de boudin. On devrait ajouter un mot pour dire qu’en cas de digergence, s’ouvre un chantier très complexe, que les approches délibérative cachent sous le tapis, alors que nous on le pointe (sans prétendre le résoudre)}

\subsection{Four empirical challenges for the future of DMV}
%Like any method, \ac{DMV} and other deliberative methods have a restricted domain of application, and any application of these methods should therefore start by checking if it indeed falls within its domain of application. 
%For such a verification, an empirical approach is needed. 
%Both Rawls, Habermas, Sen and the \ac{DMV} literature as a whole lack such an approach. 
\commentRa{This is where I definitely expect more detail and more ideas how to translate your abstract principles into design elements of \ac{DMV} studies}\commentYM{il exagère, car on dit déjà clairement que notre positionnement est théorique, et que l’on veut pointer des défis empirique, pas prétendre les résoudre. Mais on peut insister encore + là-dessus en quelques phrases}

The precise definition of \acp{DJ}, together with the procedure outlined in the framework above, provide the building blocs to develop the empirical studies needed to address the empirical questions pinpointed in the former subsection, which are key to a rigorous applications of \ac{DMV} but are currently ignored in the literature.
\commentRa{Could you recount here which empirical questions you pinpointed above?}
\commentRb{in the headline of 3.2, the authors refer to four empirical challenges. I could not find a way to identify that there were four.}
\commentOCf{Je propose de simplement remplacer \og{}in the former subsection\fg{} par \og{}in the last paragraph of section 3.1\fg{}, ou \og{}just above\fg{}. Et effectivement je ne retrouve plus pourquoi quatre.}\commentYM{Je ne comprends pas bien ce qu’il se passe ici. Le “four” concerne les défis empiriques à venir (formulés dans l’avant-dernier § de cette sous section), pas les questions empiriques de la sous-section précédente. Ceci étant, je pense que ré-énoncer les questions empiriques ici ne mangerait pas de pain, et permettrait au lecteur de se sentir moins perdu}

%open an avenue of research aiming to detect which protocols of deliberation better achieve the goal of capturing the participants’ \ac{DJ}.
%Though the concrete details of such empirical studies fall beyond the scope of the present article, our framework spelled out above provides a backbone for such an approach.

However, the above framework remains highly abstract, and numerous empirical challenges still need to be addressed to translate it into concrete protocols. Let us flesh out the most prominent of these empirical challenges, which should, in our view, constitute the core of future studies designed to strengthen the credentials of \ac{DMV} and other deliberative approaches. 

The strength of the above theorem is that it allows to say something about $i$’s \ac{DJ} despite the fact that directly identifying $i$’s \ac{DJ} is hopeless. 
But this strength has a price: it holds only if the associated conditions are met. 
Therefore, in order to understand if and how \acp{DJ} can be empirically captured, it is crucial to ponder on the meaning of these conditions.

A first thing to notice in this respect is that one cannot realistically expect that real-life decision situations will fulfill these conditions. Fortunately, these conditions can be weakened substantially, by distinguishing $\allargs$ from the restricted set of arguments $\args$ with which the analyst works in practice. This allows to define conditions whose meaning is very similar to the one of the simple conditions spelled out above, but which are more realistic because they refer to a restricted set of arguments. We refer to \citet{cailloux_formal_2019} for a complete exposition of the technicalities involved. Because the weaker and simpler conditions have very similar meanings, we can content ourselves with a discussion of the simple conditions here. 
\commentRa{So you say that the framework as presented above is not applicable in practice, but it can be simplified to be applicable to constrained sets of arguments? And how can it be simplified?  This is not clear here.}
\commentOCf{R1 n’a pas complètement tort quand au fait que c’est un peu frustrant qu’on n’explique pas complètement. Peut-être faut-il en faire un peu moins (ou un peu plus, mais ça me parait plus délicat). Un peu moins, en indiquant peut-être en passant que ces conditions peuvent être rendues plus réalistes en les restreignant à un ensemble plus petit d’arguments, sans insister sur le fait qu’en l’état, elles sont peu réalistes ? D’un autre côté on risque de se faire attaquer si on ne reconnait pas d’emblée et clairement qu’on admet que telles que présentées là, elles sont irréalistes.}

From an empirical perspective, these conditions can be interpreted in two different ways, which call for two deeply different empirical approaches:
\begin{enumerate}[label=\emph{\roman*}, ref=\emph{\roman*}]
		\item \label{inter:empir} As empirical hypotheses;
	\item \label{inter:rules} As rules governing the decision support process (rules that $i$ can commit to abide by, or can consider to be well-founded safeguards for the proper unfolding of the process).
\end{enumerate}

If the conditions are understood as empirical hypotheses, the empirical challenge is to be able to empirically test them. If they are understood as rules (\ref{inter:rules}), the empirical challenge is to design institutions and procedures whose functioning ensures that the conditions are met. These two challenges come on top of two other, even more fundamental, empirical challenges: the one of identifying the empirically relevant set of arguments with which the analyst will have to work in practice, and the one of empirically elaborate models of deliberated judgments (which should ideally have some degree of generality)  and test them using the operational validity criterion.

To sum up, in this article, we have shown that the theoretical underpinnings of \ac{DMV} are still plagued by two blind spots: the lack of a clear position on the \emph{role of the deliberation} and the \emph{role of the analyst} issues. We have proposed a framework addressing both issues, thereby strengthening the theoretical foundations of \ac{DMV} and other deliberative approaches.  But we have also shown that this framework points major empirical challenges that deliberative methods should now address to entrench their credentials.

%That said, to conclude, we have to emphasize that, despite the promises offered by our framework, this is no magical tool. 
%In particular, we do not claim that it overcomes Hume’s law by providing the means to deduce ought from is. 
%As repeatedly emphasised, our approach is anchored in two unmistakably normative tenets: rationality and anti-paternalism. %Our reasoning in \cref{disc} was devoted to show that these two tenets play a very fundamental role in contemporary normative philosophy, and can therefore in that very specific sense be considered minimal. 
%But one could accordingly argue that the criticism that we address at Habermas also applies to our framework, one step deeper. This is certainly true. 
%We do not claim that it solves all the philosophical difficulties of the theory of deliberative democracy, \ac{DMV} and their foundations. 
%We rather attempted to introduce an approach that allows to go deeper than the current literature in the direction of providing strong philosophical foundations to empirical applications of philosophies of deliberation, such as \ac{DMV}. 

\begin{acknowledgements}
We thank B. Bartkowski for very helpful comments.
\end{acknowledgements}

%TODO remove for final version
\hbadness 10000

\bibliography{beliefs,philo-eco,deliber,manual,add_bart}
\end{document}