\RequirePackage{amsmath}
\RequirePackage[l2tabu, orthodox]{nag}
%\documentclass[smallextended,nospthms,natbib]{svjour3}
\documentclass[version=3.21, pagesize, twoside=off, bibliography=totoc, DIV=calc, fontsize=12pt, a4paper, french, english]{scrartcl}
\newcommand{\institute}[1]{}
\newcommand{\keywords}[1]{}
\newenvironment{acknowledgements}{
	\section*{Acknowledgements}
}{
}
%\makeatletter \let\cl@chapter\relax \makeatother
%\smartqed  % flush right qed marks, e.g. at end of proof 
\input{preamble/packages}
\input{preamble/redac}
\input{preamble/math_basics}
%Decision Theory (MCDA and SC)
\newcommand{\allalts}{\mathscr{A}}
\newcommand{\allcrits}{\mathscr{C}}
\newcommand{\alts}{A}
\newcommand{\dm}{i}
\newcommand{\allF}{\mathscr{F}}
\newcommand{\allvoters}{\mathscr{N}}
\newcommand{\voters}{N}
\newcommand{\allprofs}{\boldsymbol{\mathcal{R}}}
\newcommand{\prof}{\boldsymbol{R}}
\newcommand{\linors}{\mathscr{L}(\allalts)}

%Deliberated Judgment
\newcommand{\allargs}{S^*}
\newcommand{\args}{S}
\newcommand{\ar}{s}
\newcommand{\ileadsto}{⇝}
\newcommand{\ibeatse}{⊳_\exists}
\newcommand{\nibeatse}{⋫_\exists}
\newcommand{\ibeatsst}{⊳_\forall}
\newcommand{\nibeatsst}{⋫_\forall}
\newcommand{\mleadsto}[1][\eta]{⇝_{#1}}
\newcommand{\mbeats}[1][\eta]{⊳_{#1}}
\newcommand{\ibeatseinv}{⊳_\exists^{-1}}

%Logic
\newcommand{\ltru}{\texttt{T}}
\newcommand{\lfal}{\texttt{F}}


%\definecolor{airforceblue}{rgb}{0.36, 0.54, 0.66}
%\definecolor{ao}{rgb}{0.0, 0.0, 1.0}
%\definecolor{ao(english)}{rgb}{0.0, 0.5, 0.0}

%\journalname{}

%I find these settings useful in draft mode. Should be removed for final versions.
	%Which line breaks are chosen: accept worse lines, therefore reducing risk of overfull lines. Default = 200.
		\tolerance=2000
	%Accept overfull hbox up to...
		\hfuzz=2cm
	%Reduces verbosity about the bad line breaks.
		\hbadness 5000
	%Reduces verbosity about the underful vboxes.
		\vbadness=1300

\begin{document}
\title{Deliberation and environmental decision}
\author{Yves Meinard \and Olivier Cailloux}
\institute{
	Yves Meinard
	\and
	Olivier Cailloux 
	\at 
	Université Paris-Dauphine, \\
	PSL Research University, \\
	CNRS, \\
	LAMSADE\\
	75016 PARIS, FRANCE\\
	\email{olivier.cailloux@dauphine.fr}
}
\makeatletter
	\hypersetup{
		pdfsubject={preference},
		pdfkeywords={decision aid, justification, empirical validation, methodology}
	}
\makeatother
\maketitle

\keywords{decision aid, justification, empirical validation, methodology} 

\begin{abstract}
A voluminous literature addresses the weaknesses of standard stated preference methods used to value non-market environmental goods and services, such as contingent valuation and choice experiment. 
Deliberative monetary valuation (DMV) has emerged as a prominent alternative to the standard versions of these methods. It combines deliberative institutions with preference elicitation. 
Despite an anchorage in an extensive philosophical literature on deliberative democracy, the theoretical foundations of DMV are underinvestigated.  
A noteworthy exception is \citet{bartkowski_beyond_2018}'s effort to use \citeauthor{sen_idea_2009}'s philosophical views to elaborate such theoretical foundations. 
The present article pursues this theoretical effort by pointing out two issues left unanswered by the above contribution: 
the first issue is the precise role that deliberation is expected to play in DMV and, more broadly, in environmental decision-making (we term this the \emph{aim of the deliberation} issue); 
the second issue is the role that economists and consultants involved in the proceedings of deliberation are supposed to play (we term this the \emph{role of the analyst} issue). 
In order to clarify the kind of investigations that DMV or any alternative method should implement to be unequivocal on these issues, we use a formal framework introduced by \citet{cailloux_formal_2018}, designed to capture the stance that an individual has on a given topic once she has participated in a deliberation: her ``deliberated judgment''. 
This framework allows to identify empirical questions that DMV do not tackle whereas answering these questions would be necessary to clarify the stance that DMV takes on the \emph{aim of the deliberation} issue. 
When it comes to the \emph{role of the analyst} issue, our framework advocates an active role of the practitioner in creating what we call models of \aclp{DJ}. 
This framework helps to characterize the normative stance adopted when implementing a deliberative approach.
\end{abstract}
\acresetall

\section{Introduction}

A voluminous literature now addresses the practical, methodological and philosophical weaknesses of the standard stated preference methods (SPM) used to value non-market environmental goods and services by eliciting individual willingness to pay (WTP) \citep{meinard_ethical_2016} . 
As summarized by \citet{bartkowski_beyond_2018,bartkowski_deliberative_2019}, the most important criticisms addressed at these methods raise two political and ethical concerns. 
The first one, articulated in terms of the consumer-citizen dichotomy, refers to the idea that SPM would discourage respondents from acting as citizens, in particular by taking into account the social implications of their  choices and statements \citep{soma_representing_2014, vatn_institutional_2009}. 
The second concern is that SPM ignore the reasons underlying respondents' statements or choices, whereas understanding those reasons is, according to some authors like \citet{sen_environmental_1995}, perhaps even more relevant and important than the statements and choices themselves. A further concern is that respondents typically need time and thinking before they can form a meaningful answer about their WTP for such subtle matters as environmental questions, but standard SPM do not consider ways of helping respondents to form their preferences.

A prominent alternative to standard SPM has emerged in the past 15 years: deliberative monetary valuation (DMV) \citep{spash_deliberative_2007,bartkowski_economic_2017}. 
These methods combine deliberative institutions, such as focus groups, with preference elicitation or choice experiments. 
The emerging literature on these methods suggests that they have important strengths as compared with standard SPM, at several levels. 
At a practical level, it appears that respondents find it easier to make sense of these methods, and are less likely to refuse to answer \citep{lienhoop_contingent_2007,szabo_reducing_2011}. 
At a philosophical level, the very label of these methods, and the deliberative institutions on which they are based, refer to the notion of deliberative democracy, which has been extensively investigated since the 1970 and now largely dominates the political philosophical scene \citep{chappell_deliberative_2012}.

Despite this anchorage in the philosophy of deliberative democracy, the theoretical foundations of DMV are arguably underinvestigated \citep{bartkowski_economic_2017,bartkowski_beyond_2018,bunse_what_2015,kenter_what_2015}. 
A noteworthy exception is \citet{bartkowski_beyond_2018}'s pioneering effort to use \citet{sen_idea_2009}'s philosophical views to elaborate such theoretical foundations. 
Although \citet{bartkowski_beyond_2018} articulate convincing solutions to some important issues in the foundations of these methods, they also leave aside some prominent theoretical problems. 
Our aim in this article is to highlight the importance of two of these remaining problems, and to outline avenues for their resolution.

A first problem has to do with the fact that authors on DMV vary in the answer they give to a very basic question: should deliberation lead to a consensus? Some authors (e.g. \citet{vatn_institutional_2009}) claim that the point of exchanges of arguments and interactions during deliberation is to reach a mutual understanding of environmental problems and to identify a common solution, or even a single social WTP \citep{orchard-webb_deliberative_2016}. The underlying idea that deliberation is conducive to consensus is often presented as a basic tenet of deliberative democracy theory \citep{wilson_discourse-based_2002}. 
However, this idea is challenged by two important objections.
The first one is that there can exist deep moral disagreements \citep{dryzek_deliberative_2013} between some people or groups, and it might prove neither possible nor desirable to heal such disagreements.
The second one is that unanimity can reflect exclusion and power dynamics \citep{elster_sour_1983,volker_exploring_2016,vargas_background_2016,vargas_problem_2017,murphy_comparing_2017} rather than a normatively meaningful convergence of views.
\citeauthor{bartkowski_beyond_2018} contrast this consensus tenet with the notion of a ``plurality of impartial reasons'' championed by \citet{sen_idea_2009}.
This stance leads them to defend an approach to DMV where consensus is neither required nor expected. Such an approach might seem more open-minded than the doctrine that deliberation unavoidably generates consensus. However, it evades a crucial question: when can one consider that there has been \emph{enough} deliberation? 
In a vision in which there is consensus if and only if there is enough deliberation, observing consensus, or the lack thereof, permits to determine whether there has been enough deliberation of the right sort.
As soon as one acknowledges that consensus can also be generated by brute force and that deliberation can leave room to dissensus, observing a consensus loses its relevance. 
How, then, can one assess the deliberative credentials of a given DMV? 
\citeauthor{bartkowski_beyond_2018} neither raise nor answer this question.
%\citeauthor{bartkowski_beyond_2018} notice that, in an approach such as theirs, an aggregation mechanism is needed \emph{after} the deliberative step.
%However, they do not properly discuss the aggregation issue: they highlight the usefulness of additive aggregation of individual WTP, ``for lack of better alternatives'', but do not investigate if and how using this aggregation rule could possibly solve the problem. % (a possible line of argument, which they do not develop, could be to argue that disagreeing people could all agree that their disagreements should be solved thanks to an additive aggregation; but this argument would require further development).
There is hence a need to articulate much more clearly what is the precise role that deliberation is supposed to play in DMV: this is what we call the \emph{aim of the deliberation} issue.

This first blindspot of the literature on DMV is associated with a second one. The literature on deliberative democracy, and especially the writings of its founding fathers \citet{rawls_political_2005} and \citet{habermas_faktizitat_1992}, place emphasis on questioning the \emph{stance} that philosophers take when they state the tenets of moral or political theories \citep{meinard_du_2014}. 
By contrast, the literature on DMV appears strikingly silent about the stance of economists and consultants involved in the proceedings of deliberation as part of DMV, as if they were transparent and neutral observers (even in practice oriented detailed contributions such as \citet{schaafsma_guidance_2018}). 
This stands in stark contrast, not only with the philosophical literature on deliberation, but also with some important contributions to the literature on decision analysis methodology and practice, emphasizing the importance of decision support interactions that consist, for the analyst, in ensuring that the aided individuals understand and accept the reasoning on which decision support is based.
This is what we term the \emph{role of the analyst} issue.

The scope of the \emph{aim of the deliberation} and the \emph{role of the analyst} issues is clearly larger than DMV, and even larger than \emph{environmental} issues. 
However, the pioneering research on DMV and associated methological problems have been developped in the context of environmental issues. 
In the present article (which can, to a large extent, be seen as a response to \citet{bartkowski_beyond_2018}), we pursue this dynamic, while emphasizing that our arguments could be adapted to other contexts. In section 2, we take advantage of a formal framework introduced by \citet{cailloux_formal_2018} to clarify the challenges raised by these two issues.
In section 3, we ponder on their philosophical meaning.

\section{A formal framework to address the \emph{aim of the deliberation} and the \emph{role of the analyst} issues in deliberative methods}
In this section, we present, in a very simplified form, the formal framework proposed by \citet{cailloux_formal_2018}, and we show how such a framework can strengthen the foundations of deliberative approaches by clearly addressing the two issues above.

\subsection{Formal definitions to address the \emph{role of the deliberation issue}}
The aim of this framework is to carve out a formal representation of an individual's stance on a given issue once he has participated in a deliberation. 
Following \citet{cailloux_formal_2018}, let us call this the individual's \ac{DJ}. 
The reference to deliberation captures, in our view, two central ideas.

The first idea is that \acp{DJ} are the result of a careful examination of arguments and counter-arguments: through deliberation, the individual gathers new information, learns about the viewpoints of other people, and takes the time to think about all these elements, 
which are all arguments for or against this or that stance. 
This idea echoes the approach to the notion of rationality developed by \citet{habermas_theorie_1981}. 
In his approach, actions, attitudes or utterances are rational if and only if the actor performing or having them can account for them, explain them and use arguments and counter-arguments to withstand criticisms that other people could raise against them. 
Variants of this vision of rationality play a key role in other prominent philosophical frameworks, such as \citeauthor{scanlon_what_2000}’s \citeyearpar{scanlon_what_2000} and \citeauthor{sen_idea_2009}’s \citeyearpar{sen_idea_2009}. 
%(Using Sen's vocabulary, \citeauthor{bartkowski_beyond_2018} talk about ``reasonableness'' to stress the interpersonal aspect of this idea. 
%This choice of vocabulary can create confusions, because the term ``reasonable'' is more classically associated with \citet{rawls_political_2005}, who understands this term in a different sense.) 
For simplicity's sake, we will simply talk about ``rationality'' to refer to this first idea.%: the \ac{DJ} of an individual is a judgment that considers all relevant arguments.

The second idea is that \acp{DJ} are nevertheless the individual's own judgments, in the sense that they do not reflect the application of any exogenous criterion. 
This second idea will be nicknamed ``anti-paternalism'' in what follows. 



Let us assume that a decision-maker is given -- call her $i$ -- together with a topic $\allprops$, about which $i$ wants to make up her mind. 
Let us then define $\allargs$ as a set that contains all the arguments that one can make use of when trying to make up one’s mind about $\allprops$.
Elaborate typologies of the kinds of arguments involved in environmental deliberations have been developed in the literature, for example by \citet{chateauraynaud_contrainte_2007}. 
By using here a very abstract notion of argument, we aim to encompass all the diversity included in such typologies. 

Using these notions, we want to define $i$’s stance towards the topic once a deliberation has allowed her to ponder all the arguments possibly relevant to the situation. 
Thus, we want to study changes of mind of $i$ (or the lack thereof).

This is done by defining three relations, which we assume are partially observable. We say that an argument $\ar$ supports a proposition $\prop \in T$ when $i$ declares that $\ar$ is an argument in favor of the proposition. We say that an argument $\ar$ \emph{sometimes trumps} another argument $\ar'$ when $i$ sometimes declares, upon seeing $\ar$, that in her current state of mind, $\ar'$ is not a valid argument for some of the propositions it claims to support or some of the propositions it claims to trump. We use the word ``sometimes'' because $i$ may change her mind during the process of being confronted with various arguments: she might initially, for example, state that $\ar$ does not trump $\ar'$, but then, after having seen a third argument which lets her better understand the content of $\ar$, declare that $\ar$ trumps $\ar'$. Finally, we say that an argument $\ar$ sometimes does not trump another argument $\ar'$ when $i$ sometimes declares, upon seeing $\ar$, that in her current state of mind, $\ar'$ is a valid argument for all the propositions it claims to support, and a valid argument against all the propositions it claims to trump. (Thus, in the previous example, it would hold that $\ar$ sometimes trumps $\ar'$ and $\ar$ sometimes does not trump $\ar'$.)

These relations permit to define $i$'s \ac{DJ}: it consists of all and only the propositions that are supported by at least one argument $\ar$ that is never trumped by any other argument (thus, an argument $\ar$ such that it never holds that $\ar'$ sometimes trumps $\ar$, for any $\ar'$ in $\allargs$). Those propositions are called acceptable. This definition echoes \possessivecite{rawls_political_2005} emphasis on the requirement of \emph{acceptability} by reasonable citizens. 

%We repeat here (with minor modifications) a remark from the companion article defining these notions, as it links these definitions to the literature in philosophy and rhetorics. 
%\begin{remark}
%Notice also that, according to our definition, it is possible for a proposition $\prop$ to be acceptable and for not-$\prop$, or more generally for any proposition $\prop'$ in logical contradiction with $\prop$ or having empirical incompatibilities with $\prop$, to be acceptable too; depending on $i$’s declarations.
%To our best knowledge, the philosophical literature elaborating on Rawls' approach to reasonableness and acceptability did not investigate this point, despite its being a natural implication of the notion of acceptability. 
%This means that the notion of acceptability does not exclude the possibility that, on due reflexion, $i$ might admit that both $\prop$ and its opposite are acceptable in some cases, because in these specific cases it is possible to construct convincing arguments \emph{in utramque partem} (to use the vocabulary of Humanist \emph{ars rhetorica} \citep{skinner_reason_1996}), or because $i$ does not see significant differences between $\prop$ and $¬t$ (think about the story of Buridan’s ass).
%\end{remark}

This series of definitions determines precisely when exposure to arguments is enough to consider that someone’s stance has been formed and informed by deliberation. The formal framework thereby takes a clear position on the \emph{aim of the deliberation issue}.

\subsection{A procedure to address the \emph{role of the analyst} issue}
\commentOCf{Le cadre du DJ laisse apparaitre deux besoins à combler : qqn doit soumettre les arguments et tester ainsi les modèles ; et qqn doit concevoir des modèles. L’analyste, dans mon esprit, est plutôt concepteur. Il me semble réducteur de le faire uniquement testeur de modèles : je vois le test comme une procédure systématique ne requérant pas beaucoup d’expertise ni d’intelligence.}\commentYMf{je ne sais pas : je ne suis pas sûr que le test soit si trivial que ça. Quoi qu'il en soit, dans le cadre du présent article, ça ne me semble pas être un enjeu majeur}\commentOCf{Je ne sais pas si l’enjeu est majeur ou mineur, mais je serais ennuyé qu’on suggère que le rôle de l’analyste se limite à tester les modèles, sans mentionner qu’il peut aussi servir à concevoir les dits modèles. Penses-tu qu’on pourrait par exemple ajouter une phrase à la fin de l’avant-dernier paragraphe : \og{}(This could also be considered as being part of the role of the analyst.)\fg{} Alternativement, on pourrait restaurer la phrase qui est maintenant commentée, à la fin de ce paragraphe : \og{}We leave aside at this stage practical questions concerning how an analyst might elaborate such a model.\fg{}}
Addressing the \emph{role of the analyst} issue, and clarifying thereby how to determine if deliberation has reached its aim in practice,  require explicitly defining a procedure to be performed by the analyst who wants to capture deliberated stances. \citet{cailloux_formal_2018}'s framework outlines such a procedure, which consists in a process of testing \emph{models} of \acp{DJ}. A model of $i$'s \ac{DJ} (given a topic $\allprops$ and a set of arguments $\allargs$) is defined as a series of hypotheses concerning the propositions that are in $i$'s \ac{DJ} and the arguments that support these propositions in $i$'s view, together with an argumentative strategy, i.e. hypotheses concerning counter-arguments and means to counter them. %We leave aside at this stage practical questions concerning how an analyst might elaborate such a model.

%\begin{example}
%	\commentOCf{J’aimerais changer l’éternel exemple de la pluviosité du lendemain ; j’aimerais aussi qu’il soit relativement sans enjeux (pour qu’il ne soit pas honteux d’affirmer qu’on peut résumer le débat en quatre arguments) ; et j’aimerais qu’il ait au moins un vague rapport avec l’environnement. Je n’ai pas réussi à satisfaire toutes ces contraintes.}
%	Consider $T=\set{\prop_1, \prop_2}$, where $\prop_1$ is the proposition “I should eat all the cake” and $\prop_2$ is “I should share half the cake with my flatmate”. The set of arguments $\allargs = \set{\ar_1, \ar_2, \ar_3, \ar_4, \ar_5}$, representing arguments in favor ($\ar_1, \ar_3, \ar_5$) and against ($\ar_2, \ar_4$) eating all the cake. The model claims that $i$’s \ac{DJ} contains only $\prop_1$. It uses the argument $\ar_1$ to support $\prop$. If attacked with argument $\ar_2$, the model’s argumentation strategy consists in playing the argument $\ar_3$ against $\ar_2$. The argumentation strategy does not play a response to $\ar_4$.
%\end{example}

The goal of a model is to capture $i$'s \ac{DJ}. Accordingly, a model is said to be \emph{valid} if $i$’s \acp{DJ} are the ones postulated by the model indeed. A major problem is then: how to determine if a given model is valid?
One cannot do that by querying $i$'s \acp{DJ} directly, since $i$ ignores his own \acp{DJ} before having performed a deliberation of the right sort, and identifying a valid model is needed to know when deliberation of the right sort has unfolded. To solve this problem, \citet{cailloux_formal_2018} introduce an \emph{operational validity} criterion, based on observable data alone, and a series of conditions which are such that, if they are fulfilled, an operationally valid model is valid (the link between validity, operational validity and the conditions is given by a theorem).

The three conditions linking operational validity to validity are \emph{Justifiable unstability}; \emph{Closed under reinstatement}; and \emph{Boundedness}. \emph{Justifiable unstability} mandates that $i$'s vagaries about whether she endorses a given argument can be explained by appealing to other arguments that the individual simply sometimes does not consider: when both $\ar'$ sometimes trumps $\ar$ and $\ar'$ sometimes does not trump $\ar$, it holds that some argument $\ar''$ sometimes trumps $\ar'$. \emph{Closed under reinstatement} mandates that if an argument $\ar_3$ always trumps another argument $\ar_2$, itself sometimes trumping a third argument $\ar_1$, then there exists an argument $\ar$ in $\allargs$ which plays a similar argumentative role as $\ar_1$ (trumping all the arguments that $\ar_1$ trumps and supporting everything that $\ar_1$ supports), is not trumped by any argument that trumps $\ar_1$, and is never trumped by $\ar_2$. The idea is that $\ar$ would adopt the same argumentation than $\ar_1$ but also incorporate the reasoning of $\ar_3$, thereby preventing the attack put forward by $\ar_2$. Finally, \emph{Boundedness} requires that the sometimes trump relation be of bounded length (including, acyclic) and width.

The operational validity criterion is that, whenever an argument can be found to convince $i$ that the he does not endorse the model's claim, the model should be able to produce a counter-argument that $i$ will agree sometimes trumps this argument. When these three conditions are met, there exists an operationally valid model, and any operationally valid model is valid \citep{cailloux_formal_2018}. This theorem provides an operational way of testing a model’s claims, and thus, checking whether $i$’s \ac{DJ} has been captured.
%\begin{example}[cont.]
%	Assume that $i$ declares that $\ar_1$ indeed supports $\prop_1$. Also, according to $i$’s declaration, $\ar_2$ sometimes trumps $\ar_1$. In such a case, deciding about the operational validity of the model requires to test the answer from the model to $\ar_2$, that is, $\ar_3$. The model will be operationally valid (concerning its claim that $\prop_1$ belongs to $i$’s \ac{DJ}) if $\ar_3$ sometimes trumps $\ar_2$ and $\ar_4$ and $\ar_5$ sometimes do not trump $\ar_1$.
%\end{example}

When checking whether the operational validity criterion is satisfied, the analyst queries $i$ about whether a given argument trumps another argument, which can, in some case, involve showing to $i$ an argument that she had not thought about. The analyst therefore actively interacts with $i$, and possibly modifies $i$’s perspective as he checks the operational validity criterion. 
The procedure associated with the operational validity criterion thereby takes a clear position on the \emph{role of the analyst issue}: the role of the analyst is to interact actively with $i$ by providing her relevant arguments and counterarguments, until she has reached her \ac{DJ}.




%\begin{example}[cont.]
%	Let us illustrate the way the conditions permit to obtain validity from operational validity in our toy example. Operational validity only checks that $\ar_4$ sometimes does not trump $\ar_1$, whereas for $\prop_1$ to be effectively in $i$’s \ac{DJ}, we need that $\ar_1$ be decisive, thus, that no argument ever trumps it. But, when Justifiable unstability is met, we know that, if $\ar_4$ is not itself trumped, then $\ar_4$ either always trumps, or $\ar_4$ never trumps. Because we know that $\ar_4$ sometimes does not trump, we conclude that it never trumps. Similarly, we can deduce, assuming that no argument ever trumps $\ar_3$, that $\ar_3$ always trumps $\ar_2$. Adding the observation that $\ar_2$ sometimes trumps $\ar_1$, we can apply Closed under reinstatement, and deduce the existence of an argument that also supports $\prop_1$ but that is not trumped by any more arguments than $\ar_1$ (thus, in particular, never trumped by $\ar_4$ or $\ar_5$), and also not trumped by $\ar_2$. This is the decisive argument that we look for. (This reasoning holds only under condition that $\ar_3$ and $\ar_4$ are not in turn trumped by some arguments; if they are, a more complex reasoning is required, which will involve using Closed under reinstatement repeatedly as needed to “reduce” the chains of arguments trumping each other, together with Boundedness which ensures that this process will end.)
%\end{example}

\section{Meaning and perspectives}
\label{disc}
\commentOCf{On ne répond pas à ce que je comprends comme la question principale de Bart. Il me semble penser qu’on pourrait se contenter de faire discuter les gens à bâtons rompus jusqu’à ce que la stabilité puisse être raisonnablement supposée (par exemple en demandant aux participants eux-mêmes de déclarer lorsqu’ils pensent être arrivés au bout de la discussion). La complexité de notre cadre formelle ne serait donc pas justifiée. Cette objection me semble a priori raisonnable, et passera j’imagine par la tête d’autres lecteurs. Dans mon esprit, on devrait indiquer (ne fût-ce que brièvement) que l’objectif principal est de construire et valider des modèles généraux. Sinon (si on ne fait que laisser discuter les gens, ou si on ne fait que concevoir des modèles ad-hoc pour une situation donnée), on n’établit pas de connaissance réutilisable. Ceci rejoint mon commentaire précédent concernant le rôle de l’analyste. Qu’en penses-tu ?}\commentYMf{Je ne me rappelle pas de cela dans les commentaires de Bart, mais c'est loin. Quoi qu'il en soit, je vois deux choses dans ce que tu dis. (1) Il y a d'abord l'idée que notre modèle formel se prend la tête pour rien, car il n'y a qu'à laisser les gens discuter. Je ne pense pas qu'un lecteur sérieux puisse s'arrêter à cela, car ce serait faire fi de tout ce qu'on dit dans l'intro sur les problèmes largement connus et débattus relatifs aux limites des approches délibératives et aux conditions qui limitent leur applicabilité. (2) Ensuite, il y a la question de la production d'un savoir transférable. Pour moi, la transférabilité du savoir n'est en aucun cas nécessaire pour pouvoir prétendre avoir produit une connaissance scientifique et valide. Une très grande majorité des connaissances produites dans la science que je fréquente le plus (la biologie de la conservation) partirait à la benne des sciences si on imposait cette exigence. Les modèles généraux ne m'intéressent pas follement, et la perspective d'en produire ne me semble en aucun cas une condition nécessaire à la production d'un savoir valable. J'ai cependant ajouté une parenthèse qui va dans le sens de ce que tu évoques, dans l'avant dernier paragraphe.}
\commentOCf{Remarque de Bartkowski, page 9 de la version qu’on lui a envoyée: “Does this not imply that the analyst should be in possession of all relevant arguments ex ante? What would one need deliberation for in such a situation? One could simply present all the arguments to respondents and ask them to form a judgement on their basis (in fact, this is what conventional stated preference studies do by presenting information to the respondents before asking for their preferences...).” Ceci correspond à ton interprétation (1), comme je comprends. Ce n’est pas sa remarque principale, contrairement à mon souvenir, puisque ce n’est qu’une remarque parmi de nombreuses, mais ça me semble être une remarque importante, puisqu’elle semble remettre fondamentalement en question l’utilité du cadre formel présenté. Si on y répond dans l’intro, en tous cas, ça semble avoir échappé à notre relecteur. Et je ne vois pas clairement qu’on y réponde : l’intro justifie uniquement que l’approche classique laisse deux questions importantes ouvertes. Mais pas qu’on a besoin de notre cadre formel pour capturer le DJ, plutôt que la méthode de débat en pagaille dont Bartkowski affirme qu’elle est déjà généralement utilisée et convient très bien. Concernant le point (2), ta remarque est certes valide en général : le savoir ne doit pas nécessairement être transférable. Mais c’est encore mieux s’il l’est, toutes choses égales par ailleurs. En l’occurrence, nous proposons (comme je le conçois) une méthodo qu’on prétend réutilisable dans diverses circonstances pour capturer le DJ, et pas simplement un moyen de savoir quel est le DJ dans telle circonstance hyper précise sans prétention de pouvoir appliquer la même méthode à d’autres circonstances. Ceci dit, maintenant que tu m’y fais penser, c’est vrai que ma réponse n’est pas terrible (car dans le fond, la méthode \og{}on jette les arguments en pagaille\fg{} est tout aussi réutilisable et générale). À la réflexion, l’objection de Bartkowski n’est pas si simple à écarter, il me semble. (C-à-d que je suis convaincu que l’obtention d’un modèle comme on propose est préférable au débat en pagaille, mais je ne vois pas immédiatement de moyen simple d’en convaincre le lecteur, à ce stade de notre développement.)}

In this section, we begin by discussing the philosophical meaning of the framework presented above, before highlighting the empirical challenges that this framework pinpoints for further developments of DMV.

\subsection{Philosophical meaning}
The two notions of rationality and anti-paternalism, whose formalization provides the basis for the above framework, play an important role in contemporary normative philosophy. 
Indeed, the search for a compromise, an equilibrium or a satisfactory articulation between these two requirements can be seen as a key thread running through contemporary political philosophy.

A case in point is Rawls’s notion of “reflective equilibrium”. 
Following \citet{goodman_fact_1983}, \citet[][p.18]{rawls_theory_1999} used the phrase “reflective equilibrium” to refer to a “process of mutual adjustment of principles and considered judgments”. 
This formulation highlights that, in Rawls's \emph{Theory of Justice}, this notion was intended to do justice both to people's moral intuitions and to the need to systematize visions of justice. 
Rawls thereby granted a prominent importance to people's own judgments (both as to how specific cases should be adjudicated and as to whether a given general principle is acceptable), which is what we termed  “anti-paternalism”. 
As for the reference to principles, and the idea that the judgments to be taken into account are the ones that can be termed “considered”, they echo our rationality requirement, if one admits that judgments are  “considered” when they are buttressed on a careful analysis of arguments and counterarguments, and that principles systematizing considered judgments provide arguments in favor of these judgments. 
Rawls's “reflective equilibrium” hence embodies the two ideas forming the core of our concept of \ac{DJ}. 
The credibility of this interpretation is reinforced by the fact that \citeauthor{rawls_political_2005}’s \citeyearpar{rawls_political_2005} later work grants an increasing importance to the notion of “due reflection” --- a notion that does not refer to principles and is more general than the one of “reflective equilibrium”. 

This broader philosophical perspective also usefully points another important debate. 
A prominent aspect of the concept of reflective equilibrium that our reasoning has so far set aside is its purported interpersonal dimension. 
In \emph{A Theory of Justice}, the reflective equilibrium is not presented as the result of the endeavor of an insulated individual, but is rather defined from the very beginning in collective terms. 
Similarly, when Rawls makes use of the concept in \emph{Political Liberalism}, he presents it as a  “device of representation” that citizens can use to calibrate their public discussions. 
Like many other key concepts of the rawlsian framework, the one of reflective equilibrium is hence systematically presented by Rawls in pluri-individual settings --- another example in Rawl's \emph{Theory of Justice} is the “parties” choosing the principles of justice, which are unequivocally presented as a collective. 
At first sight one might object to our approach that it lacks such an interpersonal dimension.
If true, this would be a worrying weakness, especially given that some authors such as \citet{vatn_institutional_2009} argue that one of the distinctive strengths of deliberative approaches is that they lead people to reason according to ``We-rationality'', as opposed to ``I-rationality''. 
It is therefore important to stress how the interpersonal dimension comes into play in our approach.

\citeauthor{habermas_moralbewustsein_1983}'s
\citeyearpar{habermas_moralbewustsein_1983} approach is of particular significance from our point of view. \citet{habermas_short_1999} famously argued that Rawls's approach involved a preemption, by the philosopher, of issues that have their proper place in public debates among citizens (this objection could be reformulated as stating that Rawls's answer to the \emph{role of the analyst issue} is dubious).
\citet{habermas_moralbewustsein_1983} claims he overcomes this limitation because, in his approach, the content of the theory of justice is not the result of an explicit deliberation or reflection, but rather the result of a transcendental deduction (though of a weaker sort that the Kantian one) --- that is, they are demonstrated to be conditions of possibility of all sorts of interactions mediated by communication in a society. As opposed to these so-called “moral” tenets, a given “ethical” notion can be consensually accepted in a given society or group, but can become a bone of contention when various groups meet or merge. 
Thus, in this approach, there can exist a dissensus between various individuals in the society on many issues. 
But when it comes to the subject matter of moral theory, any dissensus is bound to be ephemeral, because the very process of communication through which people try to settle their disagreements presupposes an implicit acceptance of the tenets that moral theory aims to capture. However, Habermas's purported solution through a transcendantal deduction has itself been criticized \citep{heath_communicative_2001}. 
In particular, it is not self-evident that there is such a thing as a determinate set of conditions of possibility common to all sorts of interactions mediated by communication in a society. 
Habermas's theory hence appears to be plagued by the same problem that he denonced in Rawls's theory: both wanted to prove that consensus will occur on certain issues, but they both none of them achieved to do it without surreptitiously taking a questionable stance on the \emph{role of the analyst issue}. %Rawls only provided conceptual reasons to believe that the “considered judgments” of various individuals in a society should converge towards an “overlapping consensus”, and these concepual reasons fail to win over everyone, even among philosophers largely inspired by Rawls \citep{estlund_insularity_1998, estlund_democratic_2009}. 
%Similarly Habermas's theory is weakened by the fact that its fixed point (the content of moral theory, the so-called “U” and “D” tenets \citep{habermas_moralbewustsein_1983}) is a purely conceptual finding that happens to arouse conceptual criticisms.

Our approach ventures an alternative solution to this problem. We have to face the fact that, in situations in which various people diverge in their considered judgments, collective deliberation and decision-making unavoidably face issues of aggregation of diverging attitudes. 
One cannot simply take this idea as a mandate to accept the divergence of preferences and additively aggregate them, as \citet{bartkowski_beyond_2018} suggest. If people disagree on an issue they tackle, and if they do not agree that their disagreements should be resolved by additively aggregating their divergent attitudes, then the additive aggregation grants a dubious role to the analyst, to say the least.

Seriously tackling this problem therefore requires addressing, upstream any DMV, the empirical question of whether people diverge in their \acp{DJ} on the topic at hand, and if they do diverge, the additional empirical question of whether they converge in their \acp{DJ} concerning how their diverging \acp{DJ} should be aggregated. 

\subsection{Four empirical challenges for the future of DMV}
%Like any method, DMV and other deliberative methods have a restricted domain of application, and any application of these methods should therefore start by checking if it indeed falls within its domain of application. 
%For such a verification, an empirical approach is needed. 
%Both Rawls, Habermas, Sen and the DMV literature as a whole lack such an approach. 
The precise definition of \acp{DJ}, together with the procedure outlined in the framework above, provide the building blocs to develop the empirical studies needed to address the empirical questions pinpointed in the former subsection, which are key to a rigorous applications of DMV but are currently ignored in the literature.

%open an avenue of research aiming to detect which protocols of deliberation better achieve the goal of capturing the participants’ \ac{DJ}.
%Though the concrete details of such empirical studies fall beyond the scope of the present article, our framework spelled out above provides a backbone for such an approach.

However, the above framework remains highly abstract, and numerous empirical challenges still need being addressed to translate it into concrete protocols. Let us flesh out the most prominent of these empirical challenges, which should, in our view, constitute the core of future studies designed to strengthen the credentials of DMV and other deliberative approaches. 

The strength of the above theorem is that it allows to say something about $i$'s \ac{DJ} despite the fact that directly identifying $i$'s \ac{DJ} is hopeless. 
But this strength has a price: it holds only if the associated conditions are met. 
Therefore, in order to understand if and how \acp{DJ} can be empirically captured, it is crucial to ponder on the meaning of these conditions.

A first thing to notice in this respect is that one cannot realistically expect that real-life decision situations will fulfill these conditions. Fortunately, these conditions can be weakened substantially, by distinguishing $\allargs$ from the restricted set of arguments $\args$ with which the analyst works in practice. This allows to define conditions whose meaning is very similar to the one of the simple conditions spelled out above, but which are more realistic because they refer to a restricted set of arguments. We refer to \citet{cailloux_formal_2018} for a complete exposition of the technicalities involved. Because the weaker and simpler conditions have very similar meanings, we can content ourselves with a discussion of the simple conditions here. 

From an empirical perspectives, these conditions can be interpreted in two different ways, which call for two deeply different empirical approaches:
\begin{enumerate}[label=\emph{\roman*}, ref=\emph{\roman*}]
		\item \label{inter:empir} As empirical hypotheses;
	\item \label{inter:rules} As rules governing the decision support process (rules that $i$ can commit to abide by, or can consider to be well-founded safeguards for the proper unfolding of the process).
\end{enumerate}

If the conditions are understood as empirical hypotheses, the empirical challenge is to be able to empirically test them. If they are understood as rules (\ref{inter:rules}), the empirical challenge is to design institutions and procedures whose functioning ensures that the conditions are met. These two challenges come on top of two other, even more fundamental, empirical challenges: the one of identifying the empirically relevant set of arguments with which the analyst will have to work in practice, and the one of empirically elaborate models of deliberated judgments (which should ideally have some degree of generality)  and test them using the operational validity criterion.

To sum up, in this article, we have shown that the theoretical underpinnings of DMV are still plagued by two blind spots: the lack of a clear position on the \emph{role of the deliberation} and the \emph{role of the analyst} issues. We have proposed a framework addressing both issues, thereby strengthening the theoretical foundations of DMV and other deliberative approaches.  But we have also shown that this framework points major empirical challenges that deliberative methods should now address to entrench their credentials.

%That said, to conclude, we have to emphasize that, despite the promises offered by our framework, this is no magical tool. 
%In particular, we do not claim that it overcomes Hume's law by providing the means to deduce ought from is. 
%As repeatedly emphasised, our approach is anchored in two unmistakably normative tenets: rationality and anti-paternalism. %Our reasoning in \cref{disc} was devoted to show that these two tenets play a very fundamental role in contemporary normative philosophy, and can therefore in that very specific sense be considered minimal. 
%But one could accordingly argue that the criticism that we address at Habermas also applies to our framework, one step deeper. This is certainly true. 
%We do not claim that it solves all the philosophical difficulties of the theory of deliberative democracy, DMV and their foundations. 
%We rather attempted to introduce an approach that allows to go deeper than the current literature in the direction of providing strong philosophical foundations to empirical applications of philosophies of deliberation, such as DMV. 

\begin{acknowledgements}
We thank B. Bartkowski for very helpful comments.
\end{acknowledgements}

%TODO remove for final version
\hbadness 10000

\bibliography{beliefs,philo-eco,deliber,manual,add_bart}
\end{document}
