\newcommand{\commentOC}[1]{{\small\color{blue}$\big[$OC: #1$\big]$}}
\newcommand{\commentOCf}[1]{{\small\color{blue}{\selectlanguage{french}$\big[$OC : #1$\big]$}}}
\newcommand{\commentYM}[1]{{\small\color{red}{\selectlanguage{french}$\big[$YM : #1$\big]$}}}
\newcommand{\innote}[1]{{\scriptsize{#1}}}
%Or: \todo[color=green!40]

%this probably requires outdated float package, see doc KomaScript for an alternative.
% \newfloat{program}{t}{lop}
% \floatname{program}{PM}

%definition, theorem, lemma, example environments, qed trickery are only needed in article mode (not Beamer)
\nottoggle{LCpres}{
%style is plain by default (italic text)
	\newtheorem{definition}{Definition}
	\newtheorem{theorem}{Theorem}
%no italic: expected.
%http://tex.stackexchange.com/questions/144653/italicizing-of-amsthm-package
	\newtheorem{lemma}{Lemma}
	\newtheorem{condition}{Condition}
%\crefname{axiom}{axiom}{axioms}%might be needed for workaround bug in cref when defining new theorems?

%\ifdefined\theorem\else
%\newtheorem{theorem}{\iflanguage{english}{Theorem}{Théorème}}
%\fi

\theoremstyle{remark}
	\newtheorem{examplex}{Example}
	\newtheorem{remarkx}{Remark}

%trickery allowing use of \qedhere and the like.
\newenvironment{example}{
	\pushQED{\qed}\renewcommand{\qedsymbol}{$\triangle$}\examplex
}{
	\popQED\endexamplex
}
\newenvironment{remark}{
	\pushQED{\qed}\renewcommand{\qedsymbol}{$\triangle$}\remarkx
}{
	\popQED\endremarkx
}
}{
}
\crefname{examplex}{example}{examples}% I wonder why this is unnecessary in case of singular
\crefname{condition}{condition}{conditions}
\makeatletter
\cref@addlanguagedefs{french}{%
	\crefname{examplex}{exemple}{exemples}%
}
\makeatother

%trickery allowing use of \qedhere and the like.
%\newcommand{\xqed}[1]{%
%    \leavevmode\unskip\penalty9999 \hbox{}\nobreak\hfill
%    \quad\hbox{#1}}
%\AtEndEnvironment{example}{
%	\xqed{$\triangle$}%
%}
%\AtEndEnvironment{proof}{
%	\qed%
%}

%which line breaks are chosen: accept worse lines, therefore reducing risk of overfull lines. Default = 200
%\tolerance=2000
%accept overfull hbox up to...
%\hfuzz=2cm
%reduces verbosity about the bad line breaks
%\hbadness 5000
%sloppy sets tolerance to 9999
%\apptocmd{\sloppy}{\hbadness 10000\relax}{}{}

\bibliographystyle{abbrvnat}
%or \bibliographystyle{apalike} for presentations?

%doi package uses old-style dx.doi url, see 3.8 DOI system Proxy Server technical details, “Users may resolve DOI names that are structured to use the DOI system Proxy Server (http://doi.org (preferred) or http://dx.doi.org).”, https://www.doi.org/doi_handbook/3_Resolution.html
\makeatletter
\patchcmd{\@doi}{dx.doi.org}{doi.org}{}{}
\makeatother

% WRITING
%\newcommand{\ie}{i.e.\@\xspace}%to try
%\newcommand{\eg}{e.g.\@\xspace}
%\newcommand{\etal}{et al.\@\xspace}
\newcommand{\ie}{i.e.\ }
\newcommand{\eg}{e.g.\ }
\newcommand{\mkkOK}{\checkmark}%\color{green}{\checkmark}
\newcommand{\mkkREQ}{\ding{53}}%requires pifont?%\color{green}{\checkmark}
\newcommand{\mkkNO}{}%\text{\color{red}{\textsf{X}}}

\makeatletter
\newcommand{\boldor}[2]{%
	\ifnum\strcmp{\f@series}{bx}=\z@
		#1%
	\else
		#2%
	\fi
}
\newcommand{\textstyleElProm}[1]{\boldor{\MakeUppercase{#1}}{\textsc{#1}}}
\makeatother
\newcommand{\electre}{\textstyleElProm{Électre}\xspace}
\newcommand{\electreIv}{\textstyleElProm{Électre Iv}\xspace}
\newcommand{\electreIV}{\textstyleElProm{Électre IV}\xspace}
\newcommand{\electreIII}{\textstyleElProm{Électre III}\xspace}
\newcommand{\electreTRI}{\textstyleElProm{Électre Tri}\xspace}
% \newcommand{\utadis}{\texorpdfstring{\textstyleElProm{utadis}\xspace}{UTADIS}}
% \newcommand{\utadisI}{\texorpdfstring{\textstyleElProm{utadis i}\xspace}{UTADIS I}}

%TODO
% \newcommand{\textstyleElProm}[1]{{\rmfamily\textsc{#1}}} 

\newcommand{\txtlaxiom}{$\alllang$\hyp{}axiom}
\newcommand{\txtlaxioms}{$\alllang$\hyp{}axioms}
\newcommand{\laxiomatisation}{$\alllang$\hyp{}axiomatisation}
\newcommand{\laxiomatization}{$\alllang$\hyp{}axiomatization}
%bold. Requires special command for math bold cal!
\newcommand{\laxiomatisationbd}{$\mathbfcal{L}$\textbf{\hyp{}axiomatisation}}
%\newcommand{\laxiomatisationtitle}{\texorpdfstring{$\alllang$\hyp{}Axiomatisation}{L\hyp{}Axiomatisation}}%Does not work because of AAMAS font switching command (tries to typeset it using ptm font), so we should use something like {\usefont{EU1}{lmr}{m}{n}©} but manually select the right font size and weight (which I ignore).
\newcommand{\laxiomatisationtitle}{\texorpdfstring{{\large$\mathcal{L}$}\hyp{}Axiomatisation}{ℒ\hyp{}Axiomatisation}}
%\commentUE{Should we maybe write $\ell$-axiom rather than l-axiom, for better readability (to avoid ell being read as one)? Or we could even write $\mathcal{L}$-axiom, to have an explicit reference to the language in which the axiom is expressed?}

\newunicodechar{ℕ}{\mathbb{N}}
\newunicodechar{ℝ}{\mathbb{R}}
\newunicodechar{−}{\ifmmode{-}\else\textminus\fi}
\newunicodechar{≠}{\ensuremath{\neq}}
\newunicodechar{≤}{\ensuremath{\leq}}
\newunicodechar{≥}{\ensuremath{\geq}}
\newunicodechar{≻}{\succ}
\newunicodechar{⊁}{\nsucc}
\newunicodechar{▷}{\triangleright}
\newunicodechar{⋫}{\ntriangleright}
\newunicodechar{→}{\ifmmode\rightarrow\else\textrightarrow\fi}
\newunicodechar{⇒}{\ensuremath{\Rightarrow}}
\newunicodechar{⇏}{\ensuremath{\nRightarrow}}
\newunicodechar{⇔}{\ensuremath{\Leftrightarrow}}
\newunicodechar{∪}{\cup}
\newunicodechar{∩}{\cap}
\newunicodechar{∧}{\land}
\newunicodechar{∨}{\lor}
\newunicodechar{¬}{\ifmmode\lnot\else\textlnot\fi}
\newunicodechar{…}{\ifmmode\ldots\else\textellipsis\fi}
\newunicodechar{×}{\ifmmode\times\else\texttimes\fi}
\newunicodechar{γ}{\ensuremath{\gamma}}
\newunicodechar{□}{\Box}

